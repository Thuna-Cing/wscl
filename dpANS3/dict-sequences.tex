% -*- Mode: TeX -*-

% Sequences
%  Sequence Mapping
%  Sequence Counting
%  Sequence Ordering
%  Sequence Search
%  Sequence Comparison
%  Sequence Substitution
%  Sequence Joining
%  Sequence Deletion

%-------------------- Sequence Type --------------------

%%% ========== SEQUENCE
\begincom{sequence}\ftype{System Class}

\label Class Precedence List::
\typeref{sequence},
\typeref{t}

\label Description::

\term{Sequences} are ordered collections of \term{objects},
called the \term{elements} of the \term{sequence}.

%% 2.15.0 26                                                 
\Thetypes{vector} and \thetype{list} are \term{disjoint} \subtypesof{sequence},
%The following is added to make an implicit vagueness explicit. -kmp 29-Mar-91
but are not necessarily an \term{exhaustive partition} of \term{sequence}.

When viewing a \term{vector} as a \term{sequence}, 
only the \term{active} \term{elements} of that \term{vector} 
are considered \term{elements} of the \term{sequence};
that is,
\term{sequence} operations respect the \term{fill pointer}
when given \term{sequences} represented as \term{vectors}.

%% 2.4.0 2
%% 2.4.0 7

\endcom%{sequence}\ftype{System Class}

%-------------------- Sequences --------------------

%%% ========== COPY-SEQ
\begincom{copy-seq}\ftype{Function}

%KMP: It's too bad this function is not called copy-sequence.

\label Syntax::

\DefunWithValues copy-seq {sequence} {copied-sequence}

\label Arguments and Values::

\param{sequence}---a \term{proper sequence}.

\param{copied-sequence}---a \term{proper sequence}.

\label Description::

%% 14.1.0 8
Creates a copy of \param{sequence}.  The \term{elements} of the new
\term{sequence} are the \term{same} as the corresponding \term{elements} of
the given \param{sequence}.

%% Moon's suggested interpretation follows:

If \param{sequence} is a \term{vector}, 
the result is a \term{fresh} \term{simple array}
of \term{rank} one 
that has the same \term{actual array element type} as \param{sequence}.
If \param{sequence} is a \term{list}, 
the result is a \term{fresh} \term{list}. 

%% End Moon's suggested interpretation.

\label Examples::
\code
 (setq str "a string") \EV "a string"
 (equalp str (copy-seq str)) \EV \term{true}
 (eql str (copy-seq str)) \EV \term{false}
\endcode

\label Side Effects:\None.

%% Sandra thinks this is excessive.
%Creates a new \term{sequence}.

\label Affected By:\None.

\label Exceptional Situations::

\Lazychecktype{sequence}{a \term{proper sequence}}

\label See Also::

\funref{copy-list}

\label Notes::

From a functional standpoint,
\code
 (copy-seq x) \EQ (subseq x 0)
\endcode
However, the programmer intent is typically very different in these two cases.

\endcom

%%% ========== ELT
\begincom{elt}\ftype{Accessor}

\label Syntax::

\DefunWithValues elt {sequence index} {object}
\Defsetf         elt {sequence index} {new-object}

\label Arguments and Values::

\param{sequence}---a \term{proper sequence}.

\param{index}---a \term{valid sequence index} for \param{sequence}.

\param{object}---an \term{object}.

\param{new-object}---an \term{object}.

\label Description::

%% 14.1.0 3
\term{Accesses} the \term{element} of \param{sequence} specified by \param{index}.

% No longer needed -- implied by "Access". -kmp 13-Jan-92
% %% 14.1.0 5
% \macref{setf} may be used with \funref{elt} to destructively replace
% a \term{sequence} element with a new value.

%Barmar thinks (and I agree) that this is redundant with the specification
%of the argument type above.
%
% %% 14.1.0 4
% \funref{elt} observes the \term{fill pointer} in those \term{vectors} 
% that have \term{fill pointers}.  

\label Examples::

\code
 (setq str (copy-seq "0123456789")) \EV "0123456789"
 (elt str 6) \EV #\\6
 (setf (elt str 0) #\\#) \EV #\\#
 str \EV "#123456789"
\endcode

\label Side Effects:\None.

\label Affected By:\None.

\label Exceptional Situations::

%% e.g., consider:
%  (LET ((X (NCONC (MAKE-LIST 1000 :INITIAL-ELEMENT 'A) '(B . C))))
%    (ELT X 1000))
%  => A
\Lazychecktype{sequence}{a \term{proper sequence}}
\Shouldchecktype{index}{a \term{valid sequence index} for \param{sequence}}

\label See Also::                               

\funref{aref},
\funref{nth},
\issue{CONSTANT-MODIFICATION:DISALLOW}
{\secref\ConstantModification}
\endissue{CONSTANT-MODIFICATION:DISALLOW}

\label Notes::

\funref{aref} may be used to \term{access} \term{vector} 
elements that are beyond the \term{vector}'s \term{fill pointer}.

\endcom

%%% ========== FILL
\begincom{fill}\ftype{Function}

\label Syntax::

\DefunWithValues fill {sequence item {\key} start end} {sequence}

\label Arguments and Values::

\param{sequence}---a \term{proper sequence}.

\param{item}---a \term{sequence}.

\issue{SUBSEQ-OUT-OF-BOUNDS}
\issue{RANGE-OF-START-AND-END-PARAMETERS:INTEGER-AND-INTEGER-NIL}
\param{start}, \param{end}---\term{bounding index designators} of \param{sequence}.
 \Defaults{\param{start} and \param{end}}{\f{0} and \nil}
\endissue{RANGE-OF-START-AND-END-PARAMETERS:INTEGER-AND-INTEGER-NIL}
\endissue{SUBSEQ-OUT-OF-BOUNDS}
                        
\label Description::

%% 14.3.0 3
Replaces the \term{elements} of \param{sequence} 
\term{bounded} by \param{start} and \param{end}
with \param{item}.

\label Examples::

\code
 (fill (list 0 1 2 3 4 5) '(444)) \EV ((444) (444) (444) (444) (444) (444))
 (fill (copy-seq "01234") #\\e :start 3) \EV "012ee"
 (setq x (vector 'a 'b 'c 'd 'e)) \EV #(A B C D E)
 (fill x 'z :start 1 :end 3) \EV #(A Z Z D E)
 x \EV #(A Z Z D E)
 (fill x 'p) \EV #(P P P P P)
 x \EV #(P P P P P)
\endcode

\label Side Effects::

\param{Sequence} is destructively modified.

\label Affected By:\None.

\label Exceptional Situations::

\Lazychecktype{sequence}{a \term{proper sequence}} 
\Shouldchecktype{start}{a non-negative \term{integer}}
\Shouldchecktype{end}{a non-negative \term{integer} or \nil}

\label See Also::

\funref{replace}, \funref{nsubstitute}

\label Notes::

{\tt  (fill \param{sequence} \param{item}) \EQ 
   (nsubstitute-if \param{item} (constantly t) \param{sequence})}

\endcom

%%% ========== MAKE-SEQUENCE
\begincom{make-sequence}\ftype{Function}

\label Syntax::

\DefunWithValues make-sequence 
		 {result-type size {\key} initial-element}
		 {sequence}

\label Arguments and Values::

\issue{CONCATENATE-SEQUENCE:SIGNAL-ERROR}
\issue{ARGUMENTS-UNDERSPECIFIED:SPECIFY}
\param{result-type}---a \typeref{sequence} \term{type specifier}.
\endissue{ARGUMENTS-UNDERSPECIFIED:SPECIFY}
\endissue{CONCATENATE-SEQUENCE:SIGNAL-ERROR}

\issue{ARGUMENTS-UNDERSPECIFIED:SPECIFY}
\param{size}---a non-negative \term{integer}.
\endissue{ARGUMENTS-UNDERSPECIFIED:SPECIFY}

\param{initial-element}---an \term{object}.
 \Default{\term{implementation-dependent}}

\param{sequence}---a \term{proper sequence}.

\label Description::

%% 14.1.0 12
Returns a \term{sequence} of the type \param{result-type} and of length \param{size},
each of the \term{elements} of which has been initialized to \param{initial-element}.

\issue{CONCATENATE-SEQUENCE:SIGNAL-ERROR}
If the \param{result-type} is a \term{subtype} of \typeref{list},
the result will be a \term{list}.

If the \param{result-type} is a \term{subtype} of \typeref{vector},
then if the implementation can determine the element type specified
for the \param{result-type}, the element type of the resulting array 
is the result of \term{upgrading} that element type; or, if the
implementation can determine that the element type is unspecified (or \f{*}),
the element type of the resulting array is \typeref{t};
otherwise, an error is signaled.
\endissue{CONCATENATE-SEQUENCE:SIGNAL-ERROR}

\label Examples::

\code
 (make-sequence 'list 0) \EV ()
 (make-sequence 'string 26 :initial-element #\\.) 
\EV ".........................."
 (make-sequence '(vector double-float) 2
                :initial-element 1d0)
\EV #(1.0d0 1.0d0)
\endcode

\issue{SEQUENCE-TYPE-LENGTH:MUST-MATCH}
\code
 (make-sequence '(vector * 2) 3) should signal an error
 (make-sequence '(vector * 4) 3) should signal an error
\endcode
\endissue{SEQUENCE-TYPE-LENGTH:MUST-MATCH}


\label Affected By::

The \term{implementation}.

\label Exceptional Situations::

The consequences are unspecified if \param{initial-element} 
is not an \term{object} which can be stored in the resulting \term{sequence}.

\issue{CONCATENATE-SEQUENCE:SIGNAL-ERROR}
An error \oftype{type-error} must be signaled if the \param{result-type} is neither
     a \term{recognizable subtype} of \typeref{list},
 nor a \term{recognizable subtype} of \typeref{vector}.
\endissue{CONCATENATE-SEQUENCE:SIGNAL-ERROR}

\issue{SEQUENCE-TYPE-LENGTH:MUST-MATCH}
An error \oftype{type-error} should be signaled if \param{result-type} specifies 
the number of elements and \param{size} is different from that number.
\endissue{SEQUENCE-TYPE-LENGTH:MUST-MATCH}

\label See Also::

\funref{make-array}, \funref{make-list}

\label Notes::
\issue{CHARACTER-PROPOSAL:2-6-5}
\code
 (make-sequence 'string 5) \EQ (make-string 5)               
\endcode
\endissue{CHARACTER-PROPOSAL:2-6-5}
\endcom

%%% ========== SUBSEQ
\begincom{subseq}\ftype{Accessor}

%KMP: It's too bad this function is not called subsequence.

\label Syntax::

\DefunWithValues subseq {sequence start {\opt} end} {subsequence}
\Defsetf         subseq {sequence start {\opt} end} {new-subsequence}

\label Arguments and Values::

\param{sequence}---a \term{proper sequence}.

\issue{SUBSEQ-OUT-OF-BOUNDS}
\issue{RANGE-OF-START-AND-END-PARAMETERS:INTEGER-AND-INTEGER-NIL}
\param{start}, \param{end}---\term{bounding index designators} of \param{sequence}.
 \DefaultFor{\param{end}}{\nil}
\endissue{RANGE-OF-START-AND-END-PARAMETERS:INTEGER-AND-INTEGER-NIL}
\endissue{SUBSEQ-OUT-OF-BOUNDS}

\param{subsequence}---a \term{proper sequence}.

\param{new-subsequence}---a \term{proper sequence}.

\label Description::

%% 14.1.0 6
\funref{subseq} creates a \term{sequence} 
that is a copy of the subsequence of \param{sequence}
\param{bounded} by \param{start} and \param{end}.

\param{Start} specifies an offset into the original \param{sequence} and
marks the beginning position of the subsequence.
\param{end} marks the position following the last element of the subsequence.
                                                
\funref{subseq} always allocates a new \term{sequence} for a result;
it never shares storage with an old \term{sequence}. 
The result subsequence is always of the same \term{type} as \param{sequence}.

%% Moon's suggested interpretation follows:

If \param{sequence} is a \term{vector},
the result is a \term{fresh} \term{simple array}
of \term{rank} one
that has the same \term{actual array element type} as \param{sequence}.
If \param{sequence} is a \term{list}, 
the result is a \term{fresh} \term{list}. 

%% End Moon's suggested interpretation.

%% 14.1.0 7
\macref{setf} may be used with \funref{subseq} to destructively replace
\term{elements} of a subsequence with \term{elements} 
taken from a \term{sequence} of new values.
If the subsequence and the new sequence are not of equal length,
the shorter length determines the number of elements that are
replaced.  The remaining \term{elements} at the end of the longer sequence 
are not modified in the operation.

\label Examples::

\code
 (setq str "012345") \EV "012345"
 (subseq str 2) \EV "2345"
 (subseq str 3 5) \EV "34"
 (setf (subseq str 4) "abc") \EV "abc"
 str \EV "0123ab"
 (setf (subseq str 0 2) "A") \EV "A"
 str \EV "A123ab"
\endcode


\label Side Effects:\None.

\label Affected By:\None.

\label Exceptional Situations::

\Lazychecktype{sequence}{a \term{proper sequence}}
\Lazychecktype{new-subsequence}{a \term{proper sequence}}

\label See Also::

\funref{replace}

\label Notes:\None.

\endcom

%-------------------- Sequence Mapping --------------------

%%% ========== MAP
\begincom{map}\ftype{Function}

\label Syntax::

\DefunWithValues map {result-type function {\rest} \plus{sequences}} {result}

\label Arguments and Values:: 

\issue{CONCATENATE-SEQUENCE:SIGNAL-ERROR}
\param{result-type}---a \typeref{sequence} \term{type specifier}, or \nil.
\endissue{CONCATENATE-SEQUENCE:SIGNAL-ERROR}

\issue{FUNCTION-TYPE:X3J13-MARCH-88}
\param{function}---a \term{function designator}.
  \param{function} must take as many arguments as 
  there are \param{sequences}.
\endissue{FUNCTION-TYPE:X3J13-MARCH-88}

\param{sequence}---a \term{proper sequence}.

\param{result}---if \param{result-type} is a \term{type specifier} other than \nil, 
		 then a \term{sequence} of the \term{type} it denotes;
		 otherwise (if the \param{result-type} is \nil), \nil.

\label Description::

%% 7.8.4 3
Applies \param{function} to successive sets of arguments in which
one argument is obtained from each \term{sequence}.
%% 14.2.0 6
The \param{function} is called first on all the elements with index \f{0},
then on all those with index \f{1}, and so on.
The \param{result-type} specifies the \term{type} of the resulting \term{sequence}.

\funref{map} returns \nil\ if \param{result-type} is \nil.      
%% 14.2.0 5
Otherwise, \funref{map} returns
a \term{sequence} such that element \f{j} is the result
of applying \param{function} to element \f{j} of each of the 
\param{sequences}.  The result \term{sequence} 
is as long as the shortest of the
\param{sequences}.
The consequences are undefined if the result of applying \param{function} 
to the successive elements of the \param{sequences} cannot
be contained in a \term{sequence} of the \term{type} given by \param{result-type}.

\issue{CONCATENATE-SEQUENCE:SIGNAL-ERROR}
If the \param{result-type} is a \term{subtype} of \typeref{list},
the result will be a \term{list}.

If the \param{result-type} is a \term{subtype} of \typeref{vector},
then if the implementation can determine the element type specified
for the \param{result-type}, the element type of the resulting array 
is the result of \term{upgrading} that element type; or, if the
implementation can determine that the element type is unspecified (or \f{*}),
the element type of the resulting array is \typeref{t};
otherwise, an error is signaled.
\endissue{CONCATENATE-SEQUENCE:SIGNAL-ERROR}

\label Examples::

\code
 (map 'string #'(lambda (x y)
                  (char "01234567890ABCDEF" (mod (+ x y) 16)))
       '(1 2 3 4)
       '(10 9 8 7)) \EV "AAAA"
 (setq seq '("lower" "UPPER" "" "123")) \EV ("lower" "UPPER" "" "123")
 (map nil #'nstring-upcase seq) \EV NIL
 seq \EV ("LOWER" "UPPER" "" "123")
 (map 'list #'- '(1 2 3 4)) \EV (-1 -2 -3 -4)
 (map 'string
      #'(lambda (x) (if (oddp x) #\\1 #\\0))
      '(1 2 3 4)) \EV "1010"
\endcode

\issue{SEQUENCE-TYPE-LENGTH:MUST-MATCH}
\code
 (map '(vector * 4) #'cons "abc" "de") should signal an error
\endcode
\endissue{SEQUENCE-TYPE-LENGTH:MUST-MATCH}

\label Affected By:\None.

\label Exceptional Situations::

\issue{CONCATENATE-SEQUENCE:SIGNAL-ERROR}
An error \oftype{type-error} must be signaled if the \param{result-type} is 
     not a \term{recognizable subtype} of \typeref{list},
     not a \term{recognizable subtype} of \typeref{vector},
 and not \nil.
\endissue{CONCATENATE-SEQUENCE:SIGNAL-ERROR}

\Lazycheckanytype{sequence}{a \term{proper sequence}}

\issue{SEQUENCE-TYPE-LENGTH:MUST-MATCH}
An error \oftype{type-error} should be signaled
if \param{result-type} specifies the
number of elements and the minimum length of the \param{sequences} 
is different from that number.
\endissue{SEQUENCE-TYPE-LENGTH:MUST-MATCH}

\label See Also::

\issue{MAPPING-DESTRUCTIVE-INTERACTION:EXPLICITLY-VAGUE}
{\secref\TraversalRules}
\endissue{MAPPING-DESTRUCTIVE-INTERACTION:EXPLICITLY-VAGUE}

\label Notes:\None.

\endcom


%%% ========== MAP-INTO
\begincom{map-into}\ftype{Function}

\issue{MAP-INTO:ADD-FUNCTION}

\label Syntax::

\DefunWithValues map-into {result-sequence function {\rest} sequences} {result-sequence}

\label Arguments and Values::

\param{result-sequence}---a \term{proper sequence}.

\param{function}---a \term{designator} for a \term{function}
		   of as many \term{arguments} as there are \param{sequences}.

\param{sequence}---a \term{proper sequence}.

\label Description::

Destructively modifies \param{result-sequence} to contain the results of
applying \param{function} to each element in the argument \param{sequences} 
in turn.
 
\param{result-sequence} and each element of \param{sequences} can each be
either a \term{list} or a \term{vector}. 
%Note that NIL is considered to be a sequence, of length zero.  
If \param{result-sequence} and each element of \param{sequences} are not all
the same length, the iteration terminates when the shortest \term{sequence}
(of any of the \param{sequences} or the \param{result-sequence})
is exhausted.
If \param{result-sequence} is a \term{vector} with a 
\term{fill pointer}, the \term{fill pointer} is ignored when deciding how
many iterations to perform, and afterwards the \term{fill pointer} is set to
the number of times \param{function} was applied.
If \param{result-sequence} is longer than the shortest element of \param{sequences},
extra elements at the end of \param{result-sequence} are left unchanged.
If \param{result-sequence} is \nil, \funref{map-into} immediately returns
\nil, since \nil\ is a \term{sequence} of length zero.
 
If \param{function} has side effects, it can count on being called
first on all of the elements with index 0, then on all of those 
numbered 1, and so on.
 
\label Examples::

\code
 (setq a (list 1 2 3 4) b (list 10 10 10 10)) \EV (10 10 10 10)
 (map-into a #'+ a b) \EV (11 12 13 14)
 a \EV (11 12 13 14)
 b \EV (10 10 10 10)
 (setq k '(one two three)) \EV (ONE TWO THREE)
 (map-into a #'cons k a) \EV ((ONE . 11) (TWO . 12) (THREE . 13) 14)
 (map-into a #'gensym) \EV (#:G9090 #:G9091 #:G9092 #:G9093)
 a \EV (#:G9090 #:G9091 #:G9092 #:G9093)
\endcode


\label Affected By:\None.                                                 

\label Exceptional Situations::

\Lazychecktype{result-sequence}{a \term{proper sequence}}
\Lazychecktype{sequence}{a \term{proper sequence}}

\label See Also:\None.

\label Notes::

\funref{map-into} differs from \funref{map} in that it modifies an
existing \term{sequence} rather than creating a new one.
In addition, \funref{map-into} can be called with only two 
arguments, while \funref{map} requires at least three arguments.

%% Barmar supplied this.
\funref{map-into} could be defined by:

\code
 (defun map-into (result-sequence function &rest sequences)
   (loop for index below (apply #'min 
                                (length result-sequence)
                                (mapcar #'length sequences))
         do (setf (elt result-sequence index)
                  (apply function
                         (mapcar #'(lambda (seq) (elt seq index))
                                 sequences))))
   result-sequence)
\endcode

\endissue{MAP-INTO:ADD-FUNCTION}
\endcom              

%%% ========== REDUCE
\begincom{reduce}\ftype{Function}

\label Syntax::

\DefunWithValues reduce {function sequence {\key} key from-end start end initial-value}
		 	{result}

\label Arguments and Values::

\param{function}---a \term{designator} for a \term{function}
   that might be called with either zero or two \term{arguments}.

\param{sequence}---a \term{proper sequence}.

\issue{REDUCE-ARGUMENT-EXTRACTION}
\param{key}---a \term{designator} for a \term{function} of one argument,
  or \nil.
\endissue{REDUCE-ARGUMENT-EXTRACTION}

\param{from-end}---a \term{generalized boolean}.
  \Default{\term{false}}
 
\issue{SUBSEQ-OUT-OF-BOUNDS}
\issue{RANGE-OF-START-AND-END-PARAMETERS:INTEGER-AND-INTEGER-NIL}
\param{start}, \param{end}---\term{bounding index designators} of \param{sequence}.
 \Defaults{\param{start} and \param{end}}{\f{0} and \nil}
\endissue{RANGE-OF-START-AND-END-PARAMETERS:INTEGER-AND-INTEGER-NIL}
\endissue{SUBSEQ-OUT-OF-BOUNDS}

\param{initial-value}---an \term{object}. 

\param{result}---an \term{object}.

\label Description::

%% 14.2.0 17
%% 14.2.0 18
\funref{reduce} uses a binary operation, \param{function},
to combine the \term{elements} of \param{sequence} 
\term{bounded} by \param{start} and \param{end}.

The \param{function} must accept as \term{arguments} two \term{elements}
of \param{sequence} or the results from combining those \term{elements}.
The \param{function} must also be able to accept no arguments.

\issue{REDUCE-ARGUMENT-EXTRACTION}
If \param{key} is supplied, it is used is used to extract the values to reduce.
The \param{key} function is applied exactly once to each element of \param{sequence}
in the order implied by the reduction order but not to the value of
\param{initial-value}, if supplied.
\endissue{REDUCE-ARGUMENT-EXTRACTION}
The \param{key} function typically returns part of the \term{element} of \param{sequence}.
If \param{key} is not supplied or is \nil, the \param{sequence} \term{element} itself is used.

The reduction is left-associative,
unless \param{from-end} is \term{true} in which case it is right-associative.  

If \param{initial-value} is supplied, 
it is logically placed before the subsequence
(or after it if \param{from-end} is \term{true})
and included in the reduction operation.

In the normal case, the result of \funref{reduce} is the combined 
result of \param{function}'s being applied to successive pairs of \term{elements} 
of \param{sequence}.
%% 14.2.0 19
If the subsequence contains exactly one \term{element} 
and no \param{initial-value} is given,
then that \term{element} is returned and \param{function} is not called.
If the subsequence is empty and an \param{initial-value} is given,
then the \param{initial-value} is returned and \param{function} is not called.
%% 14.2.0 20
If the subsequence is empty and no \param{initial-value} is given,
then the \param{function} is called with zero arguments,
and \funref{reduce} returns whatever \param{function} does.
This is the only case where the
\param{function} is called with other than two arguments.

\label Examples::
%% 14.2.0 21
\code
 (reduce #'* '(1 2 3 4 5)) \EV 120
 (reduce #'append '((1) (2)) :initial-value '(i n i t)) \EV (I N I T 1 2)
 (reduce #'append '((1) (2)) :from-end t                  
                             :initial-value '(i n i t)) \EV (1 2 I N I T) 
 (reduce #'- '(1 2 3 4)) \EQ (- (- (- 1 2) 3) 4) \EV -8
 (reduce #'- '(1 2 3 4) :from-end t)    ;Alternating sum.
\EQ (- 1 (- 2 (- 3 4))) \EV -2
 (reduce #'+ '()) \EV 0
 (reduce #'+ '(3)) \EV 3
 (reduce #'+ '(foo)) \EV FOO
 (reduce #'list '(1 2 3 4)) \EV (((1 2) 3) 4)
 (reduce #'list '(1 2 3 4) :from-end t) \EV (1 (2 (3 4)))
 (reduce #'list '(1 2 3 4) :initial-value 'foo) \EV ((((foo 1) 2) 3) 4)
 (reduce #'list '(1 2 3 4)
        :from-end t :initial-value 'foo) \EV (1 (2 (3 (4 foo))))
\endcode

\label Side Effects:\None.

\label Affected By:\None.

\label Exceptional Situations::

\Lazychecktype{sequence}{a \term{proper sequence}}

\label See Also::

\issue{MAPPING-DESTRUCTIVE-INTERACTION:EXPLICITLY-VAGUE}
{\secref\TraversalRules}
\endissue{MAPPING-DESTRUCTIVE-INTERACTION:EXPLICITLY-VAGUE}

\label Notes:\None.
                           
\endcom

%-------------------- Sequence Counting --------------------

%%% ========== COUNT
%%% ========== COUNT-IF
%%% ========== COUNT-IF-NOT
\begincom{count, count-if, count-if-not}\ftype{Function}

\label Syntax::

\DefunWithValues count        {item sequence {\key} from-end start end key test test-not} {n}
\DefunWithValues count-if     {predicate sequence {\key} from-end start end key}  {n}
\DefunWithValues count-if-not {predicate sequence {\key} from-end start end key}  {n}

\label Arguments and Values:: 

\param{item}---an \term{object}.

\param{sequence}---a \term{proper sequence}.

\param{predicate}---a \term{designator} for a \term{function} of one \term{argument}
  that returns a \term{generalized boolean}.

\param{from-end}---a \term{generalized boolean}.
  \Default{\term{false}}

\param{test}---a \term{designator} for a \term{function} of two \term{arguments}
  that returns a \term{generalized boolean}.

\param{test-not}---a \term{designator} for 
  a \term{function} of two \term{arguments}
  that returns a \term{generalized boolean}.

\issue{SUBSEQ-OUT-OF-BOUNDS}
\issue{RANGE-OF-START-AND-END-PARAMETERS:INTEGER-AND-INTEGER-NIL}
\param{start}, \param{end}---\term{bounding index designators} of \param{sequence}.
 \Defaults{\param{start} and \param{end}}{\f{0} and \nil}
\endissue{RANGE-OF-START-AND-END-PARAMETERS:INTEGER-AND-INTEGER-NIL}
\endissue{SUBSEQ-OUT-OF-BOUNDS}

\param{key}---a \term{designator} for a \term{function} of one argument,
  or \nil.

\param{n}---a non-negative \term{integer} 
	    less than or equal to the \term{length} of \param{sequence}.

\label Description::

%% 14.3.0 32
\funref{count}, \funref{count-if}, and \funref{count-if-not} 
count and return the number of \term{elements} in 
the \param{sequence} \term{bounded} by \param{start} and \param{end}
that \term{satisfy the test}.

The \param{from-end} has no direct effect on the result.
However, if \param{from-end} is \term{true},
the \term{elements} of \param{sequence} will be supplied as \term{arguments} to
      the \param{test},
      \param{test-not},
  and \param{key} in reverse order,
which may change the side-effects, if any, of those functions.

\label Examples::

\code
 (count #\\a "how many A's are there in here?") \EV 2
 (count-if-not #'oddp '((1) (2) (3) (4)) :key #'car) \EV 2
 (count-if #'upper-case-p "The Crying of Lot 49" :start 4) \EV 2 
\endcode

\label Side Effects:\None.

\label Affected By:\None.

\label Exceptional Situations::

\issue{TEST-NOT-PROGRAM-ERROR:SIGNAL-ERROR-IN-SAFE-CODE}
\Lazychecktype{sequence}{a \term{proper sequence}}
\endissue{TEST-NOT-PROGRAM-ERROR:SIGNAL-ERROR-IN-SAFE-CODE}

\label See Also::

{\secref\TestFunctionRules},
\issue{MAPPING-DESTRUCTIVE-INTERACTION:EXPLICITLY-VAGUE}
{\secref\TraversalRules}
\endissue{MAPPING-DESTRUCTIVE-INTERACTION:EXPLICITLY-VAGUE}

\label Notes::

\issue{TEST-NOT-IF-NOT:FLUSH-ALL}
The \kwd{test-not} \term{argument} is deprecated.
\endissue{TEST-NOT-IF-NOT:FLUSH-ALL}

\issue{TEST-NOT-IF-NOT:FLUSH-ALL}
\Thefunction{count-if-not} is deprecated.
\endissue{TEST-NOT-IF-NOT:FLUSH-ALL}

\endcom

%%% ========== LENGTH
\begincom{length}\ftype{Function}

\label Syntax::

\DefunWithValues length {sequence} {n}

\label Arguments and Values::

\param{sequence}---a \term{proper sequence}.

\param{n}---a non-negative \term{integer}.

\label Description::

%% 14.1.0 9
Returns the number of \term{elements} in \param{sequence}.

If \param{sequence} is a \term{vector} with a \term{fill pointer},
the active length as specified by the \term{fill pointer} is returned.

\label Examples::

\code
 (length "abc") \EV 3
 (setq str (make-array '(3) :element-type 'character 
                            :initial-contents "abc"
                            :fill-pointer t)) \EV "abc"
 (length str) \EV 3
 (setf (fill-pointer str) 2) \EV 2
 (length str) \EV 2
\endcode

\label Affected By:\None.

\label Exceptional Situations::

\Lazychecktype{sequence}{a \term{proper sequence}}
%%It's implied by lazychecktype.
%The results are undefined if \param{sequence} is a \term{circular list}.

\label See Also::

\funref{list-length},
\typeref{sequence}

%% Per X3J13. -kmp 05-Oct-93
\label Notes:\None.                

\endcom

%-------------------- Sequence Ordering --------------------

%%% ========== REVERSE
%%% ========== NREVERSE
\begincom{reverse, nreverse}\ftype{Function}

\label Syntax::

\DefunWithValues reverse {sequence} {reversed-sequence}

\DefunWithValues nreverse {sequence} {reversed-sequence}

\label Arguments and Values::

\param{sequence}---a \term{proper sequence}.

%% 14.1.0 10
\param{reversed-sequence}---a \term{sequence}.

\label Description::

\funref{reverse} and \funref{nreverse} return a new \term{sequence} 
of the same kind as \param{sequence}, containing the same \term{elements},
but in reverse order.

\funref{reverse} and \funref{nreverse} differ in that \funref{reverse} 
always creates and returns a new \term{sequence}, whereas \funref{nreverse}
might modify and return the given \param{sequence}.  \funref{reverse} never
modifies the given \param{sequence}.

%% Moon's suggested interpretation follows:

For \funref{reverse}, if \param{sequence} is a \term{vector}, 
the result is a \term{fresh} \term{simple array} of \term{rank} one
that has the same \term{actual array element type} as \param{sequence}.
If \param{sequence} is a \term{list}, the result is a \term{fresh} \term{list}. 

For \funref{nreverse}, if \param{sequence} is a \term{vector},
the result is a \term{vector}
that has the same \term{actual array element type} as \param{sequence}.
If \param{sequence} is a \term{list}, the result is a \term{list}.

%% End Moon's suggested interpretation.

%% 14.1.0 11
For \funref{nreverse},
\param{sequence} might be destroyed and re-used to produce the result.
The result might or might not be \term{identical} to \param{sequence}.
\issue{REMF-DESTRUCTION-UNSPECIFIED:X3J13-MAR-89}
Specifically, when \param{sequence} is a \term{list}, 
\funref{nreverse} is permitted to \macref{setf} any part, \funref{car} or \funref{cdr},
of any \term{cons} that is part of the \term{list structure} of \param{sequence}.
When \param{sequence} is a \term{vector},
\funref{nreverse} is permitted to re-order the elements of \param{sequence}
in order to produce the resulting \term{vector}.
\endissue{REMF-DESTRUCTION-UNSPECIFIED:X3J13-MAR-89}

\label Examples::
\code
 (setq str "abc") \EV "abc"
 (reverse str) \EV "cba"
 str \EV "abc"
 (setq str (copy-seq str)) \EV "abc"
 (nreverse str) \EV "cba"
 str \EV \term{implementation-dependent}
 (setq l (list 1 2 3)) \EV (1 2 3)
 (nreverse l) \EV (3 2 1)
 l \EV \term{implementation-dependent}
\endcode

\label Side Effects::

\funref{nreverse} might either create a new \term{sequence},
modify the argument \param{sequence}, or both.
%\funref{reverse} returns a \term{potential copy} of \param{sequence}.
(\funref{reverse} does not modify \param{sequence}.)

\label Affected By:\None.

\label Exceptional Situations::

\Lazychecktype{sequence}{a \term{proper sequence}}

\label See Also:\None.

\label Notes:\None.
                
\endcom

%%% ========== SORT
%%% ========== STABLE-SORT
\begincom{sort, stable-sort}\ftype{Function}

\label Syntax::

\DefunWithValues sort        {sequence predicate {\key} key} {sorted-sequence}
\DefunWithValues stable-sort {sequence predicate {\key} key} {sorted-sequence}

\label Arguments and Values::

\param{sequence}---a \term{proper sequence}.

\param{predicate}---a \term{designator} for
  a \term{function} of two arguments that returns a \term{generalized boolean}.

%% 14.4.0 4
\param{key}---a \term{designator} for a \term{function} of one argument,
  or \nil.

\param{sorted-sequence}---a \term{sequence}.

\label Description::

%% 14.4.0 2
\funref{sort} and \funref{stable-sort} destructively sort \param{sequences}  
according to the order determined by the \param{predicate} function.

%% Moon's suggested interpretation follows:

If \param{sequence} is a \term{vector},
the result is a \term{vector} 
that has the same \term{actual array element type} as \param{sequence}.
%% Moved to notes per Schulenburg #1 (by X3J13 vote at May 4-5, 1994 meeting).
%% -kmp 9-May-94
% The result might or might not be simple, 
% and might or might not be \term{identical} to \param{sequence}.
%% Another possible interpretation:
%% The result might or might not be \term{identical} to \param{sequence}.
%% If the result is  a \term{vector} that is not \term{identical}
%% to \param{sequence}, the result is a \term{fresh} \term{simple array}
%% of \term{rank} one.
If \param{sequence} is a \term{list},
the result is a \term{list}. 
%% End Moon's suggested interpretation.

%% 14.4.0 3
\funref{sort} determines the relationship between two elements
by giving keys extracted from the elements to the \param{predicate}.
The first argument to the \param{predicate} function is the part of one element
of \param{sequence} extracted by the \param{key} function
(if supplied); the second
argument is the part of another element
of \param{sequence} extracted by the \param{key} function
(if supplied).
\param{Predicate} should return \term{true} if and only if the first argument is
strictly less than the second (in some appropriate sense). 
If the first argument is greater than or equal to the second
(in the appropriate sense), then the \param{predicate} should return \term{false}.
           
The argument to the \param{key} function is the \param{sequence} element.
The return value of the \param{key} function 
becomes an argument to \param{predicate}.
If \param{key} is not supplied or \nil, the \param{sequence} element itself is used.
There is no guarantee on the number of times the \param{key} will be called.

%% 14.4.0 5
If the \param{key} and \param{predicate} always return,
then the sorting operation will always terminate,
producing a \term{sequence} containing the same \term{elements} as \param{sequence} 
(that is, the result is a permutation of \param{sequence}).
This is guaranteed even if the \param{predicate}
does not really consistently represent a total order
(in which case the \term{elements} will be scrambled in some unpredictable way,
but no \term{element} will be lost).  
If the \param{key} consistently returns meaningful keys,
and the \param{predicate} does reflect some total ordering criterion on those keys,
then the \term{elements} of the \param{sorted-sequence}
will be properly sorted according to that ordering.

%% 14.4.0 6
The sorting operation performed by \funref{sort} is not guaranteed stable.
Elements considered equal by the \param{predicate} might or might not
stay in their original order.  The \param{predicate} is assumed to
consider two elements \f{x} and \f{y} to be equal if
\f{(funcall \i{predicate} \i{x} \i{y})} and
\f{(funcall \i{predicate} \i{y} \i{x})} are both \term{false}.
\funref{stable-sort} guarantees stability.

%% 14.5.0 7
The sorting operation can be destructive in all cases.  In the case of a
\term{vector} 
argument, this is accomplished by permuting the elements in place.
In the case of a \term{list}, the \term{list} is
destructively reordered in the same manner as for
\funref{nreverse}.  

%% 14.4.0 9
%% this paragraph left out

\label Examples::

\code
 (setq tester (copy-seq "lkjashd")) \EV "lkjashd"
 (sort tester #'char-lessp) \EV "adhjkls"
 (setq tester (list '(1 2 3) '(4 5 6) '(7 8 9))) \EV ((1 2 3) (4 5 6) (7 8 9))
 (sort tester #'> :key #'car)  \EV ((7 8 9) (4 5 6) (1 2 3)) 
 (setq tester (list 1 2 3 4 5 6 7 8 9 0)) \EV (1 2 3 4 5 6 7 8 9 0)
 (stable-sort tester #'(lambda (x y) (and (oddp x) (evenp y))))
\EV (1 3 5 7 9 2 4 6 8 0)
 (sort (setq committee-data
             (vector (list (list "JonL" "White") "Iteration")
                     (list (list "Dick" "Waters") "Iteration")
                     (list (list "Dick" "Gabriel") "Objects")
                     (list (list "Kent" "Pitman") "Conditions")
                     (list (list "Gregor" "Kiczales") "Objects")
                     (list (list "David" "Moon") "Objects")
                     (list (list "Kathy" "Chapman") "Editorial")
                     (list (list "Larry" "Masinter") "Cleanup")
                     (list (list "Sandra" "Loosemore") "Compiler")))
       #'string-lessp :key #'cadar)
\EV #((("Kathy" "Chapman") "Editorial")
     (("Dick" "Gabriel") "Objects")
     (("Gregor" "Kiczales") "Objects")
     (("Sandra" "Loosemore") "Compiler")
     (("Larry" "Masinter") "Cleanup")
     (("David" "Moon") "Objects")
     (("Kent" "Pitman") "Conditions")
     (("Dick" "Waters") "Iteration")
     (("JonL" "White") "Iteration"))
 ;; Note that individual alphabetical order within `committees'
 ;; is preserved.
 (setq committee-data 
       (stable-sort committee-data #'string-lessp :key #'cadr))
\EV #((("Larry" "Masinter") "Cleanup")
     (("Sandra" "Loosemore") "Compiler")
     (("Kent" "Pitman") "Conditions")
     (("Kathy" "Chapman") "Editorial")
     (("Dick" "Waters") "Iteration")
     (("JonL" "White") "Iteration")
     (("Dick" "Gabriel") "Objects")
     (("Gregor" "Kiczales") "Objects")
     (("David" "Moon") "Objects"))
\endcode

\label Side Effects:\None.

\label Affected By:\None.

\label Exceptional Situations::

\Lazychecktype{sequence}{a \term{proper sequence}}

\label See Also::

\funref{merge},
\issue{CONSTANT-MODIFICATION:DISALLOW}
{\secref\ConstantModification},
\endissue{CONSTANT-MODIFICATION:DISALLOW}
\issue{MAPPING-DESTRUCTIVE-INTERACTION:EXPLICITLY-VAGUE}
{\secref\TraversalRules},
\endissue{MAPPING-DESTRUCTIVE-INTERACTION:EXPLICITLY-VAGUE}
{\secref\DestructiveOperations}

\label Notes::

%% Moved from Description per Schulenburg #1 (by X3J13 vote at May 4-5, 1994 meeting).
%% -kmp 9-May-94
If \param{sequence} is a \term{vector},
the result might or might not be simple, 
and might or might not be \term{identical} to \param{sequence}.

\endcom

%-------------------- Sequence Search --------------------

%%% ========== FIND
%%% ========== FIND-IF
%%% ========== FIND-IF-NOT
\begincom{find, find-if, find-if-not}\ftype{Function}

\label Syntax::

\DefunWithValues find {item sequence {\key} from-end test test-not start end key} {element}
\DefunWithValues find-if     {predicate sequence {\key} from-end start end key} {element}
\DefunWithValues find-if-not {predicate sequence {\key} from-end start end key} {element}

\label Arguments and Values:: 

\param{item}---an \term{object}.

\param{sequence}---a \term{proper sequence}.

\param{predicate}---a \term{designator} for a \term{function} of one \term{argument}
  that returns a \term{generalized boolean}.

\param{from-end}---a \term{generalized boolean}.
  \Default{\term{false}}

\param{test}---a \term{designator} for a \term{function} of two \term{arguments}
  that returns a \term{generalized boolean}.

\param{test-not}---a \term{designator} for 
  a \term{function} of two \term{arguments}
  that returns a \term{generalized boolean}.

\issue{SUBSEQ-OUT-OF-BOUNDS}
\issue{RANGE-OF-START-AND-END-PARAMETERS:INTEGER-AND-INTEGER-NIL}
\param{start}, \param{end}---\term{bounding index designators} of \param{sequence}.
 \Defaults{\param{start} and \param{end}}{\f{0} and \nil}
\endissue{RANGE-OF-START-AND-END-PARAMETERS:INTEGER-AND-INTEGER-NIL}
\endissue{SUBSEQ-OUT-OF-BOUNDS}

\param{key}---a \term{designator} for a \term{function} of one argument,
  or \nil.

\param{element}---an \term{element} of the \param{sequence}, or \nil.

\label Description::

\funref{find}, \funref{find-if}, and \funref{find-if-not} 
each search for an \term{element} of the \param{sequence}
\term{bounded} by \param{start} and \term{end}
   that \term{satisfies the predicate} \param{predicate} 
or that \term{satisfies the test} \param{test} or \param{test-not},
as appropriate.
                  
%% 14.3.0 28
If \param{from-end} is \term{true},
then the result is the rightmost \term{element} that \term{satisfies the test}. 

%% 14.3.0 26
If the \param{sequence} contains an \term{element} that \term{satisfies the test},
then the leftmost or rightmost \param{sequence} element, 
depending on \param{from-end},
is returned;
otherwise \nil\ is returned.

\label Examples::

\code
 (find #\\d "here are some letters that can be looked at" :test #'char>)
\EV #\\Space 
 (find-if #'oddp '(1 2 3 4 5) :end 3 :from-end t) \EV 3
 (find-if-not #'complexp                                    
             '#(3.5 2 #C(1.0 0.0) #C(0.0 1.0))
             :start 2) \EV NIL 
\endcode

\label Side Effects:\None.

\label Affected By:\None.

\label Exceptional Situations::

\issue{TEST-NOT-PROGRAM-ERROR:SIGNAL-ERROR-IN-SAFE-CODE}
\Lazychecktype{sequence}{a \term{proper sequence}}
\endissue{TEST-NOT-PROGRAM-ERROR:SIGNAL-ERROR-IN-SAFE-CODE}

\label See Also::

\funref{position},
{\secref\TestFunctionRules},
\issue{MAPPING-DESTRUCTIVE-INTERACTION:EXPLICITLY-VAGUE}
{\secref\TraversalRules}
\endissue{MAPPING-DESTRUCTIVE-INTERACTION:EXPLICITLY-VAGUE}

\label Notes::

\issue{TEST-NOT-IF-NOT:FLUSH-ALL}
The \kwd{test-not} \term{argument} is deprecated.
\endissue{TEST-NOT-IF-NOT:FLUSH-ALL}

\issue{TEST-NOT-IF-NOT:FLUSH-ALL}
\Thefunction{find-if-not} is deprecated.
\endissue{TEST-NOT-IF-NOT:FLUSH-ALL}

%% This equivalence isn't true when the element isn't found. -kmp 14-Jan-91
%
% \code
% (find item sequence ...)
%  \EQ (elt sequence (position item sequence ...))
% \endcode

\endcom

%%% ========== POSITION
%%% ========== POSITION-IF
%%% ========== POSITION-IF-NOT
\begincom{position, position-if, position-if-not}\ftype{Function}

\label Syntax::

\DefunWithValues position 
		 {item sequence {\key} from-end test test-not start end key}
		 {position}

\DefunWithValues position-if     {predicate sequence {\key} from-end start end key} {position}
\DefunWithValues position-if-not {predicate sequence {\key} from-end start end key} {position}

\label Arguments and Values::

\param{item}---an \term{object}.

\param{sequence}---a \term{proper sequence}.

\param{predicate}---a \term{designator} for a \term{function} of one argument 
  that returns a \term{generalized boolean}.

\param{from-end}---a \term{generalized boolean}.
  \Default{\term{false}}

\param{test}---a \term{designator} for a \term{function} of two \term{arguments}
  that returns a \term{generalized boolean}.

\param{test-not}---a \term{designator} for 
  a \term{function} of two \term{arguments}
  that returns a \term{generalized boolean}.

\issue{SUBSEQ-OUT-OF-BOUNDS}
\issue{RANGE-OF-START-AND-END-PARAMETERS:INTEGER-AND-INTEGER-NIL}
\param{start}, \param{end}---\term{bounding index designators} of \param{sequence}.
 \Defaults{\param{start} and \param{end}}{\f{0} and \nil}
\endissue{RANGE-OF-START-AND-END-PARAMETERS:INTEGER-AND-INTEGER-NIL}
\endissue{SUBSEQ-OUT-OF-BOUNDS}

\param{key}---a \term{designator} for a \term{function} of one argument,
  or \nil.

\param{position}---a \term{bounding index} of \param{sequence}, or \nil.

\label Description::

\funref{position}, \funref{position-if}, and \funref{position-if-not} 
each search \param{sequence} for an \term{element} that \term{satisfies the test}.

%% 14.3.0 29
%% 14.3.0 31
The \param{position} returned is the index within \param{sequence}
    of the leftmost  (if \param{from-end} is \term{true})
 or of the rightmost (if \param{from-end} is \term{false})
\term{element} that \term{satisfies the test};
otherwise \nil\ is returned.
%% 14.3.0 30
The index returned is relative to the left-hand end of the entire \param{sequence},
regardless of the value of \term{start}, \term{end}, or \term{from-end}.

\label Examples::

\code
 (position #\\a "baobab" :from-end t) \EV 4
 (position-if #'oddp '((1) (2) (3) (4)) :start 1 :key #'car) \EV 2
 (position 595 '()) \EV NIL
 (position-if-not #'integerp '(1 2 3 4 5.0)) \EV 4 
\endcode

\label Side Effects:\None.

\label Affected By:\None.

\label Exceptional Situations::

\issue{TEST-NOT-PROGRAM-ERROR:SIGNAL-ERROR-IN-SAFE-CODE}
\Lazychecktype{sequence}{a \term{proper sequence}}
\endissue{TEST-NOT-PROGRAM-ERROR:SIGNAL-ERROR-IN-SAFE-CODE}

\label See Also::

\funref{find},
\issue{MAPPING-DESTRUCTIVE-INTERACTION:EXPLICITLY-VAGUE}
{\secref\TraversalRules}
\endissue{MAPPING-DESTRUCTIVE-INTERACTION:EXPLICITLY-VAGUE}

\label Notes::

\issue{TEST-NOT-IF-NOT:FLUSH-ALL}
The \kwd{test-not} \term{argument} is deprecated.
\endissue{TEST-NOT-IF-NOT:FLUSH-ALL}

\issue{TEST-NOT-IF-NOT:FLUSH-ALL}
\Thefunction{position-if-not} is deprecated.
\endissue{TEST-NOT-IF-NOT:FLUSH-ALL}

\endcom                                        

%%% ========== SEARCH
\begincom{search}\ftype{Function}

\label Syntax::

\DefunWithValuesNewline search
                        {sequence-1 sequence-2 
                         {\key} \vtop{\hbox{from-end test test-not}
                                      \hbox{key start1 start2}
                                      \hbox{end1 end2}}}
                        {position}

\label Arguments and Values::
 
\param{Sequence-1}---a \term{sequence}.

\param{Sequence-2}---a \term{sequence}.

\param{from-end}---a \term{generalized boolean}.
  \Default{\term{false}}

\param{test}---a \term{designator} for a \term{function} of two \term{arguments}
  that returns a \term{generalized boolean}.

\param{test-not}---a \term{designator} for 
  a \term{function} of two \term{arguments}
  that returns a \term{generalized boolean}.

\param{key}---a \term{designator} for a \term{function} of one argument,
  or \nil.

\issue{SUBSEQ-OUT-OF-BOUNDS}
\issue{RANGE-OF-START-AND-END-PARAMETERS:INTEGER-AND-INTEGER-NIL}
\param{start1}, \param{end1}---\term{bounding index designators} of \param{sequence-1}.
 \Defaults{\param{start1} and \param{end1}}{\f{0} and \nil}

\param{start2}, \param{end2}---\term{bounding index designators} of \param{sequence-2}.
 \Defaults{\param{start2} and \param{end2}}{\f{0} and \nil}
\endissue{RANGE-OF-START-AND-END-PARAMETERS:INTEGER-AND-INTEGER-NIL}
\endissue{SUBSEQ-OUT-OF-BOUNDS}

\param{position}---a \term{bounding index} of \param{sequence-2},
		   or \nil.

\label Description::

%% 14.3.0 36
Searches \param{sequence-2} for a subsequence that matches \param{sequence-1}.  

%% 14.3.0 38
The implementation may choose to search \param{sequence-2} in any order;
there is no guarantee on the number of times the test is made.
For example,
when \param{start-end} is \term{true},
the \param{sequence} might actually be searched from left to right
instead of from right to left (but in either case would return
the rightmost matching subsequence). 
If the search succeeds,
\funref{search} returns the offset into \param{sequence-2} 
of the first element of the leftmost or rightmost matching subsequence, 
depending on \param{from-end};
otherwise \funref{search} returns \nil.

%% 14.3.0 37
If \param{from-end} is \term{true}, the index of the leftmost
element of the rightmost matching subsequence is returned.

\label Examples::
\code
 (search "dog" "it's a dog's life") \EV 7
 (search '(0 1) '(2 4 6 1 3 5) :key #'oddp) \EV 2
\endcode

\label Side Effects:\None.

\label Affected By:\None.

\label Exceptional Situations:

\issue{TEST-NOT-PROGRAM-ERROR:SIGNAL-ERROR-IN-SAFE-CODE}
\None.
\endissue{TEST-NOT-PROGRAM-ERROR:SIGNAL-ERROR-IN-SAFE-CODE}

\label See Also::

\issue{MAPPING-DESTRUCTIVE-INTERACTION:EXPLICITLY-VAGUE}
{\secref\TraversalRules}
\endissue{MAPPING-DESTRUCTIVE-INTERACTION:EXPLICITLY-VAGUE}

\label Notes::

\issue{TEST-NOT-IF-NOT:FLUSH-ALL}
The \kwd{test-not} \term{argument} is deprecated.
\endissue{TEST-NOT-IF-NOT:FLUSH-ALL}

\endcom                                    

%-------------------- Sequence Comparison --------------------

%%% ========== MISMATCH
\begincom{mismatch}\ftype{Function}

\label Syntax::

\DefunWithValuesNewline mismatch
			{sequence-1 sequence-2 
			 {\key} from-end test test-not key start1 start2 end1 end2}
			{position}

\label Arguments and Values::
 
\param{Sequence-1}---a \term{sequence}.

\param{Sequence-2}---a \term{sequence}.

\param{from-end}---a \term{generalized boolean}.
  \Default{\term{false}}

\param{test}---a \term{designator} for a \term{function} of two \term{arguments}
  that returns a \term{generalized boolean}.

\param{test-not}---a \term{designator} for 
  a \term{function} of two \term{arguments}
  that returns a \term{generalized boolean}.

\issue{SUBSEQ-OUT-OF-BOUNDS}
\issue{RANGE-OF-START-AND-END-PARAMETERS:INTEGER-AND-INTEGER-NIL}
\param{start1}, \param{end1}---\term{bounding index designators} of \param{sequence-1}.
 \Defaults{\param{start1} and \param{end1}}{\f{0} and \nil}

\param{start2}, \param{end2}---\term{bounding index designators} of \param{sequence-2}.
 \Defaults{\param{start2} and \param{end2}}{\f{0} and \nil}
\endissue{RANGE-OF-START-AND-END-PARAMETERS:INTEGER-AND-INTEGER-NIL}
\endissue{SUBSEQ-OUT-OF-BOUNDS}

\param{key}---a \term{designator} for a \term{function} of one argument,
  or \nil.

\param{position}---a \term{bounding index} of \param{sequence-1},
		   or \nil.

\label Description::

%% 14.3.0 34
The specified subsequences of         
\param{sequence-1} and \param{sequence-2} are compared element-wise.

The \param{key} argument is used for both the \param{sequence-1} and the \param{sequence-2}.

If \param{sequence-1} and \param{sequence-2} 
are of equal length and match in every element, the result is
\term{false}.  Otherwise, the result is a non-negative \term{integer},
the index within
\param{sequence-1} of the leftmost or rightmost position, depending
on \param{from-end}, at which the two
subsequences fail to match.
If one subsequence 
is shorter than and a matching prefix of the other,
the result is the index
relative to \param{sequence-1} beyond the last position tested.

%% 14.3.0 35
If \param{from-end} is \term{true}, then one plus the index of the rightmost
position in which the \param{sequences}
differ is returned.  In effect, the subsequences
are aligned at their right-hand ends; then, the last elements are compared,
the penultimate elements, and so on.  The index returned is 
an index relative to \param{sequence-1}.                  

\label Examples::                                 
\code
 (mismatch "abcd" "ABCDE" :test #'char-equal) \EV 4
 (mismatch '(3 2 1 1 2 3) '(1 2 3) :from-end t) \EV 3
 (mismatch '(1 2 3) '(2 3 4) :test-not #'eq :key #'oddp) \EV NIL
 (mismatch '(1 2 3 4 5 6) '(3 4 5 6 7) :start1 2 :end2 4) \EV NIL 
\endcode
                                               
\label Side Effects:\None.

\label Affected By:\None.

\label Exceptional Situations:

\issue{TEST-NOT-PROGRAM-ERROR:SIGNAL-ERROR-IN-SAFE-CODE}
\None.
\endissue{TEST-NOT-PROGRAM-ERROR:SIGNAL-ERROR-IN-SAFE-CODE}

\label See Also::

\issue{MAPPING-DESTRUCTIVE-INTERACTION:EXPLICITLY-VAGUE}
{\secref\TraversalRules}
\endissue{MAPPING-DESTRUCTIVE-INTERACTION:EXPLICITLY-VAGUE}

\label Notes::
                                                            
\issue{TEST-NOT-IF-NOT:FLUSH-ALL}
The \kwd{test-not} \term{argument} is deprecated.
\endissue{TEST-NOT-IF-NOT:FLUSH-ALL}

\endcom

%-------------------- Sequence Substitution --------------------

%%% ========== REPLACE
\begincom{replace}\ftype{Function}

\label Syntax::

\DefunWithValues replace 
		 {sequence-1 sequence-2 {\key} start1 end1 start2 end2}
		 {sequence-1}

\label Arguments and Values:: 

\param{sequence-1}---a \term{sequence}.

\param{sequence-2}---a \term{sequence}.

\issue{SUBSEQ-OUT-OF-BOUNDS}
\issue{RANGE-OF-START-AND-END-PARAMETERS:INTEGER-AND-INTEGER-NIL}
\param{start1}, \param{end1}---\term{bounding index designators} of \param{sequence-1}.
 \Defaults{\param{start1} and \param{end1}}{\f{0} and \nil}

\param{start2}, \param{end2}---\term{bounding index designators} of \param{sequence-2}.
 \Defaults{\param{start2} and \param{end2}}{\f{0} and \nil}
\endissue{RANGE-OF-START-AND-END-PARAMETERS:INTEGER-AND-INTEGER-NIL}
\endissue{SUBSEQ-OUT-OF-BOUNDS}

\label Description::

Destructively modifies \param{sequence-1} 
by replacing the \term{elements} of \param{subsequence-1}
 	      \term{bounded} by \param{start1} and \param{end1}
with the \term{elements} of \param{subsequence-2} 
      \term{bounded} by \param{start2} and \param{end2}. 

%% 14.3.0 4
\param{Sequence-1} is destructively modified by copying successive
\term{elements} into it from \param{sequence-2}.
\term{Elements} of the subsequence of \param{sequence-2} 
\term{bounded} by \param{start2} and \param{end2}
are copied into the subsequence of \param{sequence-1} 
\term{bounded} by \param{start1} and \param{end1}.
If these subsequences are not of the same length,
then the shorter length determines how many \term{elements} are copied;
the extra \term{elements} near the end of the longer subsequence 
are not involved in the operation.
The number of elements copied can be expressed as:

\code
 (min (- \i{end1} \i{start1}) (- \i{end2} \i{start2}))
\endcode

%% 14.3.0 5
If \param{sequence-1} and \param{sequence-2} are the \term{same} \term{object}
and the region being modified overlaps the region being copied
from, then it is as if the entire source region were copied to another
place and only then copied back into the target region.
However, if \param{sequence-1} and \param{sequence-2} are not the same,
but the region being modified overlaps the region being copied from
(perhaps because of shared list structure or displaced \term{arrays}),
then after the \funref{replace} operation
the subsequence of \param{sequence-1} being modified will have
unpredictable contents.
It is an error if the elements of \param{sequence-2} are not of a 
\term{type} that can be stored into \param{sequence-1}.
     
\label Examples::
\code
 (replace "abcdefghij" "0123456789" :start1 4 :end1 7 :start2 4) 
\EV "abcd456hij"
 (setq lst "012345678") \EV "012345678"
 (replace lst lst :start1 2 :start2 0) \EV "010123456"
 lst \EV "010123456"
\endcode

\label Side Effects::

The \param{sequence-1} is modified.

\label Affected By:\None.

\label Exceptional Situations:\None.

\label See Also::

\funref{fill}

%% Per X3J13. -kmp 05-Oct-93
\label Notes:\None.
    
\endcom

%%% ========== SUBSTITUTE
%%% ========== SUBSTITUTE-IF
%%% ========== SUBSTITUTE-IF-NOT
%%% ========== NSUBSTITUTE
%%% ========== NSUBSTITUTE-IF
%%% ========== NSUBSTITUTE-IF-NOT
\begincom{substitute, substitute-if, substitute-if-not,
	  nsubstitute, nsubstitute-if, nsubstitute-if-not}\ftype{Function}

\label Syntax::

\DefunWithValuesNewline substitute
		 	{newitem  olditem  sequence 
		   	  {\key} \vtop{\hbox{from-end test}
				       \hbox{test-not start}
                                       \hbox{end count key}}}
			{result-sequence}

\DefunWithValuesNewline substitute-if
			{newitem predicate sequence {\key} from-end start end count key}
			{result-sequence}

\DefunWithValuesNewline substitute-if-not
			{newitem predicate sequence {\key} from-end start end count key}
			{result-sequence}

\DefunWithValuesNewline nsubstitute
			{newitem  olditem  sequence
                          {\key} from-end test test-not start end count key}
			{sequence}

\DefunWithValuesNewline nsubstitute-if
			{newitem predicate sequence {\key} from-end start end count key}
			{sequence}

\DefunWithValuesNewline nsubstitute-if-not
			{newitem predicate sequence {\key} from-end start end count key}
			{sequence}

\label Arguments and Values::

\param{newitem}---an \term{object}.

\param{olditem}---an \term{object}.

\param{sequence}---a \term{proper sequence}.

\param{predicate}---a \term{designator} for a \term{function} of one \term{argument}
  that returns a \term{generalized boolean}.

\param{from-end}---a \term{generalized boolean}.
  \Default{\term{false}}

\param{test}---a \term{designator} for a \term{function} of two \term{arguments}
  that returns a \term{generalized boolean}.

\param{test-not}---a \term{designator} for 
  a \term{function} of two \term{arguments}
  that returns a \term{generalized boolean}.

\issue{SUBSEQ-OUT-OF-BOUNDS}
\issue{RANGE-OF-START-AND-END-PARAMETERS:INTEGER-AND-INTEGER-NIL}
\param{start}, \param{end}---\term{bounding index designators} of \param{sequence}.
 \Defaults{\param{start} and \param{end}}{\f{0} and \nil}
\endissue{RANGE-OF-START-AND-END-PARAMETERS:INTEGER-AND-INTEGER-NIL}
\endissue{SUBSEQ-OUT-OF-BOUNDS}

\issue{RANGE-OF-COUNT-KEYWORD:NIL-OR-INTEGER}
\param{count}---an \term{integer} or \nil.
\endissue{RANGE-OF-COUNT-KEYWORD:NIL-OR-INTEGER}
 \Default{\nil}

\param{key}---a \term{designator} for a \term{function} of one argument,
  or \nil.

\param{result-sequence}---a \term{sequence}.

\label Description::

\funref{substitute}, \funref{substitute-if}, and \funref{substitute-if-not}
return a 
%% Sandra is queasy about this.
% modified
copy of \param{sequence} in which each \term{element}
that \term{satisfies the test} has been replaced with \param{newitem}.

\funref{nsubstitute}, \funref{nsubstitute-if}, and \funref{nsubstitute-if-not}
are like \funref{substitute}, \funref{substitute-if}, and
\funref{substitute-if-not} respectively, but they may modify 
\param{sequence}.

%% Moon's suggested interpretation follows:

If 
\param{sequence} is a \term{vector}, the result is a
\term{vector} that has the same
\term{actual array element type} as \param{sequence}.
%% Moved to notes per Schulenburg #1 (by X3J13 vote at May 4-5, 1994 meeting).
%% -kmp 9-May-94
% The result might or might not be simple, 
% and might or might not be \term{identical} to \param{sequence}.
%% Another possible interpretation:
%% The result might or might not be \term{identical} to \param{sequence}.
%% If the result is  a \term{vector} that is not \term{identical}
%% to \param{sequence}, the result is a \term{fresh} \term{simple array}
%% of \term{rank} one.
If \param{sequence} is a \term{list}, the result is a 
\term{list}. 
%% End Moon's suggested interpretation.

%% 14.3.0 20
\param{Count}, if supplied, limits the number of elements
altered; if more than \param{count} \term{elements} \term{satisfy the test},
then of these \term{elements} only the leftmost or rightmost, depending
on \param{from-end}, are replaced,
as many as specified by \param{count}.
\issue{RANGE-OF-COUNT-KEYWORD:NIL-OR-INTEGER}
If \param{count} is supplied and negative, 
the behavior is as if zero had been supplied instead.
\endissue{RANGE-OF-COUNT-KEYWORD:NIL-OR-INTEGER}
If \param{count} is \nil, all matching items are affected.

%% 14.3.0 21
Supplying a \param{from-end} of \term{true} matters only when the
\param{count} is provided (and \term{non-nil});
in that case,
only the rightmost \param{count} \term{elements} \term{satisfying the test} are removed
(instead of the leftmost).

\param{predicate}, \param{test}, and \param{test-not} 
might be called more than once for each \term{sequence} \term{element},
and their side effects can happen in any order.         

%% 14.3.0 18
%% 14.3.0 23
The result of all these functions is a \term{sequence}
of the same \term{type} as \param{sequence}
that has the same elements except that those in the subsequence
\term{bounded} by \param{start} and \param{end} and \term{satisfying the test} 
have been replaced by \param{newitem}.  

\funref{substitute}, \funref{substitute-if}, and \funref{substitute-if-not}
return a \param{sequence} which can share with \param{sequence} 
or may be \term{identical} to the input \param{sequence}
if no elements need to be changed.

\issue{REMF-DESTRUCTION-UNSPECIFIED:X3J13-MAR-89}
\funref{nsubstitute} and  \funref{nsubstitute-if} are required to 
\macref{setf} any \funref{car} (if \param{sequence} is a \term{list}) 
or \funref{aref} (if \param{sequence} is a \term{vector})
of \param{sequence} that is required to be replaced with \param{newitem}.
If \param{sequence} is a \term{list},
none of the \term{cdrs} of the top-level \term{list} can be modified.  
\endissue{REMF-DESTRUCTION-UNSPECIFIED:X3J13-MAR-89}

\label Examples::                                                            

\code
 (substitute #\\. #\\SPACE "0 2 4 6") \EV "0.2.4.6"
 (substitute 9 4 '(1 2 4 1 3 4 5)) \EV (1 2 9 1 3 9 5)
 (substitute 9 4 '(1 2 4 1 3 4 5) :count 1) \EV (1 2 9 1 3 4 5)
 (substitute 9 4 '(1 2 4 1 3 4 5) :count 1 :from-end t)
\EV (1 2 4 1 3 9 5)
 (substitute 9 3 '(1 2 4 1 3 4 5) :test #'>) \EV (9 9 4 9 3 4 5)

 (substitute-if 0 #'evenp '((1) (2) (3) (4)) :start 2 :key #'car)
\EV ((1) (2) (3) 0)
 (substitute-if 9 #'oddp '(1 2 4 1 3 4 5)) \EV (9 2 4 9 9 4 9)
 (substitute-if 9 #'evenp '(1 2 4 1 3 4 5) :count 1 :from-end t)
\EV (1 2 4 1 3 9 5)

 (setq some-things (list 'a 'car 'b 'cdr 'c)) \EV (A CAR B CDR C)
 (nsubstitute-if "function was here" #'fboundp some-things
                 :count 1 :from-end t) \EV (A CAR B "function was here" C)
 some-things \EV (A CAR B "function was here" C)
 (setq alpha-tester (copy-seq "ab ")) \EV "ab "
 (nsubstitute-if-not #\\z #'alpha-char-p alpha-tester) \EV "abz"
 alpha-tester \EV "abz"
\endcode

\label Side Effects::

\funref{nsubstitute}, \funref{nsubstitute-if}, and \funref{nsubstitute-if-not}
modify \param{sequence}.

\label Affected By:\None.

\label Exceptional Situations::

\issue{TEST-NOT-PROGRAM-ERROR:SIGNAL-ERROR-IN-SAFE-CODE}
\Lazychecktype{sequence}{a \term{proper sequence}}
\endissue{TEST-NOT-PROGRAM-ERROR:SIGNAL-ERROR-IN-SAFE-CODE}
                        
\label See Also::

\funref{subst},
\funref{nsubst},
\issue{CONSTANT-MODIFICATION:DISALLOW}
{\secref\ConstantModification},
\endissue{CONSTANT-MODIFICATION:DISALLOW}
\issue{MAPPING-DESTRUCTIVE-INTERACTION:EXPLICITLY-VAGUE}
{\secref\TraversalRules}
\endissue{MAPPING-DESTRUCTIVE-INTERACTION:EXPLICITLY-VAGUE}

\label Notes::

%% Moved from Description per Schulenburg #1 (by X3J13 vote at May 4-5, 1994 meeting).
%% -kmp 9-May-94
If \param{sequence} is a \term{vector},
the result might or might not be simple, 
and might or might not be \term{identical} to \param{sequence}.

\issue{TEST-NOT-IF-NOT:FLUSH-ALL}
The \kwd{test-not} \term{argument} is deprecated.
\endissue{TEST-NOT-IF-NOT:FLUSH-ALL}

\issue{TEST-NOT-IF-NOT:FLUSH-ALL}
The functions \funref{substitute-if-not} and \funref{nsubstitute-if-not} are deprecated.
\endissue{TEST-NOT-IF-NOT:FLUSH-ALL}

\funref{nsubstitute} and \funref{nsubstitute-if} can be used 
in for-effect-only positions in code.

Because the side-effecting variants (\eg \funref{nsubstitute})
potentially change the path that is being traversed, their effects in
the presence of shared or circular structure may vary in surprising ways when
compared to their non-side-effecting alternatives.  To see this,
consider the following side-effect behavior, which might be exhibited by
some implementations:

\code
 (defun test-it (fn)
   (let ((x (cons 'b nil)))
     (rplacd x x)
     (funcall fn 'a 'b x :count 1)))
 (test-it #'substitute) \EV (A . #1=(B . #1#))
 (test-it #'nsubstitute) \EV (A . #1#)
\endcode

\endcom

%-------------------- Sequence Joining --------------------

%%% ========== CONCATENATE
\begincom{concatenate}\ftype{Function}

\label Syntax::

\DefunWithValues concatenate {result-type {\rest} sequences} {result-sequence}

\label Arguments and Values::

\issue{CONCATENATE-SEQUENCE:SIGNAL-ERROR}
\param{result-type}---a \typeref{sequence} \term{type specifier}.
\endissue{CONCATENATE-SEQUENCE:SIGNAL-ERROR}

\param{sequences}---a \term{sequence}.

\param{result-sequence}---a \term{proper sequence} of \term{type} \param{result-type}.

\label Description::

%% 14.2.0 3
\funref{concatenate} returns a \term{sequence} that contains 
all the individual elements of all the \param{sequences} in the order 
that they are supplied.
The \term{sequence} is of type \param{result-type}, 
which must be a \subtypeof{sequence}.

All of the \param{sequences} are copied from; the result
does not share any structure with any of the \param{sequences}.
%% 14.2.0 4
Therefore, if only one \param{sequence} is provided
and it is of type \param{result-type},
\funref{concatenate} is required to copy \param{sequence} rather than simply
returning it.  

It is an error if any element of the \param{sequences} cannot be an
element of the \term{sequence} result.
\reviewer{Barmar: Should signal?}%!!!

\issue{CONCATENATE-SEQUENCE:SIGNAL-ERROR}
If the \param{result-type} is a \term{subtype} of \typeref{list},
the result will be a \term{list}.

If the \param{result-type} is a \term{subtype} of \typeref{vector},
then if the implementation can determine the element type specified
for the \param{result-type}, the element type of the resulting array 
is the result of \term{upgrading} that element type; or, if the
implementation can determine that the element type is unspecified (or \f{*}),
the element type of the resulting array is \typeref{t};
otherwise, an error is signaled.
\endissue{CONCATENATE-SEQUENCE:SIGNAL-ERROR}

\label Examples::

\code
(concatenate 'string "all" " " "together" " " "now") \EV "all together now"
(concatenate 'list "ABC" '(d e f) #(1 2 3) #*1011)
\EV (#\\A #\\B #\\C D E F 1 2 3 1 0 1 1)
(concatenate 'list) \EV NIL
\endcode

\issue{SEQUENCE-TYPE-LENGTH:MUST-MATCH}
\code
  (concatenate '(vector * 2) "a" "bc") should signal an error
\endcode
\endissue{SEQUENCE-TYPE-LENGTH:MUST-MATCH}

\label Affected By:\None.                                                 

\label Exceptional Situations::

\issue{CONCATENATE-SEQUENCE:SIGNAL-ERROR}
An error is signaled if the \param{result-type} is neither
     a \term{recognizable subtype} of \typeref{list},
 nor a \term{recognizable subtype} of \typeref{vector}.
\endissue{CONCATENATE-SEQUENCE:SIGNAL-ERROR}

\issue{SEQUENCE-TYPE-LENGTH:MUST-MATCH}
An error \oftype{type-error} should be signaled if \param{result-type} 
specifies the number of elements and the sum of \param{sequences}
is different from that number.
\endissue{SEQUENCE-TYPE-LENGTH:MUST-MATCH}

\label See Also::

\funref{append}

\label Notes:\None.

\endcom              

%%% ========== MERGE
\begincom{merge}\ftype{Function}

\label Syntax::

\DefunWithValues merge 
		 {result-type sequence-1 sequence-2 predicate {\key} key}
		 {result-sequence}

\label Arguments and Values::

\issue{CONCATENATE-SEQUENCE:SIGNAL-ERROR}
\param{result-type}---a \typeref{sequence} \term{type specifier}.
\endissue{CONCATENATE-SEQUENCE:SIGNAL-ERROR}

\param{sequence-1}---a \term{sequence}.

\param{sequence-2}---a \term{sequence}.

\param{predicate}---a \term{designator} for
  a \term{function} of two arguments that returns a \term{generalized boolean}.

%% 14.4.0 12
\param{key}---a \term{designator} for a \term{function} of one argument,
  or \nil.

\param{result-sequence}---a \term{proper sequence} of \term{type} \param{result-type}.
     
\label Description::

%% 14.4.0 10
Destructively merges \param{sequence-1} with \param{sequence-2} according
to an order determined by the \param{predicate}.  \funref{merge} determines
the relationship between two elements by giving keys extracted from the
sequence elements to the \param{predicate}.

The first argument to the \param{predicate} function is an element of
\param{sequence-1} as returned by the \param{key} (if supplied); 
the second argument is an element of \param{sequence-2} as returned by 
the \param{key} (if supplied). 
\param{Predicate} should return \term{true} if and only if its first 
argument is strictly less than the second (in some appropriate sense). 
If the first argument is greater than or equal to the second
(in the appropriate sense), then \param{predicate} should return \term{false}.
\funref{merge}
considers two elements \f{x} and \f{y} to be equal if
\f{(funcall predicate x y)} and
\f{(funcall predicate y x)} both \term{yield} \term{false}.

%% 14.4.0 11
%% 14.4.0 12
The argument to the \param{key} is the \param{sequence} element.
Typically, the return value of the \param{key} 
becomes the argument to \param{predicate}.
If \param{key} is not supplied or \nil, the sequence element itself is used.
The \param{key} may be executed more than once for each \term{sequence} \term{element},
and its side effects may occur in any order.

%% 14.4.0 13
If \param{key} and \param{predicate} return, then the merging operation 
will terminate.  The result of merging two \term{sequences} \f{x} and \f{y}
is a new \term{sequence} of type \param{result-type} \f{z}, 
such that the length of \f{z} is the sum of the lengths of \f{x}
and \f{y}, and \f{z} contains all the elements of \f{x} and \f{y}.
If \f{x1} and \f{x2} are two elements of \f{x}, and \f{x1} precedes
\f{x2} in \f{x}, then \f{x1} precedes \f{x2} in \f{z}, and similarly for
elements of \f{y}.  In short, \f{z} is an interleaving of \f{x} and \f{y}.

%% 14.4.0 14
If \f{x} and \f{y} were correctly sorted according to the
\param{predicate}, then \f{z} will also be correctly sorted.
If \f{x} or \f{y} is not so sorted, then \f{z} will not be sorted,
but will nevertheless be an interleaving of \f{x} and \f{y}.

%% 14.4.0 14
The merging operation is guaranteed stable;
if two or more elements are considered equal by the \param{predicate}, 
then the elements from \param{sequence-1} will
precede those from \param{sequence-2} in the result.

\param{sequence-1} and/or \param{sequence-2} may be destroyed.

\issue{CONCATENATE-SEQUENCE:SIGNAL-ERROR}
If the \param{result-type} is a \term{subtype} of \typeref{list},
the result will be a \term{list}.

If the \param{result-type} is a \term{subtype} of \typeref{vector},
then if the implementation can determine the element type specified
for the \param{result-type}, the element type of the resulting array 
is the result of \term{upgrading} that element type; or, if the
implementation can determine that the element type is unspecified (or \f{*}),
the element type of the resulting array is \typeref{t};
otherwise, an error is signaled.
\endissue{CONCATENATE-SEQUENCE:SIGNAL-ERROR}

\label Examples::                               
\code
 (setq test1 (list 1 3 4 6 7))
 (setq test2 (list 2 5 8))
 (merge 'list test1 test2 #'<) \EV (1 2 3 4 5 6 7 8)
 (setq test1 (copy-seq "BOY"))
 (setq test2 (copy-seq :nosy"))
 (merge 'string test1 test2 #'char-lessp) \EV "BnOosYy"
 (setq test1 (vector ((red . 1) (blue . 4))))
 (setq test2 (vector ((yellow . 2) (green . 7))))
 (merge 'vector test1 test2 #'< :key #'cdr) 
\EV #((RED . 1) (YELLOW . 2) (BLUE . 4) (GREEN . 7)) 
\endcode
\issue{SEQUENCE-TYPE-LENGTH:MUST-MATCH}
\code
 (merge '(vector * 4) '(1 5) '(2 4 6) #'<) should signal an error
\endcode
\endissue{SEQUENCE-TYPE-LENGTH:MUST-MATCH}

\label Affected By:\None.

\label Exceptional Situations::

\issue{CONCATENATE-SEQUENCE:SIGNAL-ERROR}
An error must be signaled if the \param{result-type} is neither
     a \term{recognizable subtype} of \typeref{list},
 nor a \term{recognizable subtype} of \typeref{vector}.
\endissue{CONCATENATE-SEQUENCE:SIGNAL-ERROR}

\issue{SEQUENCE-TYPE-LENGTH:MUST-MATCH}
An error \oftype{type-error} should be signaled
if \param{result-type} specifies the number of elements 
and the sum of the lengths of \param{sequence-1} and \param{sequence-2} 
is different from that number.
\endissue{SEQUENCE-TYPE-LENGTH:MUST-MATCH}

\label See Also::

\funref{sort},
\funref{stable-sort},
\issue{CONSTANT-MODIFICATION:DISALLOW}
{\secref\ConstantModification},
\endissue{CONSTANT-MODIFICATION:DISALLOW}
\issue{MAPPING-DESTRUCTIVE-INTERACTION:EXPLICITLY-VAGUE}
{\secref\TraversalRules}
\endissue{MAPPING-DESTRUCTIVE-INTERACTION:EXPLICITLY-VAGUE}

\label Notes:\None.

\endcom

%-------------------- Sequence Deletion --------------------

%%% ========== REMOVE
%%% ========== REMOVE-IF
%%% ========== REMOVE-IF-NOT
%%% ========== DELETE
%%% ========== DELETE-IF
%%% ========== DELETE-IF-NOT
\begincom{remove, remove-if, remove-if-not, 
          delete, delete-if, delete-if-not}\ftype{Function}

\label Syntax::

\DefunWithValues remove
		 {item sequence {\key} from-end test test-not start end count key}
		 {result-sequence}

\DefunWithValues remove-if 
		 {test sequence {\key} from-end start end count key}
		 {result-sequence}

\DefunWithValues remove-if-not 
		 {test sequence {\key} from-end start end count key}
		 {result-sequence}

\DefunWithValues delete
	         {item sequence {\key} from-end test test-not start end count key}
		 {result-sequence}

\DefunWithValues delete-if
		 {test sequence {\key} from-end start end count key}
	         {result-sequence}

\DefunWithValues delete-if-not 
		 {test sequence {\key} from-end start end count key}
		 {result-sequence}

\label Arguments and Values::

\param{item}---an \term{object}.

\param{sequence}---a \term{proper sequence}.

\param{test}---a \term{designator} for a \term{function} 
  of one \term{argument} that returns a \term{generalized boolean}.

\param{from-end}---a \term{generalized boolean}.
  \Default{\term{false}}

\param{test}---a \term{designator} for a \term{function} of two \term{arguments}
  that returns a \term{generalized boolean}.

\param{test-not}---a \term{designator} for 
  a \term{function} of two \term{arguments}
  that returns a \term{generalized boolean}.

\issue{SUBSEQ-OUT-OF-BOUNDS}
\issue{RANGE-OF-START-AND-END-PARAMETERS:INTEGER-AND-INTEGER-NIL}
\param{start}, \param{end}---\term{bounding index designators} of \param{sequence}.
 \Defaults{\param{start} and \param{end}}{\f{0} and \nil}
\endissue{RANGE-OF-START-AND-END-PARAMETERS:INTEGER-AND-INTEGER-NIL}
\endissue{SUBSEQ-OUT-OF-BOUNDS}

\issue{RANGE-OF-COUNT-KEYWORD:NIL-OR-INTEGER}
\param{count}---an \term{integer} or \nil.
\endissue{RANGE-OF-COUNT-KEYWORD:NIL-OR-INTEGER}
  \Default{\nil}

\param{key}---a \term{designator} for a \term{function} of one argument,
  or \nil.

\param{result-sequence}---a \term{sequence}.

\label Description::

\funref{remove}, \funref{remove-if}, and \funref{remove-if-not}
return a \param{sequence} from which 
the elements that \term{satisfy the test} 
have been removed.  

\funref{delete}, \funref{delete-if}, and \funref{delete-if-not}
are like \funref{remove}, \funref{remove-if}, and
\funref{remove-if-not} respectively, 
but they may modify \param{sequence}.

%% Moon's suggested interpretation follows:

If \param{sequence} is a \term{vector}, the result is a
\term{vector} that has the same
\term{actual array element type} as \param{sequence}.
%% Moved to notes per Schulenburg #1 (by X3J13 vote at May 4-5, 1994 meeting).
%% -kmp 9-May-94
% The result might or might not be simple, 
% and might or might not be \term{identical} to \param{sequence}.
%% Another possible interpretation:
%% The result might or might not be \term{identical} to \param{sequence}.
%% If the result is  a \term{vector} that is not \term{identical}
%% to \param{sequence}, the result is a \term{fresh} \term{simple array}
%% of \term{rank} one.
If \param{sequence} is a \term{list}, the result is a \term{list}. 
%% End Moon's suggested interpretation.

%% 14.3.0 8
%% 14.3.0 11
Supplying a \param{from-end} of \term{true} matters only when the
\param{count} is provided; in that case only the rightmost \param{count} elements 
\term{satisfying the test} are deleted.

%% 14.3.0 7
%% 14.3.0 10
\param{Count}, if supplied, limits the number of elements
removed or deleted; if more than \param{count} elements \term{satisfy the test},
then of these elements only the leftmost or rightmost, depending on
\param{from-end},
are deleted or removed,
as many as specified by \param{count}.
\issue{RANGE-OF-COUNT-KEYWORD:NIL-OR-INTEGER}
If \param{count} is supplied and negative, 
the behavior is as if zero had been supplied instead.
\endissue{RANGE-OF-COUNT-KEYWORD:NIL-OR-INTEGER}
If \param{count} is \nil, all matching items are affected.

For all these functions,
elements 
not removed or deleted occur in the same order in the result
as they did in \param{sequence}.

%% 14.3.0 6
\funref{remove}, \funref{remove-if}, \funref{remove-if-not} return
a \term{sequence}
of the same \term{type} as \param{sequence}
that has the same elements except that those in the subsequence
\term{bounded} by \param{start} and \param{end} and \term{satisfying the test}
have been removed.  
This is a non-destructive operation. If any
elements need to be removed, the result will be a copy.
The result of \funref{remove} may share
with \param{sequence}; 
the result may be \term{identical} to the input \param{sequence}
if no elements need to be removed.

%% 14.3.0 9
\funref{delete}, \funref{delete-if}, and \funref{delete-if-not}
return a \term{sequence} 
of the same \term{type} as \param{sequence}
that has the same elements except that those in the subsequence
\term{bounded} by \param{start} and \param{end} and \term{satisfying the test}
have been deleted. 
\param{Sequence} may be destroyed and used to construct
the result; however, the result might or might not be \term{identical} 
to \param{sequence}.

\issue{REMF-DESTRUCTION-UNSPECIFIED:X3J13-MAR-89}
\funref{delete}, when \param{sequence} is a \term{list}, is permitted to 
\macref{setf} any part, \funref{car} or \funref{cdr}, of the
top-level list structure in that \param{sequence}.
When \param{sequence} is a \term{vector},  \funref{delete} is 
permitted to change the dimensions of the \term{vector} 
and to slide its elements into new positions without
permuting them to produce the resulting \term{vector}.

\funref{delete-if} is constrained to behave exactly as follows:

\code
 (delete nil \i{sequence}
             :test #'(lambda (ignore \i{item}) (funcall \i{test} \i{item}))
             ...)
\endcode
 
\endissue{REMF-DESTRUCTION-UNSPECIFIED:X3J13-MAR-89}

\label Examples::
\code
 (remove 4 '(1 3 4 5 9)) \EV (1 3 5 9)
 (remove 4 '(1 2 4 1 3 4 5)) \EV (1 2 1 3 5)
 (remove 4 '(1 2 4 1 3 4 5) :count 1) \EV (1 2 1 3 4 5)
 (remove 4 '(1 2 4 1 3 4 5) :count 1 :from-end t) \EV (1 2 4 1 3 5)
 (remove 3 '(1 2 4 1 3 4 5) :test #'>) \EV (4 3 4 5)
 (setq lst '(list of four elements)) \EV (LIST OF FOUR ELEMENTS)
 (setq lst2 (copy-seq lst)) \EV (LIST OF FOUR ELEMENTS)
 (setq lst3 (delete 'four lst)) \EV (LIST OF ELEMENTS)
 (equal lst lst2) \EV \term{false}
 (remove-if #'oddp '(1 2 4 1 3 4 5)) \EV (2 4 4)
 (remove-if #'evenp '(1 2 4 1 3 4 5) :count 1 :from-end t) 
\EV (1 2 4 1 3 5)
 (remove-if-not #'evenp '(1 2 3 4 5 6 7 8 9) :count 2 :from-end t)
\EV (1 2 3 4 5 6 8)
 (setq tester (list 1 2 4 1 3 4 5)) \EV (1 2 4 1 3 4 5)
 (delete 4 tester) \EV (1 2 1 3 5)
 (setq tester (list 1 2 4 1 3 4 5)) \EV (1 2 4 1 3 4 5)
 (delete 4 tester :count 1) \EV (1 2 1 3 4 5)
 (setq tester (list 1 2 4 1 3 4 5)) \EV (1 2 4 1 3 4 5)
 (delete 4 tester :count 1 :from-end t) \EV (1 2 4 1 3 5)
 (setq tester (list 1 2 4 1 3 4 5)) \EV (1 2 4 1 3 4 5)
 (delete 3 tester :test #'>) \EV (4 3 4 5)
 (setq tester (list 1 2 4 1 3 4 5)) \EV (1 2 4 1 3 4 5)
 (delete-if #'oddp tester) \EV (2 4 4)
 (setq tester (list 1 2 4 1 3 4 5)) \EV (1 2 4 1 3 4 5)
 (delete-if #'evenp tester :count 1 :from-end t) \EV (1 2 4 1 3 5)    
 (setq tester (list 1 2 3 4 5 6)) \EV (1 2 3 4 5 6) 
 (delete-if #'evenp tester) \EV (1 3 5) 
 tester \EV \term{implementation-dependent}
\endcode

%!!! This example (with the "or") looks like bad notation AND questionable value
\issue{REMF-DESTRUCTION-UNSPECIFIED:X3J13-MAR-89}
\code
 (setq foo (list 'a 'b 'c)) \EV (A B C)
 (setq bar (cdr foo)) \EV (B C)
 (setq foo (delete 'b foo)) \EV (A C)
 bar \EV ((C)) or ...
 (eq (cdr foo) (car bar)) \EV T or ...
\endcode
\endissue{REMF-DESTRUCTION-UNSPECIFIED:X3J13-MAR-89}

\label Side Effects::

For \funref{delete}, \funref{delete-if}, and \funref{delete-if-not},
\param{sequence} may be destroyed and used to construct the result.
                                                       
\label Affected By:\None.

\label Exceptional Situations::

\issue{TEST-NOT-PROGRAM-ERROR:SIGNAL-ERROR-IN-SAFE-CODE}
\Lazychecktype{sequence}{a \term{proper sequence}}
\endissue{TEST-NOT-PROGRAM-ERROR:SIGNAL-ERROR-IN-SAFE-CODE}

\label See Also::

\issue{CONSTANT-MODIFICATION:DISALLOW}
{\secref\ConstantModification},
\endissue{CONSTANT-MODIFICATION:DISALLOW}
\issue{MAPPING-DESTRUCTIVE-INTERACTION:EXPLICITLY-VAGUE}
{\secref\TraversalRules}
\endissue{MAPPING-DESTRUCTIVE-INTERACTION:EXPLICITLY-VAGUE}

\label Notes::

%% Moved from Description per Schulenburg #1 (by X3J13 vote at May 4-5, 1994 meeting).
%% -kmp 9-May-94
If \param{sequence} is a \term{vector},
the result might or might not be simple, 
and might or might not be \term{identical} to \param{sequence}.

\issue{TEST-NOT-IF-NOT:FLUSH-ALL}
The \kwd{test-not} \term{argument} is deprecated.
\endissue{TEST-NOT-IF-NOT:FLUSH-ALL}

\issue{TEST-NOT-IF-NOT:FLUSH-ALL}
The functions \funref{delete-if-not} and \funref{remove-if-not} are deprecated.
\endissue{TEST-NOT-IF-NOT:FLUSH-ALL}

\endcom

%%% ========== REMOVE-DUPLICATES
%%% ========== DELETE-DUPLICATES
\begincom{remove-duplicates, delete-duplicates}\ftype{Function}

\label Syntax::

\DefunWithValuesNewline remove-duplicates 
			{sequence {\key}
 			  \vtop{\hbox{from-end test test-not}
				\hbox{start end key}}}
			{result-sequence}

\DefunWithValuesNewline delete-duplicates
			{sequence {\key} 
			   \vtop{\hbox{from-end test test-not}
                                 \hbox{start end key}}}
			{result-sequence}

\label Arguments and Values::

\param{sequence}---a \term{proper sequence}.

\param{from-end}---a \term{generalized boolean}.
  \Default{\term{false}}

\param{test}---a \term{designator} for a \term{function} of two \term{arguments}
  that returns a \term{generalized boolean}.

\param{test-not}---a \term{designator} for 
  a \term{function} of two \term{arguments}
  that returns a \term{generalized boolean}.

\issue{SUBSEQ-OUT-OF-BOUNDS}
\issue{RANGE-OF-START-AND-END-PARAMETERS:INTEGER-AND-INTEGER-NIL}
\param{start}, \param{end}---\term{bounding index designators} of \param{sequence}.
 \Defaults{\param{start} and \param{end}}{\f{0} and \nil}
\endissue{RANGE-OF-START-AND-END-PARAMETERS:INTEGER-AND-INTEGER-NIL}
\endissue{SUBSEQ-OUT-OF-BOUNDS}

\param{key}---a \term{designator} for a \term{function} of one argument,
  or \nil.

\param{result-sequence}---a \term{sequence}.

\label Description::

\funref{remove-duplicates} returns a modified copy of \param{sequence} 
from which any element that matches another element occurring in
\param{sequence} has been removed. 

%% Moon's suggested interpretation follows:

If \param{sequence} is a \term{vector}, the result is a
\term{vector} that has the same
\term{actual array element type} as \param{sequence}.
%% Moved to notes per Schulenburg #1 (by X3J13 vote at May 4-5, 1994 meeting).
%% -kmp 9-May-94
% The result might or might not be simple,
% and might or might not be \term{identical} to \param{sequence}.
%% Another possible interpretation:
%% The result might or might not be \term{identical} to \param{sequence}.
%% If the result is  a \term{vector} that is not \term{identical}
%% to \param{sequence}, the result is a \term{fresh} \term{simple array}
%% of \term{rank} one.
If \param{sequence} is a \term{list}, the result is a \term{list}. 
%% End Moon's suggested interpretation.

\funref{delete-duplicates} is like \funref{remove-duplicates},
but \funref{delete-duplicates} may modify \param{sequence}.

%% 14.3.0 13
The elements of \param{sequence} are compared \term{pairwise}, and if any two match,
then the one occurring earlier in \param{sequence}
is discarded, unless \param{from-end} is \term{true}, in which case the one
later in \param{sequence} is discarded.

\funref{remove-duplicates} and \funref{delete-duplicates} 
return a \term{sequence} of the same \term{type} as 
\param{sequence} with enough elements removed so that no two of the remaining
elements match.  The order of the elements remaining in the result
is the same as the order in which they appear in \param{sequence}.

\funref{remove-duplicates} returns a \term{sequence}
%% 14.3.0 14
that may share
with \param{sequence} or may be \term{identical} to \param{sequence}
if no elements need to be removed.

\issue{REMF-DESTRUCTION-UNSPECIFIED:X3J13-MAR-89}
\funref{delete-duplicates}, when \param{sequence} is a \term{list},
is permitted to \macref{setf} any part, \funref{car} or \funref{cdr},
of the top-level list structure in that \param{sequence}.
When \param{sequence} is a \term{vector}, \funref{delete-duplicates} 
is permitted to change the dimensions of the \term{vector} 
and to slide its elements into new positions without
permuting them to produce the resulting \term{vector}.
\endissue{REMF-DESTRUCTION-UNSPECIFIED:X3J13-MAR-89}

\label Examples::

%% 14.3.0 16
\code
 (remove-duplicates "aBcDAbCd" :test #'char-equal :from-end t) \EV "aBcD"
 (remove-duplicates '(a b c b d d e)) \EV (A C B D E)
 (remove-duplicates '(a b c b d d e) :from-end t) \EV (A B C D E)
 (remove-duplicates '((foo #\\a) (bar #\\%) (baz #\\A))
     :test #'char-equal :key #'cadr) \EV ((BAR #\\%) (BAZ #\\A))
 (remove-duplicates '((foo #\\a) (bar #\\%) (baz #\\A)) 
     :test #'char-equal :key #'cadr :from-end t) \EV ((FOO #\\a) (BAR #\\%))
 (setq tester (list 0 1 2 3 4 5 6))
 (delete-duplicates tester :key #'oddp :start 1 :end 6) \EV (0 4 5 6)
\endcode

\label Side Effects::

%% 14.3.0 15
\funref{delete-duplicates} might destructively modify \param{sequence}.

\label Affected By:\None.                                                      

\label Exceptional Situations::

\issue{TEST-NOT-PROGRAM-ERROR:SIGNAL-ERROR-IN-SAFE-CODE}
\Shouldchecktype{sequence}{a \term{proper sequence}}
\endissue{TEST-NOT-PROGRAM-ERROR:SIGNAL-ERROR-IN-SAFE-CODE}

\label See Also::

\issue{CONSTANT-MODIFICATION:DISALLOW}
{\secref\ConstantModification},
\endissue{CONSTANT-MODIFICATION:DISALLOW}
\issue{MAPPING-DESTRUCTIVE-INTERACTION:EXPLICITLY-VAGUE}
{\secref\TraversalRules}
\endissue{MAPPING-DESTRUCTIVE-INTERACTION:EXPLICITLY-VAGUE}

\label Notes::

%% Moved from Description per Schulenburg #1 (by X3J13 vote at May 4-5, 1994 meeting).
%% -kmp 9-May-94
If \param{sequence} is a \term{vector},
the result might or might not be simple, 
and might or might not be \term{identical} to \param{sequence}.

\issue{TEST-NOT-IF-NOT:FLUSH-ALL}
The \kwd{test-not} \term{argument} is deprecated.
\endissue{TEST-NOT-IF-NOT:FLUSH-ALL}

%% 14.3.0 17
These functions are useful for converting \param{sequence} into a canonical
form suitable for representing a set.

\endcom


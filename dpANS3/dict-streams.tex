% -*- Mode: TeX -*-

\def\ExplainRecursiveP{If \param{recursive-p} is \term{true},
this call is expected to be embedded in a higher-level call to \funref{read} 
or a similar \term{function} used by the \term{Lisp reader}.}

% Streams
%  Stream I/O
%   Binary I/O
%    Byte I/O
%   Text I/O
%    Character I/O
%    String I/O
%   General I/O
%    File Position
%    File Opening
%    Stream Closing
%    Stream Buffering
%    Query Functions
%  File Streams
%  Synonym Streams
%  Broadcast Streams
%  Two-Way-Streams
%  Echo Streams
%  Concatenated Streams
%  String Streams
%  Stream Variables

%-------------------- Stream Types --------------------

%% 2.10.0 1
\begincom{stream}\ftype{System Class}

\label Class Precedence List::
\typeref{stream},
\typeref{t}

\label Description::
 
A \term{stream} is an \term{object} that can be used with an input or output
function to identify an appropriate source or sink of \term{characters} or 
\term{bytes} for that operation.

For more complete information, \seesection\StreamConcepts.

\issue{TYPE-OF-AND-PREDEFINED-CLASSES:UNIFY-AND-EXTEND}
% \issue{STREAM-ACCESS:ADD-TYPES-ACCESSORS}
% The types
% \typeref{broadcast-stream},
% \typeref{concatenated-stream},
% \typeref{echo-stream},
% \typeref{file-stream},
% \typeref{string-stream},
% \typeref{synonym-stream}, and
% \typeref{two-way-stream} are \term{pairwise} \term{disjoint} \subtypesof{stream}.
% \endissue{STREAM-ACCESS:ADD-TYPES-ACCESSORS}
\endissue{TYPE-OF-AND-PREDEFINED-CLASSES:UNIFY-AND-EXTEND}

\label See Also::

{\secref\StreamConcepts},
{\secref\PrintingOtherObjects},
{\chapref\Printer},
{\chapref\Reader}

\endcom%{stream}\ftype{System Class}
\begincom{broadcast-stream}\ftype{System Class}

\issue{TYPE-OF-AND-PREDEFINED-CLASSES:UNIFY-AND-EXTEND}
\issue{STREAM-ACCESS:ADD-TYPES-ACCESSORS}

\label Class Precedence List::

\typeref{broadcast-stream},
\typeref{stream},
\typeref{t}

\label Description::

A \term{broadcast stream} is an \term{output} \term{stream} which 
has associated with it a set of zero or more \term{output} \term{streams} 
such that any output sent to the \term{broadcast stream} gets passed on
as output to each of the associated \term{output} \term{streams}.
(If a \term{broadcast stream} has no \term{component streams},
then all output to the \term{broadcast stream} is discarded.)

The set of operations that may be performed on a \term{broadcast stream} 
is the intersection of those for its associated \term{output} \term{streams}.

% Except as explicitly stated otherwise, 
% the values returned by a stream operation are 
% the values resulting from performing the operation 
% on the last of its \term{component streams};
% the values resulting from performing the operation
% on all preceding \term{streams} are discarded.                    
% %Barmar: What if no streams are supplied?
% %Sandra: Actually, aren't the results of functions that operate on streams
% % specified explicitly in the descriptions of those functions?
% %KMP: I'm not sure this answers the question for objects of type BROADCAST-STREAM.
% %Sandra: CLOSE is a stream operation and this statement is certainly not
% %   true for CLOSE -- the component streams are not closed.
%% Trying to untangle all the mail on this. -kmp 11-Feb-92
Some output operations (\eg \funref{fresh-line}) return \term{values} based on the
state of the \term{stream} at the time of the operation.
\issue{BROADCAST-STREAM-RETURN-VALUES:CLARIFY-MINIMALLY}
Since these \term{values} might differ for each of the \term{component streams},
it is necessary to describe their return value specifically:

\beginlist

\itemitem{\bull}
    \funref{stream-element-type} returns
    the value from the last component stream, 
    or \typeref{t} if there are no component streams.

\itemitem{\bull}
    \funref{fresh-line} returns
     the value from the last component stream, 
     or \nil\ if there are no component streams.

\itemitem{\bull}
   The functions
          \funref{file-length},
          \funref{file-position},
          \funref{file-string-length},
      and \funref{stream-external-format}
   return the value from the last component stream;
   if there are no component streams,
       \funref{file-length} and \funref{file-position} return \f{0},
       \funref{file-string-length} returns \f{1},
   and \funref{stream-external-format} returns \kwd{default}.

\itemitem{\bull}
  The functions \funref{streamp} and \funref{output-stream-p} 
  always return \term{true} for \term{broadcast streams}.

\itemitem{\bull}
  The functions \funref{open-stream-p} tests whether the \term{broadcast stream}
  is \term{open}\meaning{2}, not whether its component streams are \term{open}.

\itemitem{\bull}
  The functions \funref{input-stream-p} and \term{interactive-stream-p} 
  return an \term{implementation-defined}, \term{generalized boolean} value.

\itemitem{\bull}
  For the input operations
        \funref{clear-input}
        \funref{listen},
        \funref{peek-char},
        \funref{read-byte},
        \funref{read-char-no-hang},
        \funref{read-char},
        \funref{read-line},
    and \funref{unread-char},
  the consequences are undefined if the indicated operation is performed.
  However, an \term{implementation} is permitted 
  to define such a behavior as an \term{implementation-dependent} extension.
\endlist

For any output operations not having their return values explicitly specified above
or elsewhere in this document, it is defined that
the \term{values} returned by such an operation are 
the \term{values} resulting from performing the operation 
on the last of its \term{component streams};
the \term{values} resulting from performing the operation
on all preceding \term{streams} are discarded.                    
If there are no \term{component streams}, 
the value is \term{implementation-dependent}.
\endissue{BROADCAST-STREAM-RETURN-VALUES:CLARIFY-MINIMALLY}

\label See Also::

\funref{broadcast-stream-streams},
\funref{make-broadcast-stream}

\endissue{STREAM-ACCESS:ADD-TYPES-ACCESSORS}
\endissue{TYPE-OF-AND-PREDEFINED-CLASSES:UNIFY-AND-EXTEND}

\endcom%{broadcast-stream}\ftype{System Class}
\begincom{concatenated-stream}\ftype{System Class}

\issue{TYPE-OF-AND-PREDEFINED-CLASSES:UNIFY-AND-EXTEND}
\issue{STREAM-ACCESS:ADD-TYPES-ACCESSORS}

\label Class Precedence List::

\typeref{concatenated-stream},
\typeref{stream},
\typeref{t}

\label Description::

A \term{concatenated stream} is an \term{input} \term{stream} which 
is a \term{composite stream} of zero or more other \term{input} \term{streams}, 
such that the sequence of data which can be read from the
\term{concatenated stream} is the same as the concatenation of the 
sequences of data which could be read from each of the 
constituent \term{streams}.

Input from a \term{concatenated stream} is taken from the first
of the associated \term{input streams} until it reaches \term{end of file}\meaning{1}; 
then that \term{stream} is discarded, and subsequent input is taken
from the next \term{input stream}, and so on.
%I added the following to round things out. -kmp 22-Apr-91
An \term{end of file} on the associated \term{input streams} is always managed
invisibly by the \term{concatenated stream}---the only time a client of
a \term{concatenated stream} sees an \term{end of file} is when an attempt is
made to obtain data from the \term{concatenated stream} but it has no
remaining \term{input streams} from which to obtain such data.

%!!!
%Moon:
%  I'm not sure about a concatenated stream with no streams.  Presumably the
%  valid set of operations for a concatenated stream is the intersection of
%  the valid set of operations for the component streams (this is never
%  specified, though).  Maybe an empty concatenated stream accepts all
%  operations, both character and binary, and always gives eof?  Or maybe
%  it shouldn't be allowed?

\endissue{STREAM-ACCESS:ADD-TYPES-ACCESSORS}
\endissue{TYPE-OF-AND-PREDEFINED-CLASSES:UNIFY-AND-EXTEND}

\label See Also::

\funref{concatenated-stream-streams},
\funref{make-concatenated-stream}

\endcom%{concatenated-stream}\ftype{System Class}
\begincom{echo-stream}\ftype{System Class}

\issue{TYPE-OF-AND-PREDEFINED-CLASSES:UNIFY-AND-EXTEND}
\issue{STREAM-ACCESS:ADD-TYPES-ACCESSORS}

\label Class Precedence List::

\typeref{echo-stream},
\typeref{stream},
\typeref{t}

\label Description::

An \term{echo stream} is a \term{bidirectional} \term{stream}
that gets  its input  from an associated \term{input}  \term{stream}
and  sends its output to   an associated \term{output} \term{stream}.

All input taken from the \term{input} \term{stream} 
is echoed to the \term{output} \term{stream}.
Whether the input is echoed immediately after it is encountered,
or after it has been read from the \term{input stream}
is \term{implementation-dependent}.

\endissue{STREAM-ACCESS:ADD-TYPES-ACCESSORS}
\endissue{TYPE-OF-AND-PREDEFINED-CLASSES:UNIFY-AND-EXTEND}

\label See Also::

\funref{echo-stream-input-stream},
\funref{echo-stream-output-stream},
\funref{make-echo-stream}

\endcom%{echo-stream}\ftype{System Class}
\begincom{file-stream}\ftype{System Class}

\issue{TYPE-OF-AND-PREDEFINED-CLASSES:UNIFY-AND-EXTEND}
\issue{STREAM-ACCESS:ADD-TYPES-ACCESSORS}

\label Class Precedence List::

\typeref{file-stream},
\typeref{stream},
\typeref{t}

\label Description::

An \term{object} \oftype{file-stream} is a \term{stream} the direct
source or sink of which is a \term{file}.  Such a \term{stream} is
created explicitly by \funref{open} and \macref{with-open-file}, and
implicitly by \term{functions} such as \funref{load} that process \term{files}.

\endissue{STREAM-ACCESS:ADD-TYPES-ACCESSORS}
\endissue{TYPE-OF-AND-PREDEFINED-CLASSES:UNIFY-AND-EXTEND}

\label See Also::

\funref{load},
\funref{open},
\macref{with-open-file}

\endcom%{file-stream}\ftype{System Class}
\begincom{string-stream}\ftype{System Class}

\issue{TYPE-OF-AND-PREDEFINED-CLASSES:UNIFY-AND-EXTEND}
\issue{STREAM-ACCESS:ADD-TYPES-ACCESSORS}

\label Class Precedence List::

\typeref{string-stream},
\typeref{stream},
\typeref{t}

\label Description::

A \term{string stream} is a \term{stream} 
which reads input from or writes output to an associated \term{string}.

The \term{stream element type} of a \term{string stream} is always
a \subtypeof{character}.

\label See Also::

\funref{make-string-input-stream},
\funref{make-string-output-stream},
\macref{with-input-from-string},
\macref{with-output-to-string}

\endissue{STREAM-ACCESS:ADD-TYPES-ACCESSORS}
\endissue{TYPE-OF-AND-PREDEFINED-CLASSES:UNIFY-AND-EXTEND}

\endcom%{string-stream}\ftype{Type}
\begincom{synonym-stream}\ftype{System Class}

\issue{TYPE-OF-AND-PREDEFINED-CLASSES:UNIFY-AND-EXTEND}
\issue{STREAM-ACCESS:ADD-TYPES-ACCESSORS}

\label Class Precedence List::

\typeref{synonym-stream},
\typeref{stream},
\typeref{t}

\label Description::

A \term{stream} that is an alias for another \term{stream},
which is the \term{value} of a \term{dynamic variable}
whose \term{name} is the \term{synonym stream symbol} of the \term{synonym stream}.

Any operations on a \term{synonym stream} will be performed 
on the \term{stream} that is then the \term{value} of the
\term{dynamic variable} named by the \term{synonym stream symbol}.
If the \term{value} of the \term{variable} should change,
or if the \term{variable} should be \term{bound},
then the \term{stream} will operate on the new \term{value} of the \term{variable}.

\label See Also::

\funref{make-synonym-stream},
\funref{synonym-stream-symbol}

%% This should go without saying.
% \label Notes::
% 
% A \term{synonym stream} is created by \funref{make-synonym-stream}.

\endissue{STREAM-ACCESS:ADD-TYPES-ACCESSORS}
\endissue{TYPE-OF-AND-PREDEFINED-CLASSES:UNIFY-AND-EXTEND}

\endcom%{synonym-stream}\ftype{System Class}
\begincom{two-way-stream}\ftype{System Class}

\issue{TYPE-OF-AND-PREDEFINED-CLASSES:UNIFY-AND-EXTEND}
\issue{STREAM-ACCESS:ADD-TYPES-ACCESSORS}

\label Class Precedence List::

\typeref{two-way-stream},
\typeref{stream},
\typeref{t}

\label Description::

A \term{bidirectional} \term{composite stream} that 
     receives its input  from an associated \term{input}  \term{stream} 
 and sends    its output to   an associated \term{output} \term{stream}.

\label See Also::

\funref{make-two-way-stream},
\funref{two-way-stream-input-stream},
\funref{two-way-stream-output-stream}

\endissue{STREAM-ACCESS:ADD-TYPES-ACCESSORS}

\endissue{TYPE-OF-AND-PREDEFINED-CLASSES:UNIFY-AND-EXTEND}

\endcom%{two-way-stream}\ftype{System Class}

%-------------------- Streams --------------------

%%% ========== INPUT-STREAM-P
%%% ========== OUTPUT-STREAM-P
\begincom{input-stream-p, output-stream-p}\ftype{Function}

\label Syntax::

\DefunWithValues input-stream-p {stream} {generalized-boolean}
\DefunWithValues output-stream-p {stream} {generalized-boolean}

\label Arguments and Values::

\param{stream}---a \term{stream}.

\param{generalized-boolean}---a \term{generalized boolean}.

\label Description::

%% 21.3.0 4
\NamedPredicate{input-stream-p}{stream}{an \term{input} \term{stream}}

%% 21.3.0 5
\NamedPredicate{output-stream-p}{stream}{an \term{output} \term{stream}}

\label Examples::

\code
 (input-stream-p *standard-input*) \EV \term{true}
 (input-stream-p *terminal-io*) \EV \term{true}
 (input-stream-p (make-string-output-stream)) \EV \term{false}

 (output-stream-p *standard-output*) \EV \term{true}
 (output-stream-p *terminal-io*) \EV \term{true}
 (output-stream-p (make-string-input-stream "jr")) \EV \term{false}
\endcode

\label Side Effects:\None!

\label Affected By:\None.

\label Exceptional Situations::

\Shouldchecktype{stream}{a \term{stream}}

\label See Also:\None.

\label Notes:\None.

\endcom

%%% ========== INTERACTIVE-STREAM-P
\begincom{interactive-stream-p}\ftype{Function}

\issue{STREAM-CAPABILITIES:INTERACTIVE-STREAM-P}

\label Syntax::

\DefunWithValues interactive-stream-p {stream} {generalized-boolean}

\label Arguments and Values::

\param{stream}---a \term{stream}.

\param{generalized-boolean}---a \term{generalized boolean}.

\label Description::

\Predicate{stream}{an \term{interactive stream}}
 
\label Examples::

\code
 (when (> measured limit)
   (let ((error (round (* (- measured limit) 100)
                       limit)))
     (unless (if (interactive-stream-p *query-io*)
                 (yes-or-no-p "The frammis is out of tolerance by ~D%.~@
                               Is it safe to proceed? " error)
                 (< error 15))  ;15% is acceptable
       (error "The frammis is out of tolerance by ~D%." error))))
\endcode

\label Affected By:\None.

\label Exceptional Situations::

% The cleanup said "signals", but that's stylistically inconsistent.
% Barrett and I decided to just ignore the cleanup and fix things up. -kmp 22-Jan-92
\Shouldchecktype{stream}{a \term{stream}}

\label See Also::

{\secref\StreamConcepts}

\label Notes:\None.

\endissue{STREAM-CAPABILITIES:INTERACTIVE-STREAM-P}

\endcom

%%% ========== OPEN-STREAM-P
\begincom{open-stream-p}\ftype{Function}

\issue{STREAM-ACCESS:ADD-TYPES-ACCESSORS}

\label Syntax::

\DefunWithValues open-stream-p {stream} {generalized-boolean}

\label Arguments and Values::

\param{stream}---a \term{stream}.

\param{generalized-boolean}---a \term{generalized boolean}.

\label Description::

\Predicate{stream}{an \term{open} \term{stream}}
 
\term{Streams} are open until they have been explicitly closed with \funref{close},
or until they are implicitly closed due to exit from a
  \macref{with-output-to-string}, 
  \macref{with-open-file},
  \macref{with-input-from-string},  or 
  \macref{with-open-stream} \term{form}.

\label Examples::

\code
 (open-stream-p *standard-input*) \EV \term{true}
\endcode

\label Side Effects:\None!

\label Affected By::

\funref{close}.

\label Exceptional Situations::

\Shouldchecktype{stream}{a \term{stream}}

\label See Also:\None.

\label Notes:\None.

\endissue{STREAM-ACCESS:ADD-TYPES-ACCESSORS}

\endcom

%%% ========== STREAM-ELEMENT-TYPE
\begincom{stream-element-type}\ftype{Function}

\label Syntax::

\DefunWithValues stream-element-type {stream} {typespec}

\label Arguments and Values::

\param{stream}---a \term{stream}.

\param{typespec}---a \term{type specifier}.

\label Description::

%% 21.3.0 6
\funref{stream-element-type} returns a \term{type specifier} that
indicates the \term{types} of \term{objects} that may be read from 
or written to \param{stream}.

\term{Streams} created by \funref{open} have an \term{element type}
restricted to \typeref{integer} or a \subtypeof{character}.
%%KMP (2-Jan-91): How?? What would it mean?  Does this have formal impact?
% but in principle a \term{stream} may conduct transactions using any
% \term{objects}.
%% Removed on instruction of Barrett, who concurred, saying:
%%  I'm not convinced this is even true, since OPEN just says "acceptable",
%%  which I read as saying undefined consequences, which permits an implementation
%%  to extend to allow other kinds of objects.  Even without that, the statement 
%%  isn't all that interesting, since OPEN only defined to deal with filestreams,
%%   and an implementation could provide other kinds of streams which pass aribitrary objects.

\label Examples::

\code
;; Note that the stream must accomodate at least the specified type,
;; but might accomodate other types.  Further note that even if it does
;; accomodate exactly the specified type, the type might be specified in
;; any of several ways.
 (with-open-file (s "test" :element-type '(integer 0 1)
                           :if-exists :error
                           :direction :output)
   (stream-element-type s))
\EV INTEGER
\OV (UNSIGNED-BYTE 16)
\OV (UNSIGNED-BYTE 8)
\OV BIT
\OV (UNSIGNED-BYTE 1)
\OV (INTEGER 0 1)
\OV (INTEGER 0 (2))
\endcode

\label Side Effects:\None.

\label Affected By:\None.

\label Exceptional Situations::

\Shouldchecktype{stream}{a \term{stream}}

\label See Also:\None.

\label Notes:\None.

\endcom

%%% ========== STREAMP
\begincom{streamp}\ftype{Function}

\label Syntax::

\DefunWithValues {streamp} {object} {generalized-boolean}

\label Arguments and Values::

\param{object}---an \term{object}.

\param{generalized-boolean}---a \term{generalized boolean}.

\label Description::

%% 21.3.0 3
\TypePredicate{object}{stream}

\issue{CLOSED-STREAM-FUNCTIONS:ALLOW-INQUIRY}
\funref{streamp} is unaffected by whether \param{object},
if it is a \term{stream}, is \term{open} or closed.
\endissue{CLOSED-STREAM-FUNCTIONS:ALLOW-INQUIRY}

\label Examples::

\code
 (streamp *terminal-io*) \EV \term{true}
 (streamp 1) \EV \term{false}
\endcode

\label Side Effects:\None.

\label Affected By:\None.

\label Exceptional Situations:\None!

\label See Also:\None.

\label Notes::

\code
 (streamp \param{object}) \EQ (typep \param{object} 'stream)
\endcode

\endcom

%-------------------- Byte Io --------------------

%%% ========== READ-BYTE
\begincom{read-byte}\ftype{Function}

\label Syntax::

\DefunWithValues {read-byte} {stream {\opt} eof-error-p eof-value} {byte}

\label Arguments and Values::

\param{stream}---a \term{binary} \term{input} \term{stream}.

\param{eof-error-p}---a \term{generalized boolean}.
 \Default{\term{true}}

\issue{ARGUMENTS-UNDERSPECIFIED:SPECIFY}
\param{eof-value}---an \term{object}.
 \Default{\nil}
\endissue{ARGUMENTS-UNDERSPECIFIED:SPECIFY}

\param{byte}---an \term{integer},
%% Removed by KMP because in the presence of stream element type upgrading, 
%% this won't necessarily work as expected if the file was written with an
%% incompatible element type.
%  of the \term{stream element type}
% 		  of \term{stream}
 or the \param{eof-value}.

\label Description::

%% 22.2.2 3
\funref{read-byte} reads and returns one byte from \param{stream}.

If an \term{end of file}\meaning{2} occurs and \param{eof-error-p} is \term{false}, 
the \param{eof-value} is returned.  

\label Examples::
\code
 (with-open-file (s "temp-bytes" 
                     :direction :output
                     :element-type 'unsigned-byte)
    (write-byte 101 s)) \EV 101
 (with-open-file (s "temp-bytes" :element-type 'unsigned-byte)
    (format t "~S ~S" (read-byte s) (read-byte s nil 'eof)))
\OUT 101 EOF
\EV NIL
\endcode

\label Side Effects::

Modifies \param{stream}.

\label Affected By:\None.

\label Exceptional Situations::

\Shouldchecktype{stream}{a \term{stream}}

Should signal an error \oftype{error} 
if \param{stream} is not  a \term{binary} \term{input} \term{stream}.

If there are no \term{bytes} remaining in the \param{stream} 
and \param{eof-error-p} is \term{true}, an error \oftype{end-of-file} is signaled.

\label See Also::

\funref{read-char},
\issue{READ-AND-WRITE-BYTES:NEW-FUNCTIONS}
\funref{read-sequence},
\endissue{READ-AND-WRITE-BYTES:NEW-FUNCTIONS}
\funref{write-byte}

\label Notes:\None.

\endcom

%%% ========== WRITE-BYTE
\begincom{write-byte}\ftype{Function}

\label Syntax::

\DefunWithValues {write-byte} {byte stream} {byte}

\label Arguments and Values::

\param{byte}---an \term{integer} of the \term{stream element type}
		  of \term{stream}.

\param{stream}---a \term{binary} \term{output} \term{stream}.

\label Description::

%% 22.3.2 4
\funref{write-byte} writes one byte, \param{byte}, to \param{stream}.

%The size of the byte written depends on the {\bf :element-type} argument
%from \funref{open} or \macref{with-open-file}.  
%Unless the byte size of that
%element type is one, two, or four bits, each call to \funref{write-byte}
%generates an integral number of 8-bit bytes, namely the minimum number
%necessary to hold the number of bits indicated by the given element type.
%If the byte size of the element type is one, two, or four bits, then as many
%elements as possible (eight, four, or two respectively) are packed into
%each 8-bit byte.

\label Examples::

\code
 (with-open-file (s "temp-bytes" 
                    :direction :output
                    :element-type 'unsigned-byte)
    (write-byte 101 s)) \EV 101
\endcode

\label Side Effects::

\param{stream} is modified.

\label Affected By::

The \term{element type} of the \param{stream}.

\label Exceptional Situations::

\Shouldchecktype{stream}{a \term{stream}}
Should signal an error \oftype{error} 
if \param{stream} is not  a \term{binary} \term{output} \term{stream}.

Might signal an error \oftype{type-error} if \param{byte} is not
an \term{integer} of the \term{stream element type} of \term{stream}.

\label See Also::

\funref{read-byte},
\funref{write-char},
\issue{READ-AND-WRITE-BYTES:NEW-FUNCTIONS}
\funref{write-sequence}
\endissue{READ-AND-WRITE-BYTES:NEW-FUNCTIONS}

\label Notes:\None.

\endcom

%-------------------- Character Io --------------------

%%% ========== PEEK-CHAR
\begincom{peek-char}\ftype{Function}

\label Syntax::

\DefunWithValues {peek-char}
		 {{\opt} \vtop{\hbox{peek-type input-stream eof-error-p}
			       \hbox{eof-value recursive-p}}}
		 {char}

\label Arguments and Values::

\param{peek-type}---a \term{character} or \t\ or \nil.

\param{input-stream}---\term{input} \term{stream designator}.
  \Default{\term{standard input}}

\param{eof-error-p}---a \term{generalized boolean}.
 \Default{\term{true}}

\issue{ARGUMENTS-UNDERSPECIFIED:SPECIFY}
\param{eof-value}---an \term{object}.
 \Default{\nil}
\endissue{ARGUMENTS-UNDERSPECIFIED:SPECIFY}

\param{recursive-p}---a \term{generalized boolean}.
 \Default{\term{false}}

\param{char}---a \term{character} or the \param{eof-value}.

\label Description::

\funref{peek-char} obtains the next character in \param{input-stream}
without actually reading it, thus leaving the character
to be read at a later time.  It can
also be used to skip over and discard intervening
characters in the \param{input-stream} 
until a particular character is found.

%% 22.2.1 33
If \param{peek-type} is not supplied or \nil,
\funref{peek-char} returns the next character to be read from 
\param{input-stream}, without actually removing it from 
\param{input-stream}.
The next time input is done from \param{input-stream}, the character will still
be there.  
%% 22.2.1 34
If \param{peek-type} is \t,
then \funref{peek-char} skips over \term{whitespace}\meaning{2} \term{characters},
but not comments,
and then performs the peeking operation on the next
character.
The last character examined, the one that starts an \term{object},
is not removed from \param{input-stream}.
%% 22.2.1 35
If \param{peek-type} is a \term{character},
then \funref{peek-char} skips
over input characters until a character that
is \funref{char=} to that \term{character} is found;          
that character is left in \param{input-stream}.

If an \term{end of file}\meaning{2} occurs and \param{eof-error-p} is \term{false},
\param{eof-value} is returned.  

{$ $}{\ExplainRecursiveP}

\issue{PEEK-CHAR-READ-CHAR-ECHO:FIRST-READ-CHAR}
When \param{input-stream} is an \term{echo stream},
characters that are only peeked at are not echoed. In the
case that \param{peek-type} is not \nil,
the characters that are passed by \funref{peek-char} 
are treated as if by \funref{read-char},
and so are echoed unless they have been marked otherwise by \funref{unread-char}.
\endissue{PEEK-CHAR-READ-CHAR-ECHO:FIRST-READ-CHAR}


\label Examples::
\code
 (with-input-from-string (input-stream "    1 2 3 4 5")
    (format t "~S ~S ~S" 
            (peek-char t input-stream)
            (peek-char #\\4 input-stream)
            (peek-char nil input-stream)))
\OUT #\\1 #\\4 #\\4
\EV NIL
\endcode

\label Affected By::

\varref{*readtable*},
\varref{*standard-input*},
\varref{*terminal-io*}.

\label Exceptional Situations::

If \param{eof-error-p} is \term{true} and an \term{end of file}\meaning{2} occurs
an error \oftype{end-of-file} is signaled.

If     \param{peek-type} is a \term{character},
       an \term{end of file}\meaning{2} occurs,
   and \param{eof-error-p} is \term{true},
an error \oftype{end-of-file} is signaled.

%!!! Barrett is concerned about the case of eof-error-p here.
If \param{recursive-p} is \term{true}
and an \term{end of file}\meaning{2} occurs,
an error \oftype{end-of-file} is signaled.

\label See Also:\None.

\label Notes:\None.

\endcom

%%% ========== READ-CHAR
\begincom{read-char}\ftype{Function}

\label Syntax::

\DefunWithValues {read-char}
		 {{\opt} input-stream eof-error-p eof-value recursive-p}
		 {char}

\label Arguments and Values::

\param{input-stream}---an \term{input} \term{stream designator}.
 \Default{\term{standard input}}

\param{eof-error-p}---a \term{generalized boolean}.
 \Default{\term{true}}

\issue{ARGUMENTS-UNDERSPECIFIED:SPECIFY}
\param{eof-value}---an \term{object}.
 \Default{\nil}
\endissue{ARGUMENTS-UNDERSPECIFIED:SPECIFY}

\param{recursive-p}---a \term{generalized boolean}.
 \Default{\term{false}}

\param{char}---a \term{character} or the \param{eof-value}.

\label Description::

%% 22.2.1 28
\funref{read-char} returns the next \term{character} from \param{input-stream}.

\issue{PEEK-CHAR-READ-CHAR-ECHO:FIRST-READ-CHAR}
When \param{input-stream} is an \param{echo stream},
the character is echoed on \param{input-stream} the first time the character is 
%Sandra: "seen"? awfully vague.
seen.
  Characters that are not echoed by \funref{read-char} 
are those that were
  put there by \funref{unread-char} 
and hence are assumed to have been echoed
  already by a previous call to \funref{read-char}.
\endissue{PEEK-CHAR-READ-CHAR-ECHO:FIRST-READ-CHAR}

{$ $}{\ExplainRecursiveP}

If an \term{end of file}\meaning{2} occurs and \param{eof-error-p} is \term{false}, 
\param{eof-value} is returned.  

\label Examples::
\code
 (with-input-from-string (is "0123")
    (do ((c (read-char is) (read-char is nil 'the-end)))
        ((not (characterp c)))
     (format t "~S " c)))
\OUT #\\0 #\\1 #\\2 #\\3
\EV NIL
\endcode


\label Affected By::

\varref{*standard-input*},
\varref{*terminal-io*}.

\label Exceptional Situations::

If an \term{end of file}\meaning{2} occurs before a character can be read, and 
\param{eof-error-p} is \term{true}, 
an error \oftype{end-of-file} is signaled.

\label See Also::

\funref{read-byte},
\issue{READ-AND-WRITE-BYTES:NEW-FUNCTIONS}
\funref{read-sequence},
\endissue{READ-AND-WRITE-BYTES:NEW-FUNCTIONS}
\funref{write-char},
\funref{read}

\label Notes::
%% 22.2.1 29
The corresponding output function is \funref{write-char}.

\endcom

%%% ========== READ-CHAR-NO-HANG
\begincom{read-char-no-hang}\ftype{Function}

\label Syntax::

\DefunWithValues read-char-no-hang
		 {{\opt} \vtop{\hbox{input-stream eof-error-p}
			       \hbox{eof-value recursive-p}}}
		 {char}

\label Arguments and Values::

\param{input-stream} -- an \term{input} \term{stream designator}.
 \Default{\term{standard input}}

\param{eof-error-p}---a \term{generalized boolean}.
 \Default{\term{true}}

\issue{ARGUMENTS-UNDERSPECIFIED:SPECIFY}
\param{eof-value}---an \term{object}.
 \Default{\nil}
\endissue{ARGUMENTS-UNDERSPECIFIED:SPECIFY}

\param{recursive-p}---a \term{generalized boolean}.
 \Default{\term{false}}

\param{char}---a \term{character} or \nil\ or the \param{eof-value}.

\label Description::

\funref{read-char-no-hang} returns a character
from \param{input-stream} if such a character is available.  If no character
is available, \funref{read-char-no-hang} returns \nil.

{$ $}{\ExplainRecursiveP}

If an \term{end of file}\meaning{2} occurs and \param{eof-error-p} is \term{false}, 
\param{eof-value} is returned.  

\label Examples::

%!!! This example suffers from the `newline problem' -kmp 8-May-91
\code
;; This code assumes an implementation in which a newline is not
;; required to terminate input from the console.
 (defun test-it ()
   (unread-char (read-char))
   (list (read-char-no-hang) 
         (read-char-no-hang) 
         (read-char-no-hang)))
\EV TEST-IT
;; Implementation A, where a Newline is not required to terminate
;; interactive input on the console.
 (test-it)
\OUT \IN{a}
\EV (#\\a NIL NIL)
;; Implementation B, where a Newline is required to terminate
;; interactive input on the console, and where that Newline remains
;; on the input stream.
 (test-it)
\OUT \IN{a\CRLF}
\EV (#\\a #\\Newline NIL)
\endcode

\label Affected By::

\varref{*standard-input*},
\varref{*terminal-io*}.

\label Exceptional Situations::

If an \term{end of file}\meaning{2} occurs
when \param{eof-error-p} is \term{true},
an error \oftype{end-of-file} is signaled .

\label See Also::

\funref{listen}

\label Notes::

%% 22.2.1 37
\funref{read-char-no-hang} is exactly like \funref{read-char}, except
that if it would be necessary to wait in order to get a character (as
from a keyboard), \nil\ is immediately returned without waiting.  

%!!! Symbolics doesn't echo the char from this. Should it? -kmp

\endcom

%%% ========== TERPRI
%%% ========== FRESH-LINE
\begincom{terpri, fresh-line}\ftype{Function}

\label Syntax::

\DefunWithValues {terpri}     {{\opt} output-stream} {\nil}
\DefunWithValues {fresh-line} {{\opt} output-stream} {generalized-boolean}

\label Arguments and Values::

\param{output-stream} -- an \term{output} \term{stream designator}.
 \Default{\term{standard output}}

\param{generalized-boolean}---a \term{generalized boolean}.

\label Description::

\funref{terpri} outputs a \term{newline} to \param{output-stream}.

%% 22.3.1 18

\funref{fresh-line} is similar to \funref{terpri} but outputs a \term{newline}
only if the \param{output-stream} is not already at the start of a line.
If for some reason this cannot be determined, then a \term{newline} is output anyway.
\funref{fresh-line} returns \term{true} if it outputs a \term{newline};
otherwise it returns \term{false}.

\label Examples::

\code
 (with-output-to-string (s)
    (write-string "some text" s)
    (terpri s)
    (terpri s)
    (write-string "more text" s))
\EV "some text

more text"
 (with-output-to-string (s)
    (write-string "some text" s)
    (fresh-line s)
    (fresh-line s)
    (write-string "more text" s))
\EV "some text
more text"
\endcode

\label Side Effects::

The \param{output-stream} is modified.

\label Affected By::

\varref{*standard-output*},
\varref{*terminal-io*}.

\label Exceptional Situations::

None.
\reviewer{Barmar: What if stream is closed?}%!!!

\label See Also:\None.

\label Notes::

\funref{terpri} is identical in effect to

\code
 (write-char #\\Newline output-stream)
\endcode

\endcom

%%% ========== UNREAD-CHAR
\begincom{unread-char}\ftype{Function}

\label Syntax::

%% 22.2.1 30
\DefunWithValues {unread-char} {character {\opt} input-stream} {\nil}

\label Arguments and Values::

%% 22.2.1 31
\param{character}---a \term{character};
  must be the last \term{character} that was read from \param{input-stream}.

\param{input-stream}---an \term{input} \term{stream designator}.
 \Default{\term{standard input}}

\label Description::

\funref{unread-char} places \param{character} back onto the front of
\param{input-stream} so that it will again be the next character
in \param{input-stream}.

\issue{PEEK-CHAR-READ-CHAR-ECHO:FIRST-READ-CHAR}
When \param{input-stream} is an \term{echo stream},
no attempt is made to undo any echoing of the character that might already
have been done on \param{input-stream}. However, characters placed on 
\param{input-stream} by \funref{unread-char} are marked in such a way 
as to inhibit later re-echo by \funref{read-char}.
\endissue{PEEK-CHAR-READ-CHAR-ECHO:FIRST-READ-CHAR}

It is an error to invoke \funref{unread-char} 
twice consecutively on the same \term{stream}
without an intervening call to \funref{read-char} 
(or some other input operation which implicitly reads characters)
on that \term{stream}.

\issue{UNREAD-CHAR-AFTER-PEEK-CHAR:DONT-ALLOW}
Invoking \funref{peek-char} or \funref{read-char} commits all previous characters.
The consequences of invoking \funref{unread-char}
on any character preceding that which is returned by 
\funref{peek-char} (including those passed over by 
\funref{peek-char} that has a \term{non-nil} \param{peek-type})
are unspecified.
In particular, the consequences of 
invoking \funref{unread-char} after \funref{peek-char}
are unspecified.
\endissue{UNREAD-CHAR-AFTER-PEEK-CHAR:DONT-ALLOW}

\label Examples::

\code
 (with-input-from-string (is "0123")
    (dotimes (i 6)
      (let ((c (read-char is)))
        (if (evenp i) (format t "~&~S ~S~%" i c) (unread-char c is)))))
\OUT 0 #\\0
\OUT 2 #\\1
\OUT 4 #\\2
\EV NIL
\endcode

\label Affected By::

\varref{*standard-input*},
\varref{*terminal-io*}.

\label Exceptional Situations:\None.

\label See Also::

\funref{peek-char},
\funref{read-char},
{\secref\StreamConcepts}

\label Notes::

%% 22.2.1 32
\funref{unread-char} is intended to be an efficient mechanism for allowing
the \term{Lisp reader} and other parsers to perform one-character lookahead
in \param{input-stream}.

\endcom

%%% ========== WRITE-CHAR
\begincom{write-char}\ftype{Function}

\label Syntax::

\DefunWithValues {write-char} {character {\opt} output-stream} {character}

\label Arguments and Values::

\param{character}---a \term{character}.

\param{output-stream} -- an \term{output} \term{stream designator}.
 \Default{\term{standard output}}

\label Description::

%% 22.3.1 12
\funref{write-char} outputs \param{character} to \param{output-stream}.

\label Examples::
\code
 (write-char #\\a)
\OUT a
\EV #\\a
 (with-output-to-string (s) 
   (write-char #\\a s)
   (write-char #\\Space s)
   (write-char #\\b s))
\EV "a b"
\endcode

\label Side Effects::

The \param{output-stream} is modified.

\label Affected By::

\varref{*standard-output*},
\varref{*terminal-io*}.

\label Exceptional Situations:\None.

\label See Also::

\funref{read-char},
\funref{write-byte},
\issue{READ-AND-WRITE-BYTES:NEW-FUNCTIONS}
\funref{write-sequence}
\endissue{READ-AND-WRITE-BYTES:NEW-FUNCTIONS}

\label Notes:\None.

\endcom

%-------------------- String I/O --------------------

%%% ========== READ-LINE
\begincom{read-line}\ftype{Function}

\label Syntax::

\DefunWithValuesNewline read-line
       		 	{{\opt} input-stream eof-error-p eof-value recursive-p}
		 	{line, missing-newline-p}
%!!! Barrett wonders if the second return value should really be required
%    when the first value is an eof-value...

\label Arguments and Values::

\param{input-stream}---an \term{input} \term{stream designator}.
 \Default{\term{standard input}}

\param{eof-error-p}---a \term{generalized boolean}.
 \Default{\term{true}}

\issue{ARGUMENTS-UNDERSPECIFIED:SPECIFY}
\param{eof-value}---an \term{object}.
 \Default{\nil}
\endissue{ARGUMENTS-UNDERSPECIFIED:SPECIFY}

\param{recursive-p}---a \term{generalized boolean}.
 \Default{\term{false}}

\param{line}---a \term{string} or the \param{eof-value}.

\param{missing-newline-p}---a \term{generalized boolean}.

\label Description::

%% 22.2.1 26
Reads from \param{input-stream} a line of text
that is terminated by a \term{newline} or \term{end of file}.
%% This should follow without saying. -kmp 11-Feb-92
%Interactively, this \term{function} can be used to get a line of input from the \term{user}.

{$ $}{\ExplainRecursiveP}

The \term{primary value}, \param{line}, is the line that is read,
represented as a \term{string} (without the trailing \term{newline}, if any).
If  \param{eof-error-p} is \term{false} 
and the \term{end of file} for \param{input-stream} is reached
     before any \term{characters} are read,
\param{eof-value} is returned as the \param{line}.

The \term{secondary value}, \param{missing-newline-p}, 
is a \term{generalized boolean} that is
    \term{false} if the \param{line} was terminated by a \term{newline},
 or \term{true}  if the \param{line} was terminated by 
		   the \term{end of file} for \param{input-stream}
%KMP: What's the right second value when the first return value is the eof flag?
%Moon: Surely true, since there was no newline.
%KMP: Added the following:
		   (or if the \param{line} is the \param{eof-value}).

\label Examples::

\code
 (setq a "line 1
 line2")
\EV "line 1
 line2"
 (read-line (setq input-stream (make-string-input-stream a)))
\EV "line 1", \term{false}
 (read-line input-stream)
\EV "line2", \term{true}
 (read-line input-stream nil nil)
\EV NIL, \term{true}
\endcode

\label Side Effects:\None.

\label Affected By::

\varref{*standard-input*},
\varref{*terminal-io*}.

\label Exceptional Situations::

If an \term{end of file}\meaning{2} occurs before any characters are read in the line,
an error is signaled if \param{eof-error-p} is \term{true}.  

\label See Also::

\funref{read}

\label Notes::

%% 22.2.1 27
The corresponding output function is \funref{write-line}.

\endcom

%%% ========== WRITE-STRING
%%% ========== WRITE-LINE
\begincom{write-string, write-line}\ftype{Function}

\label Syntax::

\DefunWithValues {write-string} {string {\opt} output-stream {\key} start end} {string}
\DefunWithValues {write-line}   {string {\opt} output-stream {\key} start end} {string}

\label Arguments and Values::

\param{string}---a \term{string}.

\param{output-stream} -- an \term{output} \term{stream designator}.
 \Default{\term{standard output}}

\issue{SUBSEQ-OUT-OF-BOUNDS}
\issue{RANGE-OF-START-AND-END-PARAMETERS:INTEGER-AND-INTEGER-NIL}
\param{start}, \param{end}---\term{bounding index designators} of \param{string}.
 \Defaults{\param{start} and \param{end}}{\f{0} and \nil}
\endissue{RANGE-OF-START-AND-END-PARAMETERS:INTEGER-AND-INTEGER-NIL}
\endissue{SUBSEQ-OUT-OF-BOUNDS}

\label Description::

%% 22.3.1 13
\funref{write-string} writes the \term{characters} of 
the subsequence of \param{string} \term{bounded} by \param{start} and \param{end}
to \param{output-stream}.
\funref{write-line} does the same thing, 
but then outputs a newline afterwards.  

%% Redundant with "bounded by" above. -kmp 23-Jan-92
% \param{start} specifies the offset into \param{string}.
% \param{end} marks the the position following the last element of the substring.

%% 22.3.1 16
%% this paragraph was left out

\label Examples::

\code
 (prog1 (write-string "books" nil :end 4) (write-string "worms"))
\OUT bookworms
\EV "books"
 (progn (write-char #\\*)
        (write-line "test12" *standard-output* :end 5) 
        (write-line "*test2")
        (write-char #\\*)
        nil)
\OUT *test1
\OUT *test2
\OUT *
\EV NIL
\endcode

\label Side Effects:\None.

\label Affected By::

\varref{*standard-output*},
\varref{*terminal-io*}.

\label Exceptional Situations:\None.

\label See Also::

\funref{read-line}, \funref{write-char}

\label Notes::

\funref{write-line} and \funref{write-string} return \param{string},
not the substring \term{bounded} by \param{start} and \param{end}.

\code
 (write-string string)
\EQ (dotimes (i (length string)
      (write-char (char string i)))

 (write-line string)
\EQ (prog1 (write-string string) (terpri))
\endcode
%The WRITE-LINE equivalence supplied by Barmar.

\endcom

%-------------------- Block-Mode I/O --------------------

\issue{READ-AND-WRITE-BYTES:NEW-FUNCTIONS}

%%% ========== READ-SEQUENCE
\begincom{read-sequence}\ftype{Function}

\label Syntax::

\DefunWithValues {read-sequence} {sequence stream {\key} start end} {position}

\param{sequence}---a \term{sequence}.

\param{stream}---an \term{input} \term{stream}.

\param{start}, \param{end}---\term{bounding index designators} of
 \param{sequence}.  \Defaults{\param{start} and \param{end}}{\f{0} and \nil}

\param{position}---an \term{integer} greater than or equal to zero, and
 less than or equal to the \term{length} of the \param{sequence}.

\label Description::

Destructively modifies \param{sequence} by replacing the \term{elements}
of \param{sequence} \term{bounded} by \param{start} and \param{end} with
\term{elements} read from \param{stream}.

\param{Sequence} is destructively modified by copying successive
\term{elements} into it from \param{stream}.  If the \term{end of file} for
\param{stream} is reached before copying all \term{elements} of the
subsequence, then the extra \term{elements} near the end of \param{sequence}
are not updated.

\param{Position} is the index of the first \term{element} of \param{sequence}
that was not updated, which might be less than \param{end} because the
\term{end of file} was reached.

\label Examples::

\code
 (defvar *data* (make-array 15 :initial-element nil))
 (values (read-sequence *data* (make-string-input-stream "test string")) *data*)
 \EV 11, #(#\\t #\\e #\\s #\\t #\\Space #\\s #\\t #\\r #\\i #\\n #\\g NIL NIL NIL NIL)
\endcode

\label Side Effects::

Modifies \param{stream} and \param{sequence}.

\label Affected By:\None.

\label Exceptional Situations::

\Lazychecktype{sequence}{a \term{proper sequence}}
\Shouldchecktype{start}{a non-negative \term{integer}}
\Shouldchecktype{end}{a non-negative \term{integer} or \nil}

Might signal an error \oftype{type-error} if an \term{element} read from
the \param{stream} is not a member of the \term{element type} of the
\param{sequence}.

\label See Also::

{\secref\ConstantModification},
\funref{write-sequence},
\funref{read-line}

\label Notes::

\funref{read-sequence} is identical in effect to iterating over the indicated
subsequence and reading one \term{element} at a time from \param{stream} and
storing it into \param{sequence}, but may be more efficient than the
equivalent loop.  An efficient implementation is more likely to exist
for the case where the \param{sequence} is a \term{vector} with the same
\term{element type} as the \param{stream}.

\endcom

%%% ========== WRITE-SEQUENCE
\begincom{write-sequence}\ftype{Function}

\label Syntax::

\DefunWithValues {write-sequence} {sequence stream {\key} start end} {sequence}

\param{sequence}---a \term{sequence}.

\param{stream}---an \term{output} \term{stream}.

\param{start}, \param{end}---\term{bounding index designators} of
 \param{sequence}.  \Defaults{\param{start} and \param{end}}{\f{0} and \nil}

\label Description::

\funref{write-sequence} writes the \term{elements} of the subsequence
of \param{sequence} \term{bounded} by \param{start} and \param{end} to
\param{stream}.

\label Examples::

\code
 (write-sequence "bookworms" *standard-output* :end 4)
 \OUT book
 \EV "bookworms"
\endcode

\label Side Effects::

Modifies \param{stream}.

\label Affected By:\None.

\label Exceptional Situations::

\Lazychecktype{sequence}{a \term{proper sequence}}
\Shouldchecktype{start}{a non-negative \term{integer}}
\Shouldchecktype{end}{a non-negative \term{integer} or \nil}

Might signal an error \oftype{type-error} if an \term{element} of the
\term{bounded} \term{sequence} is not a member of the
\term{stream element type} of the \param{stream}.

\label See Also::

{\secref\ConstantModification},
\funref{read-sequence},
\funref{write-string},
\funref{write-line}

\label Notes::

\funref{write-sequence} is identical in effect to iterating over the indicated
subsequence and writing one \term{element} at a time to \param{stream}, but
may be more efficient than the equivalent loop.  An efficient implementation
is more likely to exist for the case where the \param{sequence} is a
\term{vector} with the same \term{element type} as the \param{stream}.

\endcom

\endissue{READ-AND-WRITE-BYTES:NEW-FUNCTIONS}


%-------------------- File Position --------------------

%%% ========== FILE-LENGTH
\begincom{file-length}\ftype{Function}

\label Syntax::

\DefunWithValues file-length {stream} {length}

\label Arguments and Values::

\param{stream}---a \term{stream associated with a file}.

\param{length}---a non-negative \term{integer} or \nil.

\label Description::

%% 23.3.0 21
\funref{file-length} returns the length of \param{stream},  
or \nil\ if the length cannot be determined.

For a binary file, the length is measured in units of
the \term{element type} of the \param{stream}.

\label Examples::

\code
 (with-open-file (s "decimal-digits.text" 
                    :direction :output :if-exists :error)
   (princ "0123456789" s)
   (truename s))
\EV #P"A:>Joe>decimal-digits.text.1"
 (with-open-file (s "decimal-digits.text")
   (file-length s))
\EV 10
\endcode

\label Side Effects:\None.

\label Affected By:\None.

\label Exceptional Situations::

\Shouldchecktype{stream}{a \term{stream associated with a file}}
%!!! Is this going to run into trouble in cases like:
%    (with-input-from-string (s "foo") (file-length s))
%    which an implementation might want to support even thought it's not required?
%    -kmp 26-Apr-91

\label See Also::

\funref{open}

\label Notes:\None.

\endcom

%%% ========== FILE-POSITION
\begincom{file-position}\ftype{Function}

\label Syntax::

\DefunWithValues file-position {stream} {position}
\DefunWithValues file-position {stream position-spec} {success-p}

\label Arguments and Values::

\param{stream}---a \term{stream}.

\param{position-spec}---a \term{file position designator}.

\param{position}---a \term{file position} or \nil.

\param{success-p}---a \term{generalized boolean}.

\label Description::

%% 23.3.0 16
Returns or changes the current position within a \param{stream}.

%% 23.3.0 17
When \param{position-spec} is not supplied,
\funref{file-position} returns the current \term{file position} in the \param{stream},
or \nil\ if this cannot be determined.

%% 23.3.0 18
When \param{position-spec} is supplied,
the \term{file position} in \param{stream} is set to that \term{file position} (if possible).
\funref{file-position} returns \term{true} 
if the repositioning is performed successfully,
or \term{false} if it is not.

%%KMP: I could find no reference which supported this claim. It looks well-defined to me.
%
% The consequences are undefined if \param{file-stream} 
% is not a \term{stream} that is open to a random-access file. 

An \term{integer} returned by \funref{file-position} of one argument 
should be acceptable as \param{position-spec} for use with the same file.

For a character file,
performing a single \funref{read-char} or \funref{write-char} operation
may cause the file position to be increased by more than 1 because of
character-set translations (such as translating between the \clisp\
\f{\#\\Newline} character and an external ASCII
carriage-return/line-feed sequence) and other aspects of the
implementation.  For a binary file, every \funref{read-byte} 
or \funref{write-byte}
operation increases the file position by 1.

\label Examples::

\code
 (defun tester ()
   (let ((noticed '()) file-written)
     (flet ((notice (x) (push x noticed) x))
       (with-open-file (s "test.bin" 
                          :element-type '(unsigned-byte 8)
                          :direction :output
                          :if-exists :error)
          (notice (file-position s)) ;1
          (write-byte 5 s) 
          (write-byte 6 s)
          (let ((p (file-position s)))
            (notice p) ;2
            (notice (when p (file-position s (1- p))))) ;3
          (write-byte 7 s)
          (notice (file-position s)) ;4
          (setq file-written (truename s)))
        (with-open-file (s file-written
                           :element-type '(unsigned-byte 8)
                           :direction :input)
          (notice (file-position s)) ;5
          (let ((length (file-length s)))
            (notice length) ;6
            (when length
              (dotimes (i length)
                (notice (read-byte s)))))) ;7,...
        (nreverse noticed))))
\EV tester
 (tester)
\EV (0 2 T 2 0 2 5 7)
\OV (0 2 NIL 3 0 3 5 6 7)
\OV (NIL NIL NIL NIL NIL NIL)
\endcode

\label Side Effects::

When the \param{position-spec} argument is supplied,
the \term{file position} in the \param{stream} might be moved.

\label Affected By::

The value returned by \funref{file-position} increases monotonically
as input or output operations are performed.  
          
\label Exceptional Situations::

%Note: KMP thinks there are implemetantions which might provide extensions to allow
%   non-integer position specs and that we should use special error terminology here
%   in order to give them some latitude.
If \param{position-spec} is supplied, but is too large or otherwise inappropriate, 
an error is signaled.

\label See Also::

\funref{file-length},
\funref{file-string-length},
\funref{open}

\label Notes::

%% 23.3.0 19
Implementations that have character files represented
as a sequence of records of bounded size might choose to encode the
file position as, for example,
\metavar{record-number}*\metavar{max-record-size}+\metavar{character-within-record}.
This is a valid encoding because it increases monotonically as
each character is read or written, though not necessarily by 1 at
each step.  An \term{integer} might then be considered ``inappropriate''
as \param{position-spec} to \funref{file-position} if, when decoded into
record number and character number, it turned out that the
supplied record was too short for the specified character number.

\endcom

%%% ========== FILE-STRING-LENGTH
\begincom{file-string-length}\ftype{Function}

\issue{CHARACTER-PROPOSAL:2-5-7}

\label Syntax::

\DefunWithValues file-string-length {stream object} {length}

\label Arguments and Values::

\param{stream}---an \term{output} \term{character} \term{file stream}.

\param{object}---a \term{string} or a \term{character}.

\param{length}---a non-negative \term{integer}, or \nil.

\label Description::

\funref{file-string-length} returns the difference between what 
\f{(file-position \param{stream})} would be after writing 
\param{object} and its current value, or \nil\ if this cannot be determined.  
 
The returned value corresponds to the current state of \param{stream}
at the time of the call and might not be 
% valid if it is called twice;
% two different values may occur.
the same if it is called again
when the state of the \term{stream} has changed.

\label Examples:\None.

\label Affected By:\None.

\label Exceptional Situations:\None.

\label See Also:\None.

\label Notes:\None.

\endissue{CHARACTER-PROPOSAL:2-5-7}

\endcom


%-------------------- File Opening --------------------

% %%% ========== OPEN
% \begincom{open}\ftype{Function}
% 
% \label Syntax::
% 
% \DefunWithValuesNewline open
% 		 	{filespec {\key} \vtop{\hbox{direction element-type}
% 				               \hbox{if-exists if-does-not-exist
% 						     external-format}}}
% 		        {stream}
% 
% \label Arguments and Values::
% 
% \param{filespec}---a \term{pathname designator}.
% 
% \param{direction}---one of \kwd{input}, \kwd{output}, \kwd{io}, or \kwd{probe}.
%  \Default{\kwd{input}}
% 
% %% 23.2.0 9
% %% 23.2.0 10
% %% 23.2.0 11
% %% 23.2.0 12
% %% 23.2.0 13
% %% 23.2.0 14
% %% 23.2.0 15
% %% 23.2.0 16
% %% 23.2.0 17
% %% 23.2.0 18
% %% 23.2.0 19
% \param{element-type}---a \term{type specifier} 
% 		       for \term{recognizable subtype} of \typeref{character};
% 		    or a \term{type specifier} 
% 		       for a \term{finite} \term{recognizable subtype} of \term{integer};
% 		    or one of the \term{symbols} 
%                           \misc{signed-byte},
%                           \misc{unsigned-byte},
% 		       or \kwd{default}.
% % \param{element-type}---\term{type specifier};
% %   a \term{subtype} of \typeref{character} or \typeref{integer};
% %   one of the following:
% %      \typeref{character},
% %      \f{(unsigned-byte \i{n})},
% %      \typeref{unsigned-byte},
% %      \f{(signed-byte \i{n})},
% %      \typeref{signed-byte},
% %      \typeref{bit},
% %      \f{(mod \i{n})},
% %   or \kwd{default},
% % where \i{n} is an \term{integer}.
%   \Default{\typeref{character}}
% 
% %% 23.2.0 21
% %% 23.2.0 22
% %% 23.2.0 23
% %% 23.2.0 24
% %% 23.2.0 25
% %% 23.2.0 26
% %% 23.2.0 27
% %% 23.2.0 28
% \param{if-exists}---one of \kwd{error}, \kwd{new-version}, \kwd{rename},
% 		  \kwd{rename-and-delete}, \kwd{overwrite}, \kwd{append},
% 		  \kwd{supersede}, or \nil.
% 
% %% 23.2.0 35
% %% 23.2.0 36
% \param{if-does-not-exist}---one of \kwd{error}, \kwd{create}, or \nil.
% 
% \issue{CHARACTER-PROPOSAL:2-5-2}
% \param{external-format}---an \term{external file format designator}.
%   \Default{\kwd{default}}
% \endissue{CHARACTER-PROPOSAL:2-5-2}
% 
% \issue{STREAM-ACCESS:ADD-TYPES-ACCESSORS}
% \param{stream}---a \term{file stream} or \nil.
% \endissue{STREAM-ACCESS:ADD-TYPES-ACCESSORS}
% 
% \label Description::
% 
% %% 23.2.0 2
% %% 23.2.0 3
% \funref{open} creates, opens, and returns a \term{file stream} 
% that is connected to the file specified by \param{filespec}.
% \param{Filespec} is the name of the file to be opened.  
% %% Moon: Mostly redundant.
% % \issue{CLOSED-STREAM-FUNCTIONS:ALLOW-INQUIRY}
% % If \param{filespec} is a \term{stream},
% % \param{filespec} effectively
% % becomes {\tt (pathname \param{filespec})}. \funref{open}
% % can be used on either an open or a closed \term{stream},
% % \endissue{CLOSED-STREAM-FUNCTIONS:ALLOW-INQUIRY}
% % and the \term{stream} is not
% % closed first or otherwise affected.
% % \issue{PATHNAME-STREAM}
% % Also, if \param{filespec} is a \term{stream}, 
% % that \term{stream}
% % can only have been originally opened by
% % \funref{open} or \macref{with-open-file}, 
% % or is a \term{synonym stream} whose \term{symbol} is bound
% % to a \term{stream} originally opened by
% % \funref{open} or \macref{with-open-file}.
% % If \param{filespec} is a \term{pathname} (as returned by \funref{pathname})
% % it represents the name used to open the file. This might be, but is
% % not required to be, the actual name of the file. 
% % \issue{PATHNAME-LOGICAL:ADD}
% % If \param{filespec} is a \term{logical pathname}, it is translated
% % into a physical pathname as if by calling \funref{translate-logical-pathname}.
% % \endissue{PATHNAME-LOGICAL:ADD}
% %% Salvaged from the above. -kmp 16-Feb-92
% If the \param{filespec} \term{designator} is a \term{stream},
% that \term{stream} is not closed first or otherwise affected.
% 
% %% Implied by pathname designator. -kmp 16-Feb-92
% % It is an error if \param{filespec} is a \term{stream} that is 
% % created with \funref{make-two-way-stream}, 
% % \funref{make-echo-stream},
% % \funref{make-broadcast-stream}, 
% % \funref{make-concatenated-stream}, \funref{make-string-input-stream},
% % \funref{make-string-output-stream}.
% %\endissue{PATHNAME-STREAM}
% 
% The keyword arguments to \funref{open} specify the characteristics 
% of the \term{file stream} that is returned, and how to handle errors.
% 
% %% 23.2.0 30
% If \param{direction} is \kwd{input}
% %% Barrett: defaults
% %, not supplied, 
% or \kwd{probe},
% or if \param{if-exists} is not \kwd{new-version}
% and the version component of the \param{filespec} is \kwd{newest},
% then the file opened is that file already existing in the file system
% that has a version greater than that of any other file in the file system
% whose other pathname components are the same as those of \param{filespec}.
% 
% 
% %% 23.2.0 31
% %% 23.2.0 32
% An implementation is required to recognize all of 
% the \funref{open} keyword options 
% and to do something reasonable in the context of the host operating
% system.  
% For example, if a file system does not support distinct file
% versions and does not distinguish the notions of deletion and expunging,
% \kwd{new-version} might be treated the same as
% \kwd{rename} or \kwd{supersede}, and \kwd{rename-and-delete} might
% be treated the same as \kwd{supersede}.
% 
% \beginlist
% \itemitem{\bf :direction}
% 
% 
% \Thenextfigure\ lists the possible values for \param{direction} options and their meanings.
% %% 23.2.0 4
% %% 23.2.0 5
% %% 23.2.0 6
% %% 23.2.0 7
% %% 23.2.0 8
% \boxfig
% {\dimen0=.75pc
% \tabskip \dimen0 plus .5 fil
% \halign to \hsize {#\hfil\tabskip \dimen0 plus 1fil&#\hfil\cr 
% \noalign{\vskip -9pt}
% \hfil{\bf Direction} & {\bf Result with respect to the created \term{stream}} \cr
% \noalign{\vskip 2pt\hrule\vskip 2pt}
% \kwd{input}
% % or not supplied
% 			    & Creates an \term{input} \term{file stream}. \cr
% 			    & \cr
% \kwd{output} 	            & Creates an \term{output} \term{file stream}. \cr
% 			    & \cr
% \kwd{io} 		    & Creates a \term{bidirectional} \term{file stream}. \cr
% 			    & \cr
% \kwd{probe} 	            & Creates a no-directional \term{file stream};
% 			      the \term{file stream}  \cr
%              		    & is in effect created and then closed. \cr
% \noalign{\vskip -9pt}
% }}
% \caption{Options for open---1}
% \endfig
% 
% 
% \itemitem{\bf :element-type}
% 
% \param{element-type} specifies the unit of transaction for the \term{file stream}.
% %% This is the only salvaged info from the next block of commented out stuff. -kmp 14-Feb-92
% If it is \kwd{default}, the unit is determined by \term{file system},
% possibly based on the \term{file}.
% % 
% % \Thenextfigure\ gives the correspondence between the listed values for
% % \param{element-type} and the unit of transaction.
% % 
% % \boxfig
% % {\dimen0=.75pc
% % \tabskip \dimen0 plus .5 fil
% % \halign to \hsize {#\hfil\tabskip \dimen0 plus 1fil&#\hfil\cr 
% % \noalign{\vskip -9pt}
% % \hfil{\bf Element-type} & {\bf Unit of Transaction} \cr
% % \noalign{\vskip 2pt\hrule\vskip 2pt}
% % %{\tt string-char} or not supplied & a \term{string-char} \cr
% % %& \cr
% % {\tt character} & Any \term{character} \cr
% % & \cr
% % {\tt (unsigned-byte \i{n})} & A non-negative \term{integer} of size
% % \i{n} \cr
% % & \cr
% % {\tt unsigned-byte} & A non-negative \term{integer} (size determined
% % by file system) \cr
% % & \cr
% % {\tt (signed-byte \i{n})} & Signed byte of size \i{n} \cr
% % & \cr
% % {\tt signed-byte} & Signed byte (size determined by file system) \cr
% % & \cr
% % {\tt bit} & Values \f{0} and \f{1} \cr
% % & \cr
% % {\tt (mod \i{n})} & A non-negative \term{integer} less than \i{n}
% % \cr
% % & \cr
% % \kwd{default} & Determined by file system based on the file \cr
% % \noalign{\vskip -9pt}
% % }}
% % \caption{Options for open---2}
% % \endfig
% 
% \itemitem{\bf :if-exists}
% 
% \param{if-exists} specifies the action to be taken if \param{direction} is
% \kwd{output} or \kwd{io} and a file of the name \param{filespec} 
% already exists.
% If \param{direction} is \kwd{input}, not supplied, or \kwd{probe}, 
% \param{if-exists} is ignored.
% \Thenextfigure\ gives the result of \funref{open} as modified 
% by \param{if-exists}.
% %% 23.2.0 29
% %% 23.2.0 20
% 
% \boxfig
% {\dimen0=.75pc
% \tabskip \dimen0 plus .5 fil
% \halign to \hsize {#\hfil\tabskip \dimen0 plus 1fil&#\hfil\cr 
% \noalign{\vskip -9pt}
% \hfil{\bf If-exists} & {\bf Result with respect to file named \param{filespec}} \cr
% \noalign{\vskip 2pt\hrule\vskip 2pt}
% \kwd{error} or not supplied & An error \oftype{file-error} is signaled. \cr
% when the version component of & \cr
% \param{filespec} is not \kwd{newest} & \cr
%  & \cr
% \kwd{new-version} or not supplied & A new file is created \cr
% when the version component of &	with a larger version number. \cr
% \param{filespec} is \kwd{newest} & \cr
%  & \cr
% \kwd{rename} & The existing file is renamed \cr
% 	& to some other name and then a new file is created. \cr
% & \cr
% \kwd{rename-and-delete} & The existing file is renamed \cr
% 	& to some other name, then it is deleted \cr
% 	& but not expunged, and then a new file is created. \cr
% & \cr
% \kwd{overwrite} & Output operations on the \term{stream} \cr
% 	& destructively modify the existing file. \cr
% 	& If \param{direction} is \kwd{io} the file is  \cr
%       	& opened in a bidirectional mode that allows both \cr
% 	& reading and writing.  The file pointer is initially \cr
%    	& positioned at the beginning of the file; however, \cr
%         & the file is not truncated back to length zero \cr
%    	& when it is opened. \cr
% & \cr
% \kwd{append} & Output operations on the \term{stream} \cr
% 	& destructively modify the existing file.  \cr
% 	& The file pointer is initially positioned at \cr
%         & the end of the file. \cr
% 	& If \param{direction} is \kwd{io}, \cr
%  	& the file is opened in a bidirectional mode that \cr
% 	& allows both reading and writing. \cr
% & \cr
% \kwd{supersede} & The existing file is superseded,   \cr
% 	& \ie a new file with the same name as the old \cr
% 	& one is created. If possible, the implementation \cr
% 	& should not destroy the old file until the new \cr
%         &  \term{stream} is closed. \cr
% & \cr
% \nil\ & No file or \term{stream} is created. \cr
% 	& \nil\ is returned to indicate failure. \cr
% \noalign{\vskip -9pt}
% }}
% \caption{Options for open---3}
% \endfig
%                                      
% \itemitem{\bf :if-does-not-exist}
% 
% %% 23.2.0 34
% \param{if-does-not-exist}
% specifies the action to be taken if
% a file of name \param{filespec} does not already exist.
% \Thenextfigure\  gives the result of \funref{open} as modified by 
% \param{if-does-not-exist}.
% 
% 
% %% 23.2.0 37
% \boxfig                             
% {\dimen0=.75pc
% \tabskip \dimen0 plus .5 fil
% \halign to \hsize {#\hfil\tabskip \dimen0 plus 1fil&#\hfil\cr 
% \noalign{\vskip -9pt}
% \hfil{\bf If-does-not-exist} & {\bf Result with respect to file named \param{filespec}} \cr
% \noalign{\vskip 2pt\hrule\vskip 2pt}
% \kwd{error} or not supplied & An error \oftype{file-error} is signaled. \cr
% and \param{direction} is \kwd{input}, & \cr
% or if \param{if-exists} is  & \cr
% \kwd{overwrite} or \kwd{append}. & \cr
%  & \cr
% \kwd{create} or not supplied & An empty file is created. \cr
% when \param{direction} is \kwd{output}& Processing continues  as if the file \cr
% or \kwd{io}, and \param{if-exists} & had already existed but no processing as \cr
% is anything but \kwd{overwrite} & directed by \param{if-exists} is performed. \cr
% or \kwd{append}. & \cr
%  & \cr
% \nil\ or not supplied & No file or \term{stream} is created. \cr
% when \param{direction} is \kwd{probe}.& \nil\ is returned to indicate failure. \cr
% \noalign{\vskip -9pt}
% }}
% \caption{Options for open---4}
% \endfig
% 
% \issue{CHARACTER-PROPOSAL:2-5-2}
% \itemitem{\kwd{external-format}}
% 
% This option selects an \term{external file format} for the \term{file}:
% % There are no conforming
% % \term{values} for \param{external-format} other than the default \term{value},
% % \kwd{default}, which selects a format that can accomodate at least the
% % \term{base characters}.  \term{Implementations} are permitted to define
% % additional external file formats.
% The only \term{standardized} value for this option is \kwd{default},
% although \term{implementations} are permitted to define additional 
% \term{external file formats} and \term{implementation-dependent} values 
% returned by \funref{stream-external-format} can also be used by \term{conforming programs}.
%  
% % The \param{external-format} is meaningful for 
% %      \term{input},
% %      \term{output},
% %  and \term{bidirectional} \term{streams}.
% %% Rewritten by KMP based on discussions with RWK:
% The \param{external-format} is meaningful for
% any kind of \term{file stream} whose \term{element type}
% is a \term{subtype} of \term{character}.
% This option is ignored for \term{streams} for which it is not meaningful;
% however, \term{implementations} may define other \term{element types} 
% for which it is meaningful.
% %End of rewrite.
% The consequences are unspecified if a \term{character} is written 
% that cannot be represented by the given \term{external file format}.
% \endissue{CHARACTER-PROPOSAL:2-5-2}
% 
% \endlist
% 
% %% 23.2.0 1 
% When a file is opened, a \term{file stream} is constructed to serve
% as the file system's ambassador to the \Lisp\ environment;
% operations on the \term{file stream} are reflected by operations on the file
% in the file system.  
% 
% A file can be deleted, renamed, or destructively modified by \funref{open}.
% 
% For information about opening relative pathnames,
% \seesection\MergingPathnames.
% 
% \label Examples::
% 
% \code
%  (open \i{filespec} :direction :probe)  \EV #<Closed Probe File Stream...>
%  (setq q (merge-pathnames (user-homedir-pathname) "test"))
% \EV #<PATHNAME :HOST NIL :DEVICE \i{device-name} :DIRECTORY \i{directory-name}
%     :NAME "test" :TYPE NIL :VERSION :NEWEST>
%  (open \i{filespec} :if-does-not-exist :create) \EV #<Input File Stream...>
%  (setq s (open \i{filespec} :direction :probe)) \EV #<Closed Probe File Stream...>
%  (truename s) \EV #<PATHNAME :HOST NIL :DEVICE \i{device-name} :DIRECTORY
%     \i{directory-name} :NAME \i{filespec} :TYPE \i{extension} :VERSION 1>
%  (open s :direction :output :if-exists nil) \EV NIL 
% \endcode
% 
% \label Affected By::
% 
% The nature and state of the host computer's \term{file system}.
% 
% \label Exceptional Situations::                        
% 
% If \param{if-exists} is \kwd{error}, (subject to the
% constraints on the meaning of \param{if-exists} listed above),
% an error \oftype{file-error} is signaled.                        
% 
% If \param{if-does-not-exist} is \kwd{error} (subject to the
% constraints on the meaning of \param{if-does-not-exist} listed above),
% an error \oftype{file-error} is signaled.
% 
% %% 23.2.0 33
% If it is impossible for an implementation to handle some option
% in a manner close to what is specified here,
% an error \oftype{error} might be signaled.
% 
% \issue{PATHNAME-WILD:NEW-FUNCTIONS}
% % !!! Moon: I would have assumed type-error rather than file-error for this one.
% %           I have no strong opinion, though.
% An error \oftype{file-error} is signaled if
% {\tt (wild-pathname-p \param{filespec})} returns true.
% \endissue{PATHNAME-WILD:NEW-FUNCTIONS}
% 
% \issue{CHARACTER-PROPOSAL:2-5-2}
% An error \oftype{error} is signaled if the \param{external-format} 
% is not understood by the \term{implementation}.
% \endissue{CHARACTER-PROPOSAL:2-5-2}
% 
% % Refer to minutes of Dec-91 meeting (issue STREAM-ELEMENT-TYPE-UPGRADING) for
% % an explanation of where this came from. -kmp 12-Feb-92
% The various \term{file systems} in existence today have widely differing capabilities,
% and some aspects of the \term{file system} are beyond the scope of this specification
% to define.  A given \term{implementation} might not be able to support all of these options 
% in exactly the manner stated.  An \term{implementation} is required to recognize all of 
% these option keywords and to try to do something ``reasonable'' in the context of the 
% host \term{file system}.  Where necessary to accomodate the \term{file system},
% an \term{implementation} deviate slightly from the semantics specified here without 
% being disqualified for consideration as a \term{conforming implementation}.
% If it is utterly impossible for an \term{implementation} to handle some option
% in a manner similar to what is specified here, it may simply signal an error.
% 
% With regard to the \kwd{element-type} option, if a \term{type} is
% requested that is not supported by the \term{file system}, a substitution of types 
% such as that which goes on in \term{upgrading} is permissible.  As a minimum 
% requirement, it should be the case that opening an \term{output} \term{stream}
% to a \term{file} in a given \term{element type} and later opening 
% an \term{input} \term{stream} to the same \term{file} in the same \term{element type}
% should work compatibly.
% 
% \label See Also::
% 
% \issue{PATHNAME-LOGICAL:ADD}
% \macref{with-open-file},
% \funref{close},
% \typeref{pathname},
% \typeref{logical-pathname},
% \endissue{PATHNAME-LOGICAL:ADD}
% {\secref\MergingPathnames}
% 
% \label Notes::
% 
% \issue{PATHNAME-HOST-PARSING:RECOGNIZE-LOGICAL-HOST-NAMES}
% % \issue{PATHNAME-LOGICAL:ADD}
% % Whether or not \funref{open} recognizes \term{logical pathname} \term{namestrings}
% % is \term{implementation-defined}.
% % \endissue{PATHNAME-LOGICAL:ADD}
% \endissue{PATHNAME-HOST-PARSING:RECOGNIZE-LOGICAL-HOST-NAMES}
%  
% %% 23.2.0 38
% \funref{open} does not automatically close the file when an abnormal
% exit occurs.
% 
% When \param{element-type} is a \term{subtype} of \typeref{character},
% \funref{read-char} and/or \funref{write-char} can be
% used on the resulting \term{file stream}. 
% 
% When \param{element-type} is a \term{subtype} of \term{integer},
% \funref{read-byte} and/or \funref{write-byte} can be used on the resulting \term{file stream}.
% 
% % When \param{element-type} is 
% %     \typeref{unsigned-byte},
% %     {\tt (unsigned-byte \i{n})},
% %     \typeref{signed-byte},
% %     {\tt (signed-byte \i{n})},
% %     \typeref{bit}, 
% %  or {\tt (mod \i{n})},
% % \funref{read-byte} and/or \funref{write-byte} can be
% % used on the resulting \term{file stream}.
% 
% When \param{element-type} is \kwd{default},
% the \term{type} can be determined by using \funref{stream-element-type}.
% 
% \endcom

%%% ========== OPEN
\begincom{open}\ftype{Function}

\label Syntax::

\DefunWithValuesNewline open
		 	{filespec {\key} \vtop{\hbox{direction element-type}
				               \hbox{if-exists if-does-not-exist
						     external-format}}}
		        {stream}

\label Arguments and Values::

\param{filespec}---a \term{pathname designator}.

\param{direction}---one of \kwd{input}, \kwd{output}, \kwd{io}, or \kwd{probe}.
 \Default{\kwd{input}}

%% 23.2.0 9
%% 23.2.0 10
%% 23.2.0 11
%% 23.2.0 12
%% 23.2.0 13
%% 23.2.0 14
%% 23.2.0 15
%% 23.2.0 16
%% 23.2.0 17
%% 23.2.0 18
%% 23.2.0 19
\param{element-type}---a \term{type specifier} 
		       for \term{recognizable subtype} of \typeref{character};
		    or a \term{type specifier} 
		       for a \term{finite} \term{recognizable subtype} of \term{integer};
		    or one of the \term{symbols} 
                          \misc{signed-byte},
                          \misc{unsigned-byte},
		       or \kwd{default}.
 \Default{\typeref{character}}

%% 23.2.0 21
%% 23.2.0 22
%% 23.2.0 23
%% 23.2.0 24
%% 23.2.0 25
%% 23.2.0 26
%% 23.2.0 27
%% 23.2.0 28
\param{if-exists}---one of \kwd{error}, \kwd{new-version}, \kwd{rename},
		  \kwd{rename-and-delete}, \kwd{overwrite}, \kwd{append},
		  \kwd{supersede}, or \nil.
 \Default{\kwd{new-version} if the version component of \param{filespec} is \kwd{newest},
          or \kwd{error} otherwise}

%% 23.2.0 35
%% 23.2.0 36
\param{if-does-not-exist}---one of \kwd{error}, \kwd{create}, or \nil.
 \Default{\kwd{error} if \param{direction} is \kwd{input} 
		      or \param{if-exists} is \kwd{overwrite} or \kwd{append};
	  \kwd{create} if \param{direction} is \kwd{output} or \kwd{io},
		      and \param{if-exists} is neither \kwd{overwrite} nor \kwd{append};
       or \nil\ when \param{direction} is \kwd{probe}}

\issue{CHARACTER-PROPOSAL:2-5-2}
\param{external-format}---an \term{external file format designator}.
  \Default{\kwd{default}}
\endissue{CHARACTER-PROPOSAL:2-5-2}

\issue{STREAM-ACCESS:ADD-TYPES-ACCESSORS}
\param{stream}---a \term{file stream} or \nil.
\endissue{STREAM-ACCESS:ADD-TYPES-ACCESSORS}

\label Description::

%% 23.2.0 2
%% 23.2.0 3
\funref{open} creates, opens, and returns a \term{file stream} 
that is connected to the file specified by \param{filespec}.
\param{Filespec} is the name of the file to be opened.  
%% Moon: Mostly redundant.
% \issue{CLOSED-STREAM-FUNCTIONS:ALLOW-INQUIRY}
% If \param{filespec} is a \term{stream},
% \param{filespec} effectively
% becomes {\tt (pathname \param{filespec})}. \funref{open}
% can be used on either an open or a closed \term{stream},
% \endissue{CLOSED-STREAM-FUNCTIONS:ALLOW-INQUIRY}
% and the \term{stream} is not
% closed first or otherwise affected.
% \issue{PATHNAME-STREAM}
% Also, if \param{filespec} is a \term{stream}, 
% that \term{stream}
% can only have been originally opened by
% \funref{open} or \macref{with-open-file}, 
% or is a \term{synonym stream} whose \term{symbol} is bound
% to a \term{stream} originally opened by
% \funref{open} or \macref{with-open-file}.
% If \param{filespec} is a \term{pathname} (as returned by \funref{pathname})
% it represents the name used to open the file. This might be, but is
% not required to be, the actual name of the file. 
% \issue{PATHNAME-LOGICAL:ADD}
% If \param{filespec} is a \term{logical pathname}, it is translated
% into a physical pathname as if by calling \funref{translate-logical-pathname}.
% \endissue{PATHNAME-LOGICAL:ADD}
%% Salvaged from the above. -kmp 16-Feb-92
If the \param{filespec} \term{designator} is a \term{stream},
that \term{stream} is not closed first or otherwise affected.

The keyword arguments to \funref{open} specify the characteristics 
of the \term{file stream} that is returned, and how to handle errors.

%% 23.2.0 30
If \param{direction} is \kwd{input}
%% Barrett: defaults
%, not supplied, 
or \kwd{probe},
or if \param{if-exists} is not \kwd{new-version}
and the version component of the \param{filespec} is \kwd{newest},
then the file opened is that file already existing in the file system
that has a version greater than that of any other file in the file system
whose other pathname components are the same as those of \param{filespec}.

%% 23.2.0 31
%% 23.2.0 32
An implementation is required to recognize all of 
the \funref{open} keyword options 
and to do something reasonable in the context of the host operating
system.  
For example, if a file system does not support distinct file
versions and does not distinguish the notions of deletion and expunging,
\kwd{new-version} might be treated the same as
\kwd{rename} or \kwd{supersede}, and \kwd{rename-and-delete} might
be treated the same as \kwd{supersede}.

\beginlist

\itemitem{\kwd{direction}}

These are the possible values for \param{direction}, 
and how they affect the nature of the \term{stream} that is created:

\beginlist
%% 23.2.0 4
%% 23.2.0 5
%% 23.2.0 6
%% 23.2.0 7
%% 23.2.0 8
\itemitem{\kwd{input}}

Causes the creation of an \term{input} \term{file stream}.

\itemitem{\kwd{output}} 	            

Causes the creation of an \term{output} \term{file stream}.

\itemitem{\kwd{io}}

Causes the creation of a \term{bidirectional} \term{file stream}.

\itemitem{\kwd{probe}}

Causes the creation of a ``no-directional'' \term{file stream}; 
in effect, the \term{file stream} is created 
and then closed prior to being returned by \funref{open}.

\endlist

\itemitem{\kwd{element-type}}

The \param{element-type} specifies the unit of transaction for the \term{file stream}.
%% This is the only salvaged info from the next block of commented out stuff. -kmp 14-Feb-92
If it is \kwd{default},
the unit is determined by \term{file system},
possibly based on the \term{file}.

\itemitem{\kwd{if-exists}}

\param{if-exists} specifies the action to be taken if \param{direction} is
\kwd{output} or \kwd{io} and a file of the name \param{filespec} 
already exists.
If \param{direction} is \kwd{input}, not supplied, or \kwd{probe}, 
\param{if-exists} is ignored.
These are the results of \funref{open} as modified by \param{if-exists}:
%% 23.2.0 29
%% 23.2.0 20

\beginlist 

\itemitem{\kwd{error}}

An error \oftype{file-error} is signaled.

\itemitem{\kwd{new-version}}

A new file is created with a larger version number.

\itemitem{\kwd{rename}} 

The existing file is renamed to some other name and then a new file is created.

\itemitem{\kwd{rename-and-delete}}

The existing file is renamed to some other name,
then it is deleted but not expunged, and then a new file is created.

\itemitem{\kwd{overwrite}}

Output operations on the \term{stream} destructively modify the existing file.
If \param{direction} is \kwd{io} the file is opened in a bidirectional mode
that allows both reading and writing.  The file pointer is initially 
positioned at the beginning of the file; however, the file is not truncated
back to length zero when it is opened.

\itemitem{\kwd{append}}

Output operations on the \term{stream} destructively modify the existing file.
The file pointer is initially positioned at the end of the file.

If \param{direction} is \kwd{io},
the file is opened in a bidirectional mode that allows both reading and writing.

\itemitem{\kwd{supersede}}

The existing file is superseded;
that is, a new file with the same name as the old one is created.
If possible, the implementation should not destroy the old file until the new
\term{stream} is closed.

\itemitem{\nil}

No file or \term{stream} is created;
instead, \nil\ is returned to indicate failure.

\endlist
                                     
\itemitem{\kwd{if-does-not-exist}}

%% 23.2.0 34
\param{if-does-not-exist}
specifies the action to be taken if
a file of name \param{filespec} does not already exist.
These are the results of \funref{open} as modified by \param{if-does-not-exist}:

%% 23.2.0 37
\beginlist

\itemitem{\kwd{error}}

An error \oftype{file-error} is signaled.

\itemitem{\kwd{create}}

An empty file is created.
Processing continues  as if the file 
had already existed but no processing as 
directed by \param{if-exists} is performed.

\itemitem{\nil}

No file or \term{stream} is created;
instead, \nil\ is returned to indicate failure.

\endlist

\issue{CHARACTER-PROPOSAL:2-5-2}
\itemitem{\kwd{external-format}}

This option selects an \term{external file format} for the \term{file}:
% There are no conforming
% \term{values} for \param{external-format} other than the default \term{value},
% \kwd{default}, which selects a format that can accomodate at least the
% \term{base characters}.  \term{Implementations} are permitted to define
% additional external file formats.
The only \term{standardized} value for this option is \kwd{default},
although \term{implementations} are permitted to define additional 
\term{external file formats} and \term{implementation-dependent} values 
returned by \funref{stream-external-format} can also be used by \term{conforming programs}.
 
% The \param{external-format} is meaningful for 
%      \term{input},
%      \term{output},
%  and \term{bidirectional} \term{streams}.
%% Rewritten by KMP based on discussions with RWK:
The \param{external-format} is meaningful for
any kind of \term{file stream} whose \term{element type}
is a \term{subtype} of \term{character}.
This option is ignored for \term{streams} for which it is not meaningful;
however, \term{implementations} may define other \term{element types} 
for which it is meaningful.
%End of rewrite.
The consequences are unspecified if a \term{character} is written 
that cannot be represented by the given \term{external file format}.
\endissue{CHARACTER-PROPOSAL:2-5-2}

\endlist

%% 23.2.0 1 
When a file is opened, a \term{file stream} is constructed to serve
as the file system's ambassador to the \Lisp\ environment;
operations on the \term{file stream} are reflected by operations on the file
in the file system.  

A file can be deleted, renamed, or destructively modified by \funref{open}.

For information about opening relative pathnames,
\seesection\MergingPathnames.

\label Examples::

\code
 (open \i{filespec} :direction :probe)  \EV #<Closed Probe File Stream...>
 (setq q (merge-pathnames (user-homedir-pathname) "test"))
\EV #<PATHNAME :HOST NIL :DEVICE \i{device-name} :DIRECTORY \i{directory-name}
    :NAME "test" :TYPE NIL :VERSION :NEWEST>
 (open \i{filespec} :if-does-not-exist :create) \EV #<Input File Stream...>
 (setq s (open \i{filespec} :direction :probe)) \EV #<Closed Probe File Stream...>
 (truename s) \EV #<PATHNAME :HOST NIL :DEVICE \i{device-name} :DIRECTORY
    \i{directory-name} :NAME \i{filespec} :TYPE \i{extension} :VERSION 1>
 (open s :direction :output :if-exists nil) \EV NIL 
\endcode

\label Affected By::

The nature and state of the host computer's \term{file system}.

\label Exceptional Situations::                        

If \param{if-exists} is \kwd{error}, (subject to the
constraints on the meaning of \param{if-exists} listed above),
an error \oftype{file-error} is signaled.                        

If \param{if-does-not-exist} is \kwd{error} (subject to the
constraints on the meaning of \param{if-does-not-exist} listed above),
an error \oftype{file-error} is signaled.

%% 23.2.0 33
If it is impossible for an implementation to handle some option
in a manner close to what is specified here,
an error \oftype{error} might be signaled.

\issue{PATHNAME-WILD:NEW-FUNCTIONS}
% !!! Moon: I would have assumed type-error rather than file-error for this one.
%           I have no strong opinion, though.
An error \oftype{file-error} is signaled if
{\tt (wild-pathname-p \param{filespec})} returns true.
\endissue{PATHNAME-WILD:NEW-FUNCTIONS}

\issue{CHARACTER-PROPOSAL:2-5-2}
An error \oftype{error} is signaled if the \param{external-format} 
is not understood by the \term{implementation}.
\endissue{CHARACTER-PROPOSAL:2-5-2}

% Refer to minutes of Dec-91 meeting (issue STREAM-ELEMENT-TYPE-UPGRADING) for
% an explanation of where this came from. -kmp 12-Feb-92
The various \term{file systems} in existence today have widely differing capabilities,
and some aspects of the \term{file system} are beyond the scope of this specification
to define.  A given \term{implementation} might not be able to support all of these options 
in exactly the manner stated.  An \term{implementation} is required to recognize all of 
these option keywords and to try to do something ``reasonable'' in the context of the 
host \term{file system}.  Where necessary to accomodate the \term{file system},
an \term{implementation} deviate slightly from the semantics specified here without 
being disqualified for consideration as a \term{conforming implementation}.
If it is utterly impossible for an \term{implementation} to handle some option
in a manner similar to what is specified here, it may simply signal an error.

With regard to the \kwd{element-type} option, if a \term{type} is
requested that is not supported by the \term{file system}, a substitution of types 
such as that which goes on in \term{upgrading} is permissible.  As a minimum 
requirement, it should be the case that opening an \term{output} \term{stream}
to a \term{file} in a given \term{element type} and later opening 
an \term{input} \term{stream} to the same \term{file} in the same \term{element type}
should work compatibly.

\label See Also::

\issue{PATHNAME-LOGICAL:ADD}
\macref{with-open-file},
\funref{close},
\typeref{pathname},
\typeref{logical-pathname},
\endissue{PATHNAME-LOGICAL:ADD}
{\secref\MergingPathnames},
\issue{FILE-OPEN-ERROR:SIGNAL-FILE-ERROR}
{\secref\PathnamesAsFilenames}
\endissue{FILE-OPEN-ERROR:SIGNAL-FILE-ERROR}

\label Notes::

\issue{PATHNAME-HOST-PARSING:RECOGNIZE-LOGICAL-HOST-NAMES}
% \issue{PATHNAME-LOGICAL:ADD}
% Whether or not \funref{open} recognizes \term{logical pathname} \term{namestrings}
% is \term{implementation-defined}.
% \endissue{PATHNAME-LOGICAL:ADD}
\endissue{PATHNAME-HOST-PARSING:RECOGNIZE-LOGICAL-HOST-NAMES}
 
%% 23.2.0 38
\funref{open} does not automatically close the file when an abnormal
exit occurs.

When \param{element-type} is a \term{subtype} of \typeref{character},
\funref{read-char} and/or \funref{write-char} can be
used on the resulting \term{file stream}. 

When \param{element-type} is a \term{subtype} of \term{integer},
\funref{read-byte} and/or \funref{write-byte} can be used on the resulting \term{file stream}.

% When \param{element-type} is 
%     \typeref{unsigned-byte},
%     {\tt (unsigned-byte \i{n})},
%     \typeref{signed-byte},
%     {\tt (signed-byte \i{n})},
%     \typeref{bit}, 
%  or {\tt (mod \i{n})},
% \funref{read-byte} and/or \funref{write-byte} can be
% used on the resulting \term{file stream}.

When \param{element-type} is \kwd{default},
the \term{type} can be determined by using \funref{stream-element-type}.

\endcom


%%% ========== STREAM-EXTERNAL-FORMAT
\begincom{stream-external-format}\ftype{Function}

\label Syntax::

\DefunWithValues stream-external-format {stream} {format}

\label Arguments and Values::

\param{stream}---a \term{file stream}.

\param{format}---an \term{external file format}.

\label Description::

Returns an \term{external file format designator} for the \param{stream}.

\label Examples::

\code
 (with-open-file (stream "test" :direction :output)
   (stream-external-format stream))
\EV :DEFAULT
\OV :ISO8859/1-1987
\OV (:ASCII :SAIL)
\OV ACME::PROPRIETARY-FILE-FORMAT-17
\OV #<FILE-FORMAT :ISO646-1983 2343673>
\endcode

\label Side Effects:\None!

\label Affected By:\None!

\label Exceptional Situations:\None.

\label See Also::

the \kwd{external-format} \term{argument} to \thefunction{open} and
\themacro{with-open-file}.

\label Notes::

The \param{format} returned is not necessarily meaningful 
to other \term{implementations}.

\endcom

%%% ========== WITH-OPEN-FILE
\begincom{with-open-file}\ftype{Macro}

\issue{DECLS-AND-DOC}

\label Syntax::

\DefmacWithValuesNewline with-open-file
		  {\paren{stream filespec \starparam{options}}
		   \starparam{declaration}
		   \starparam{form}}
		  {results}

\label Arguments and Values::

\param{stream}---a variable.

\issue{STREAM-ACCESS:ADD-TYPES-ACCESSORS}
\param{filespec}---a \term{pathname designator}.
\endissue{STREAM-ACCESS:ADD-TYPES-ACCESSORS}

%!!! I doubt this can be right. -kmp 22-Apr-91
\param{options}---\term{forms}; \eval.

\param{declaration}---a \misc{declare} \term{expression}; \noeval.

\param{forms}---an \term{implicit progn}.
                                   
\param{results}---the \term{values} returned by the \param{forms}.

\label Description::

\issue{STREAM-ACCESS:ADD-TYPES-ACCESSORS}
\macref{with-open-file} uses \funref{open} to create a \term{file stream}
\endissue{STREAM-ACCESS:ADD-TYPES-ACCESSORS}
to \term{file} named by \param{filespec}. 
%% 23.2.0 41
\param{Filespec} is the name of the file to be opened.
\param{Options} are used as keyword arguments to \funref{open}.
%% Moon thought this was "drivel". -kmp 11-Feb-92
% Then \macref{with-open-file} performs a 
% specified series of actions on the open file.
%% This is implied by "pathname designator". -kmp 16-Feb-92
% \issue{PATHNAME-STREAM}
% If \param{filespec} is a \term{file stream}, that \term{file stream}
% can only have been originally opened by
% \funref{open} or \macref{with-open-file}, 
% or is a \term{synonym stream} whose \term{symbol} is bound
% to a \term{file stream} originally opened by
% \funref{open} or \macref{with-open-file}.
% If \param{filespec} is a \term{pathname} (as returned by \funref{pathname}
% it represents the name used to open the file. This may be, but is
% not required to be, the actual name of the file. 
% \endissue{PATHNAME-STREAM}
% \issue{PATHNAME-LOGICAL:ADD}
% If \param{filespec} is a \term{logical pathname}, it is translated
% into a physical pathname as if by calling \funref{translate-logical-pathname}.
% \endissue{PATHNAME-LOGICAL:ADD}

\issue{WITH-OPEN-FILE-STREAM-EXTENT:DYNAMIC-EXTENT}
The \term{stream} \term{object} to which the \param{stream} \term{variable} 
is \term{bound} has \term{dynamic extent};
its \term{extent} ends when the \term{form} is exited.
\endissue{WITH-OPEN-FILE-STREAM-EXTENT:DYNAMIC-EXTENT}

%% 23.2.0 39
\macref{with-open-file} evaluates the \param{forms} as an \term{implicit progn} 
with \param{stream} bound to 
\issue{WITH-OPEN-FILE-DOES-NOT-EXIST:STREAM-IS-NIL}
the value returned by \funref{open}.
\endissue{WITH-OPEN-FILE-DOES-NOT-EXIST:STREAM-IS-NIL}

%% 23.2.0 40
When control leaves the body, either normally or abnormally (such as by
use of \specref{throw}), the file is automatically closed.  If a new
output file is being written, and control leaves abnormally, the file is
aborted and the file system is left, so far as possible, as if the file
had never been opened.  

It is possible by the use of \f{:if-exists nil} 
or \f{:if-does-not-exist nil} for
\param{stream} to be bound to \nil.
\issue{WITH-OPEN-FILE-DOES-NOT-EXIST:STREAM-IS-NIL}
Users of \f{:if-does-not-exist nil} should check for a valid \term{stream}.
\endissue{WITH-OPEN-FILE-DOES-NOT-EXIST:STREAM-IS-NIL}


%% 23.2.0 42
% %This looks suspect to me. -kmp 22-Apr-91
% %Moon: Yes, this is drivel.
%
% \macref{with-open-file} opens, modifies, and 
% tries to close the \param{file stream}.

\issue{WITH-OPEN-FILE-SETQ:EXPLICITLY-VAGUE}
The consequences are undefined if an attempt is made to \term{assign} the 
\param{stream} \term{variable}.  The compiler may choose to issue a
warning if such an attempt is detected.
\endissue{WITH-OPEN-FILE-SETQ:EXPLICITLY-VAGUE}

\label Examples::

\code
 (setq p (merge-pathnames "test"))
\EV #<PATHNAME :HOST NIL :DEVICE \i{device-name} :DIRECTORY \i{directory-name}
    :NAME "test" :TYPE NIL :VERSION :NEWEST>
 (with-open-file (s p :direction :output :if-exists :supersede)
    (format s "Here are a couple~%of test data lines~%")) \EV NIL
 (with-open-file (s p)
    (do ((l (read-line s) (read-line s nil 'eof)))
        ((eq l 'eof) "Reached end of file.")
     (format t "~&*** ~A~%" l)))
\OUT *** Here are a couple
\OUT *** of test data lines
\EV "Reached end of file."
\endcode

%\code
% (with-open-file (ifile name :direction :input)
%   (with-open-file (ofile (merge-pathname-defaults ifile
%                                                   nil
%                                                   "out")
%                          :direction :output
%                          :if-exists :supersede)
%     (transduce-file ifile ofile)))
%\endcode

\issue{WITH-OPEN-FILE-DOES-NOT-EXIST:STREAM-IS-NIL}
%!!! This example is -way- too weird to be included without more commentary.
\code
;; Normally one would not do this intentionally because it is
;; not perspicuous, but beware when using :IF-DOES-NOT-EXIST NIL
;; that this doesn't happen to you accidentally...
 (with-open-file (foo "no-such-file" :if-does-not-exist nil)
   (read foo))
\OUT \IN{hello?}
\EV HELLO? ;This value was read from the terminal, not a file!

;; Here's another bug to avoid...
 (with-open-file (foo "no-such-file" :direction :output :if-does-not-exist nil)
   (format foo "Hello"))
\EV "Hello" ;FORMAT got an argument of NIL!
\endcode
\endissue{WITH-OPEN-FILE-DOES-NOT-EXIST:STREAM-IS-NIL}

\label Side Effects::

Creates a \term{stream} to the \term{file} named by \param{filename} (upon entry),
and closes the \term{stream} (upon exit).
In some \term{implementations},
the \term{file} might be locked in some way while it is open.
If the \term{stream} is an \term{output} \term{stream},
a \term{file} might be created.

\label Affected By::

The host computer's file system.

\label Exceptional Situations::

\issue{FILE-OPEN-ERROR:SIGNAL-FILE-ERROR}
%% This restriction (and more) is already manifest in OPEN, so just omit it here.
%
% \issue{PATHNAME-WILD:NEW-FUNCTIONS}
% % !!! Moon: I would have assumed type-error rather than file-error for this one.
% %           I have no strong opinion, though.
% An error \oftype{file-error} is signaled if
% \f{(wild-pathname-p \param{filespec})} returns true.
% \endissue{PATHNAME-WILD:NEW-FUNCTIONS}
\Seefun{open}.
\endissue{FILE-OPEN-ERROR:SIGNAL-FILE-ERROR}

\label See Also::

\issue{PATHNAME-LOGICAL:ADD}
\funref{open},
\funref{close},
\typeref{pathname},
\typeref{logical-pathname},
\endissue{PATHNAME-LOGICAL:ADD}
\issue{FILE-OPEN-ERROR:SIGNAL-FILE-ERROR}
{\secref\PathnamesAsFilenames}
\endissue{FILE-OPEN-ERROR:SIGNAL-FILE-ERROR}

\label Notes:\None.

\issue{PATHNAME-HOST-PARSING:RECOGNIZE-LOGICAL-HOST-NAMES}
% \issue{PATHNAME-LOGICAL:ADD}
% Whether or not \macref{with-open-file} recognizes \term{logical pathname} \term{namestrings}
% is \term{implementation-defined}.
% \endissue{PATHNAME-LOGICAL:ADD}
\endissue{PATHNAME-HOST-PARSING:RECOGNIZE-LOGICAL-HOST-NAMES}

\endissue{DECLS-AND-DOC}

\endcom

%-------------------- Stream Closing --------------------

%%% ========== CLOSE
\begincom{close}\ftype{Function}

\label Syntax::

\issue{RETURN-VALUES-UNSPECIFIED:SPECIFY}
\DefunWithValues close {stream {\key} abort} {result}
\endissue{RETURN-VALUES-UNSPECIFIED:SPECIFY}

\label Arguments and Values::

%% 21.3.0 7
\param{stream}---a \term{stream} (either \term{open} or \term{closed}).

\param{abort}---a \term{generalized boolean}.
 \Default{\term{false}}

\issue{CLOSED-STREAM-FUNCTIONS:ALLOW-INQUIRY}
\param{result}---\t\ if the \param{stream} was \term{open} at the time it was
		     received as an \term{argument},
		 or \term{implementation-dependent} otherwise.
\endissue{CLOSED-STREAM-FUNCTIONS:ALLOW-INQUIRY}


\label Description::

%% 21.3.0 7
%% 23.2.0 1
\funref{close} closes \param{stream}. 
Closing a \term{stream} means
that it may no longer be used in input or output operations.
The act of \term{closing} a \term{file stream}
ends the association between the \term{stream} and its associated \term{file}; 
the transaction with the \term{file system} is terminated,
and input/output may no longer be performed on the \term{stream}. 

If \param{abort} is \term{true}, an attempt is made to clean up any side
effects of having created \param{stream}.
If \param{stream} performs output to a file
that was created when the \term{stream} was created, the
file is deleted and any previously existing file is not superseded.

It is permissible to close an already closed \term{stream},
but in that case the \param{result} is \term{implementation-dependent}.
 
After \param{stream} is closed, it is still possible to perform 
the following query operations upon it:
\issue{CLOSED-STREAM-FUNCTIONS:ALLOW-INQUIRY}
\funref{streamp}, \funref{pathname}, \funref{truename}, 
\funref{merge-pathnames}, \funref{pathname-host}, \funref{pathname-device},
\funref{pathname-directory},\funref{pathname-name}, 
\funref{pathname-type}, \funref{pathname-version}, \funref{namestring}, 
\funref{file-namestring}, \funref{directory-namestring}, 
\funref{host-namestring}, \funref{enough-namestring}, \funref{open},
\funref{probe-file}, and \funref{directory}. 
\endissue{CLOSED-STREAM-FUNCTIONS:ALLOW-INQUIRY}

\issue{CLOSE-CONSTRUCTED-STREAM:ARGUMENT-STREAM-ONLY}
The effect of \funref{close} on a \term{constructed stream} is 
 to close the argument \param{stream} only. 
There is no effect on the \term{constituents} of \term{composite streams}.
 
For a \term{stream} created with \funref{make-string-output-stream},
the result of \funref{get-output-stream-string} is unspecified after \funref{close}.
%% Sandra: Redundant.
% For a \term{composite stream},
% the call to \funref{close} has no effect on any of its \term{constituents}.
\endissue{CLOSE-CONSTRUCTED-STREAM:ARGUMENT-STREAM-ONLY}
 
\label Examples::

%!!! Need to look at this!  -kmp 23-Apr-91
\code
 (setq s (make-broadcast-stream)) \EV #<BROADCAST-STREAM>
 (close s) \EV T
 (output-stream-p s) \EV \term{true}
\endcode

\label Side Effects::

The \param{stream} is \term{closed} (if necessary).
If \param{abort} is \term{true} and the \param{stream} is
an \term{output} \term{file stream}, its associated \term{file}
might be deleted.

\label Affected By:\None!

\label Exceptional Situations:\None.

\label See Also::

\funref{open}

\label Notes:\None.

\endcom

%%% ========== WITH-OPEN-STREAM
\begincom{with-open-stream}\ftype{Macro}

\issue{DECLS-AND-DOC}

\label Syntax::

\DefmacWithValuesNewline with-open-stream 
		  {\paren{var stream}
		   \starparam{declaration} 
		   \starparam{form}}
		  {\starparam{result}}

\label Arguments and Values::

\param{var}---a \term{variable} \term{name}.

\param{stream}---a \term{form}; evaluated to produce a \term{stream}.

\param{declaration}---a \misc{declare} \term{expression}; \noeval.

\param{forms}---an \term{implicit progn}.

\param{results}---the \term{values} returned by the \param{forms}.

\label Description::

\macref{with-open-stream} performs a series of operations on 
\param{stream}, returns a value, and then closes the \param{stream}.

%% 21.2.0 11
\param{Var} is bound to the value of \param{stream},
and then \param{forms} are executed
as an \term{implicit progn}. 
\param{stream}
is automatically closed on exit from \macref{with-open-stream},
no matter whether the exit is normal or abnormal.
\issue{WITH-OPEN-FILE-STREAM-EXTENT:DYNAMIC-EXTENT}
%!!!
%Moon: I don't see how the following could possibly be true, 
%      since WITH-OPEN-STREAM does not create the stream.  
%      WITH-OPEN-FILE is different.  On the other hand,
%      this accurately reflects the cleanup and CLtL.
%KMP: I concur, but it will take a technical vote to fix it.
The \param{stream} has \term{dynamic extent};
its \term{extent} ends when the \term{form} is exited.
\endissue{WITH-OPEN-FILE-STREAM-EXTENT:DYNAMIC-EXTENT}

\issue{WITH-OPEN-FILE-SETQ:EXPLICITLY-VAGUE}
The consequences are undefined if an attempt is made to \term{assign} the 
the \term{variable} \param{var} with the \param{forms}.
%The compiler may choose to issue a warning if such an attempt is detected.
\endissue{WITH-OPEN-FILE-SETQ:EXPLICITLY-VAGUE}

\label Examples::

\code
 (with-open-stream (s (make-string-input-stream "1 2 3 4"))
    (+ (read s) (read s) (read s))) \EV 6
\endcode

\label Side Effects::

The \param{stream} is closed (upon exit).

\label Affected By:\None.

\label Exceptional Situations:\None.

\label See Also::

\funref{close}

\label Notes:\None.

\endissue{DECLS-AND-DOC}

\endcom

%-------------------- Stream Buffering --------------------

%%% ========== LISTEN
\begincom{listen}\ftype{Function}

\label Syntax::

\DefunWithValues {listen} {{\opt} input-stream} {generalized-boolean}

\label Arguments and Values::

\param{input-stream}---an \term{input} \term{stream designator}.
 \Default{\term{standard input}}

\param{generalized-boolean}---a \term{generalized boolean}.

\label Description::

%% 22.2.1 36
Returns \term{true} if 
there is a character immediately available from \param{input-stream};
otherwise, returns \term{false}.
On a non-interactive \param{input-stream},
\funref{listen} returns \term{true} except when at \term{end of file}\meaning{1}.
If an \term{end of file} is encountered, \funref{listen} returns \term{false}.
\funref{listen} is intended to be used 
when \param{input-stream} obtains characters
from an interactive device such as a keyboard.  

\label Examples::

\code
 (progn (unread-char (read-char)) (list (listen) (read-char)))
\OUT \IN{1}
\EV (T #\\1)
 (progn (clear-input) (listen))
\EV NIL ;Unless you're a very fast typist!
\endcode

\label Side Effects:\None.

\label Affected By::

\varref{*standard-input*}

\label Exceptional Situations:\None.

\label See Also::

\funref{interactive-stream-p},
\funref{read-char-no-hang}

\label Notes:\None.

\endcom

%%% ========== CLEAR-INPUT
\begincom{clear-input}\ftype{Function}

\label Syntax::

\DefunWithValues {clear-input} {{\opt} input-stream} {\nil}

\label Arguments and Values::

\param{input-stream}---an \term{input} \term{stream designator}.
  \Default{\term{standard input}}

\label Description::

%% 22.2.1 38
Clears any available input from \param{input-stream}.

If \funref{clear-input} does not make sense for \param{input-stream}, 
then \funref{clear-input} does nothing. 

\label Examples::
\code
;; The exact I/O behavior of this example might vary from implementation
;; to implementation depending on the kind of interactive buffering that
;; occurs.  (The call to SLEEP here is intended to help even out the 
;; differences in implementations which do not do line-at-a-time buffering.)

(defun read-sleepily (&optional (clear-p nil) (zzz 0))
  (list (progn (print '>) (read))
        ;; Note that input typed within the first ZZZ seconds 
        ;; will be discarded.
        (progn (print '>) 
               (if zzz (sleep zzz))
               (print '>>)
               (if clear-p (clear-input))
               (read))))

(read-sleepily)
\OUT > \IN{10}
\OUT >
\OUT >> \IN{20}
\EV (10 20)

(read-sleepily t)
\OUT > \IN{10}
\OUT >
\OUT >> \IN{20}
\EV (10 20)

(read-sleepily t 10)
\OUT > \IN{10}
\OUT > \IN{20}  ; Some implementations won't echo typeahead here.
\OUT >> \IN{30}
\EV (10 30)
\endcode


\label Side Effects::

The \param{input-stream} is modified.

\label Affected By::

\varref{*standard-input*}

\label Exceptional Situations::

\Shouldchecktype{input-stream}{a \term{stream designator}}

\label See Also::

\funref{clear-output}

\label Notes:\None.

\endcom

%%% ========== FINISH-OUTPUT
%%% ========== FORCE-OUTPUT
%%% ========== CLEAR-OUTPUT
\begincom{finish-output, force-output, clear-output}\ftype{Function}

\label Syntax::

\DefunWithValues {finish-output} {{\opt} output-stream} {\nil}
\DefunWithValues {force-output}  {{\opt} output-stream} {\nil}
\DefunWithValues {clear-output}  {{\opt} output-stream} {\nil}

\label Arguments and Values::

\param{output-stream}---an \term{output} \term{stream designator}.
 \Default{\term{standard output}}

\label Description::

\funref{finish-output}, \funref{force-output}, and \funref{clear-output}
exercise control over the internal handling of buffered stream output.  

\funref{finish-output} attempts to ensure that any buffered output
sent to \param{output-stream} has reached its destination, and then returns.

\funref{force-output} initiates the emptying of any
internal buffers but does not wait for completion
or acknowledgment to return.

%% 22.3.1 19
\funref{clear-output} attempts to abort any
outstanding output operation in progress in order
to allow as little output as possible
to continue to the destination.  

% This next sentence was added per Moon, since it was present in CLtL for CLEAR-INPUT
% and probably only accidentally omitted here. (I agree.) -kmp 11-Feb-92
If any of these operations does not make sense for \param{output-stream}, 
then it does nothing. 
%% 22.3.1 20
The precise actions of these \term{functions} are \term{implementation-dependent}.

\label Examples::
\code
;; Implementation A
 (progn (princ "am i seen?") (clear-output))
\EV NIL

;; Implementation B
 (progn (princ "am i seen?") (clear-output))
\OUT am i seen?
\EV NIL
\endcode

\label Side Effects:\None.

\label Affected By::

\varref{*standard-output*}

\label Exceptional Situations::

\Shouldchecktype{output-stream}{a \term{stream designator}}

%!!! Barmar: Are conditions signaled if stream is closed?

\label See Also::

\funref{clear-input}

\label Notes:\None.

\endcom

%-------------------- Query Functions --------------------

%%% ========== Y-OR-N-P
%%% ========== YES-OR-NO-P
\begincom{y-or-n-p, yes-or-no-p}\ftype{Function}

\label Syntax::

\DefunWithValues {y-or-n-p}    {{\opt} control {\rest} arguments} {generalized-boolean}
\DefunWithValues {yes-or-no-p} {{\opt} control {\rest} arguments} {generalized-boolean}

\label Arguments and Values::

\issue{FORMAT-STRING-ARGUMENTS:SPECIFY}
\param{control}---a \term{format control}.
\endissue{FORMAT-STRING-ARGUMENTS:SPECIFY}

\param{arguments}---\term{format arguments} for \param{control}.

\param{generalized-boolean}---a \term{generalized boolean}.

\label Description::

These functions ask a question and parse a response from the user.
They return \term{true} if the answer is affirmative,
or \term{false} if the answer is negative.

%% 22.4.0 2
%% 22.4.0 9
\funref{y-or-n-p} is for asking the user a question whose answer is either
``yes'' or ``no.''  
It is intended that the reply require
the user to answer a yes-or-no question with a single
character.
%% 22.4.0 7
\funref{yes-or-no-p} is also for asking the user a question
whose answer is either ``Yes'' or ``No.''  
It is intended that the reply require
the user to take more action than just a single keystroke, such as typing
the full word \f{yes} or \f{no} followed by a newline.


%% 22.4.0 6
%% this paragraph was left out


\funref{y-or-n-p} types out a message (if supplied), reads an answer
in some \term{implementation-dependent} manner (intended to be short and simple,
such as reading a single character such as \f{Y} or \f{N}).
\funref{yes-or-no-p} types out a message (if supplied), 
attracts the user's attention (for example, by ringing
the terminal's bell),
and reads an answer
in some \term{implementation-dependent} manner (intended to be multiple characters,
such as \f{YES} or \f{NO}).

%% 22.4.0 3
%% 22.4.0 8
If \param{format-control} is supplied and not \nil,
then a \funref{fresh-line} operation is performed; then
a message is printed as if \param{format-control} and \param{arguments}
were given to \funref{format}.
In any case, \funref{yes-or-no-p} and \funref{y-or-n-p} will provide 
a prompt such as ``\f{(Y or N)}'' or ``\f{(Yes or No)}'' if appropriate.

%% 22.4.0 4
%% 22.4.0 9
%% 22.4.0 1
All input and output are performed using \term{query I/O}.

\label Examples::
%% 22.4.0 5
\code
 (y-or-n-p "(t or nil) given by")
\OUT (t or nil) given by (Y or N) \IN{Y}
\EV \term{true}
 (yes-or-no-p "a ~S message" 'frightening) 
\OUT a FRIGHTENING message (Yes or No) \IN{no}
\EV \term{false}
 (y-or-n-p "Produce listing file?") 
\OUT Produce listing file?
\OUT Please respond with Y or N. \IN{n}
\EV \term{false}
\endcode

\label Side Effects::

Output to and input from \term{query I/O} will occur.

\label Affected By::

\varref{*query-io*}.

\label Exceptional Situations:\None.

\label See Also::

\funref{format}

\label Notes::

\funref{yes-or-no-p} and \funref{yes-or-no-p} do not add question marks
to the end of the prompt string, so any desired question mark or other
punctuation should be explicitly included in the text query.

\endcom

%-------------------- File Streams --------------------



%-------------------- Synonym Streams --------------------

%%% ========== MAKE-SYNONYM-STREAM
\begincom{make-synonym-stream}\ftype{Function}

\label Syntax::

\DefunWithValues make-synonym-stream {symbol} {synonym-stream}

\label Arguments and Values::

\param{symbol}---a \term{symbol} that names a \term{dynamic variable}.

\issue{STREAM-ACCESS:ADD-TYPES-ACCESSORS}
\param{synonym-stream}---a \term{synonym stream}.
\endissue{STREAM-ACCESS:ADD-TYPES-ACCESSORS}

\label Description::

%% 21.2.0 3
Returns a \term{synonym stream} whose \term{synonym stream symbol} is \param{symbol}.

\label Examples::

\code
 (setq a-stream (make-string-input-stream "a-stream")
        b-stream (make-string-input-stream "b-stream"))
\EV #<String Input Stream> 
 (setq s-stream (make-synonym-stream 'c-stream))
\EV #<SYNONYM-STREAM for C-STREAM> 
 (setq c-stream a-stream)
\EV #<String Input Stream> 
 (read s-stream) \EV A-STREAM
 (setq c-stream b-stream)
\EV #<String Input Stream> 
 (read s-stream) \EV B-STREAM
\endcode

\label Side Effects:\None.

\label Affected By:\None.

\label Exceptional Situations::

Should signal \typeref{type-error} if its argument is not a \term{symbol}.

\label See Also::

{\secref\StreamConcepts}

\label Notes:\None.

\endcom

%%% ========== SYNONYM-STREAM-SYMBOL
\begincom{synonym-stream-symbol}\ftype{Function}

\issue{STREAM-ACCESS:ADD-TYPES-ACCESSORS}

\label Syntax::

\DefunWithValues synonym-stream-symbol {synonym-stream} {symbol}

\label Arguments and Values::

\param{synonym-stream}---a \term{synonym stream}.

\param{symbol}---a \term{symbol}.

\label Description::

Returns the \term{symbol} whose \funref{symbol-value} the \param{synonym-stream} is using.  

\label Examples:\None.

\label Side Effects:\None.

\label Affected By:\None.

\label Exceptional Situations:\None.

\label See Also::

\funref{make-synonym-stream}

\label Notes:\None.

\endissue{STREAM-ACCESS:ADD-TYPES-ACCESSORS}

\endcom

%-------------------- Broadcast Streams --------------------

%%% ========== BROADCAST-STREAM-STREAMS
\begincom{broadcast-stream-streams}\ftype{Function}

\issue{STREAM-ACCESS:ADD-TYPES-ACCESSORS}

\label Syntax::

\DefunWithValues broadcast-stream-streams {broadcast-stream} {streams}

\label Arguments and Values::

\param{broadcast-stream}---a \term{broadcast stream}.

\param{streams}---a \term{list} of \term{streams}.

\label Description::

Returns a \term{list} of output \term{streams} that constitute
all the \term{streams} to which the \param{broadcast-stream} is broadcasting.  

\label Examples:\None.

\label Affected By:\None.

\label Exceptional Situations:\None.

\label See Also:\None.

\label Notes:\None.

\endissue{STREAM-ACCESS:ADD-TYPES-ACCESSORS}

\endcom

%%% ========== MAKE-BROADCAST-STREAM
\begincom{make-broadcast-stream}\ftype{Function}

\label Syntax::

\DefunWithValues make-broadcast-stream {{\rest} streams} {broadcast-stream}

\label Arguments and Values::

\param{stream}---an \term{output} \term{stream}.

\issue{STREAM-ACCESS:ADD-TYPES-ACCESSORS}

\param{broadcast-stream}---a \term{broadcast stream}.

\label Description::

%% 21.2.0 4
Returns a \term{broadcast stream}.
\endissue{STREAM-ACCESS:ADD-TYPES-ACCESSORS}

\label Examples::

\code
 (setq a-stream (make-string-output-stream)
        b-stream (make-string-output-stream)) \EV #<String Output Stream>
 (format (make-broadcast-stream a-stream b-stream)
          "this will go to both streams") \EV NIL
 (get-output-stream-string a-stream) \EV "this will go to both streams"
 (get-output-stream-string b-stream) \EV "this will go to both streams"
\endcode

\label Side Effects:\None.

%% Sandra thinks this is excessive.
%A \term{broadcast stream} is created.
 
\label Affected By:\None.

\label Exceptional Situations::

\Shouldcheckanytype{stream}{an \term{output} \term{stream}}

\label See Also::

\funref{broadcast-stream-streams}

\label Notes:\None.

\endcom

%-------------------- two-Way Streams --------------------

%%% ========== MAKE-TWO-WAY-STREAM
\begincom{make-two-way-stream}\ftype{Function}

\label Syntax::

\DefunWithValues make-two-way-stream {input-stream output-stream} {two-way-stream}

\label Arguments and Values::

\param{input-stream}---a \term{stream}.

\param{output-stream}---a \term{stream}.

\issue{STREAM-ACCESS:ADD-TYPES-ACCESSORS}
\param{two-way-stream}---a \term{two-way stream}.
\endissue{STREAM-ACCESS:ADD-TYPES-ACCESSORS}

\label Description::

%% 21.2.0 6
Returns a \term{two-way stream} 
that gets  its input  from \param{input-stream}
and  sends its output to   \param{output-stream}.

\label Examples::

\code
 (with-output-to-string (out)
    (with-input-from-string (in "input...")
      (let ((two (make-two-way-stream in out)))
        (format two "output...")
        (setq what-is-read (read two))))) \EV "output..."
 what-is-read \EV INPUT... 
\endcode

\label Side Effects:\None.

\label Affected By:\None.

\label Exceptional Situations::

\Shouldchecktype{input-stream}{an \term{input} \term{stream}}
\Shouldchecktype{output-stream}{an \term{output} \term{stream}}

\label See Also:\None.

\label Notes:\None.

\endcom

%%% ========== TWO-WAY-STREAM-INPUT-STREAM
%%% ========== TWO-WAY-STREAM-OUTPUT-STREAM
\begincom{two-way-stream-input-stream, two-way-stream-output-stream}\ftype{Function}

\issue{STREAM-ACCESS:ADD-TYPES-ACCESSORS}

\label Syntax::

\DefunWithValues two-way-stream-input-stream  {two-way-stream} {input-stream}
\DefunWithValues two-way-stream-output-stream {two-way-stream} {output-stream}

\label Arguments and Values::

\param{two-way-stream}---a \term{two-way stream}.

\param{input-stream}---an \term{input} \term{stream}.

\param{output-stream}---an \term{output} \term{stream}.

\label Description::

\funref{two-way-stream-input-stream} returns the \term{stream}
from which \param{two-way-stream} receives input.

\funref{two-way-stream-output-stream} returns the \term{stream} 
to which \param{two-way-stream} sends output.

\label Examples:\None.

\label Side Effects:\None.

\label Affected By:\None.

\label Exceptional Situations:\None.

\label See Also:\None.

\label Notes:\None.

\endissue{STREAM-ACCESS:ADD-TYPES-ACCESSORS}

\endcom

%-------------------- Echo Streams --------------------

%%% ========== ECHO-STREAM-INPUT-STREAM
%%% ========== ECHO-STREAM-OUTPUT-STREAM
\begincom{echo-stream-input-stream, echo-stream-output-stream}\ftype{Function}

\issue{STREAM-ACCESS:ADD-TYPES-ACCESSORS}

\label Syntax::

\DefunWithValues echo-stream-input-stream  {echo-stream} {input-stream}
\DefunWithValues echo-stream-output-stream {echo-stream} {output-stream}

\label Arguments and Values::

\param{echo-stream}---an \term{echo stream}.

\param{input-stream}---an \term{input} \term{stream}.

\param{output-stream}---an \term{output} \term{stream}.

\label Description::

\funref{echo-stream-input-stream} returns the \term{input} \term{stream} 
from which \param{echo-stream} receives input.

\funref{echo-stream-output-stream} returns the \term{output} \term{stream}
to which \param{echo-stream} sends output.

\label Examples:\None.

\label Side Effects:\None!

\label Affected By:\None.

\label Exceptional Situations:\None.

\label See Also:\None.

\label Notes:\None.

\endissue{STREAM-ACCESS:ADD-TYPES-ACCESSORS}

\endcom

%%% ========== MAKE-ECHO-STREAM
\begincom{make-echo-stream}\ftype{Function}

\label Syntax::

\issue{STREAM-ACCESS:ADD-TYPES-ACCESSORS}

\DefunWithValues make-echo-stream {input-stream output-stream} {echo-stream}

\label Arguments and Values::

\param{input-stream}---an \term{input} \term{stream}.

\param{output-stream}---an \term{output} \term{stream}.

\param{echo-stream}---an \term{echo stream}.

\label Description::

%% 21.2.0 7
Creates and returns an \term{echo stream} 
that takes input  from \param{input-stream}
and  sends output to   \param{output-stream}.

\endissue{STREAM-ACCESS:ADD-TYPES-ACCESSORS}

\label Examples::
\code
 (let ((out (make-string-output-stream)))
    (with-open-stream 
        (s (make-echo-stream
            (make-string-input-stream "this-is-read-and-echoed")
            out))
      (read s)
      (format s " * this-is-direct-output")
      (get-output-stream-string out)))
\EV "this-is-read-and-echoed * this-is-direct-output"
\endcode

\label Side Effects:\None.

\label Affected By:\None.

\label Exceptional Situations:\None.

\label See Also::

\funref{echo-stream-input-stream},
\funref{echo-stream-output-stream},
\funref{make-two-way-stream}

\label Notes:\None.

\endcom

%-------------------- Concatenated Streams --------------------

%%% ========== CONCATENATED-STREAM-STREAMS
\begincom{concatenated-stream-streams}\ftype{Function}

\issue{STREAM-ACCESS:ADD-TYPES-ACCESSORS}

\label Syntax::

\DefunWithValues concatenated-stream-streams {concatenated-stream} {streams}

\label Arguments and Values::

\param{concatenated-stream}---a \term{concatenated stream}.

\param{streams}---a \term{list} of \term{input} \term{streams}.

\label Description::

Returns a \term{list} of \term{input} \term{streams} that constitute the
ordered set of \term{streams} the \param{concatenated-stream} still 
has to read from, starting with the current one it is reading from.
The list may be \term{empty} if no more \term{streams} remain to be read.

The consequences are undefined if the \term{list structure} of the \param{streams}
is ever modified.

\label Examples:\None.

\label Side Effects:\None.

\label Affected By:\None.

\label Exceptional Situations:\None.

\label See Also:\None.

\label Notes:\None.

\endissue{STREAM-ACCESS:ADD-TYPES-ACCESSORS}

\endcom

%%% ========== MAKE-CONCATENATED-STREAM
\begincom{make-concatenated-stream}\ftype{Function}

\label Syntax::

\DefunWithValues make-concatenated-stream {{\rest} input-streams} {concatenated-stream}

\label Arguments and Values::

\param{input-stream}---an \term{input} \term{stream}.

\issue{STREAM-ACCESS:ADD-TYPES-ACCESSORS}

\param{concatenated-stream}---a \term{concatenated stream}.

\label Description::

%% 21.2.0 5
Returns a \term{concatenated stream} that has the indicated \param{input-streams}
initially associated with it.

\endissue{STREAM-ACCESS:ADD-TYPES-ACCESSORS}

\label Examples::
%% POSSIBLE CLEAN-UP ISSUE Symbolics returns 1.
\code
 (read (make-concatenated-stream
         (make-string-input-stream "1")
         (make-string-input-stream "2"))) \EV 12
\endcode

\label Side Effects:\None.

\label Affected By:\None.

\label Exceptional Situations::

Should signal \typeref{type-error} if any argument is not an \term{input} \term{stream}.

\label See Also::

\funref{concatenated-stream-streams}

\label Notes:\None.

\endcom

%-------------------- String Streams --------------------

%%% ========== GET-OUTPUT-STREAM-STRING
\begincom{get-output-stream-string}\ftype{Function}

\label Syntax::

\DefunWithValues get-output-stream-string {string-output-stream} {string}

\label Arguments and Values::

\param{string-output-stream}---a \term{stream}.

\param{string}---a \term{string}.

\label Description::

%% 21.2.0 10
Returns a \term{string} containing, in order, all the \term{characters} 
that have been output to \param{string-output-stream}.
This operation clears any \term{characters} on \param{string-output-stream},
so the \param{string} contains only those \term{characters} which have been output
   since the last call to \funref{get-output-stream-string}
or since the creation of the \param{string-output-stream},
whichever occurred most recently.

\label Examples::
\code
 (setq a-stream (make-string-output-stream)
        a-string "abcdefghijklm") \EV "abcdefghijklm"
 (write-string a-string a-stream) \EV "abcdefghijklm"
 (get-output-stream-string a-stream) \EV "abcdefghijklm"
 (get-output-stream-string a-stream) \EV ""
\endcode


\label Side Effects::

The \param{string-output-stream} is cleared.

\label Affected By:\None.

\label Exceptional Situations::

\issue{CLOSE-CONSTRUCTED-STREAM:ARGUMENT-STREAM-ONLY}
%The result of calling \funref{get-output-stream-string} on 
% the \param{string-stream} returned by \funref{make-string-output-stream} 
% is unspecified after \funref{close}. 
%% Rewritten for Sandra:
The consequences are undefined if \param{stream-output-string} is \term{closed}.
\endissue{CLOSE-CONSTRUCTED-STREAM:ARGUMENT-STREAM-ONLY}

The consequences are undefined if \param{string-output-stream} is a \term{stream} that
was not produced by \funref{make-string-output-stream}.
\issue{STRING-OUTPUT-STREAM-BASHING:UNDEFINED}
The consequences are undefined if \param{string-output-stream} was 
created implicitly by \macref{with-output-to-string} or \funref{format}.

%% Redundant -- Said already in WITH-OUTPUT-TO-STRING and FORMAT. -kmp 14-Feb-92
% In the cases where a \term{string} argument with a \term{fill pointer} is supplied
% as an \term{argument} to \macref{with-output-to-string} or \funref{format},
% the consequences are undefined if destructive modifications are performed
% directly on the \term{string} during the \term{dynamic extent} of the call.
\endissue{STRING-OUTPUT-STREAM-BASHING:UNDEFINED}

\label See Also::

\funref{make-string-output-stream}

\label Notes:\None.

\endcom

%%% ========== MAKE-STRING-INPUT-STREAM
\begincom{make-string-input-stream}\ftype{Function}

\label Syntax::

\DefunWithValues make-string-input-stream {string {\opt} start end} {string-stream}

\label Arguments and Values::

\param{string}---a \term{string}.

\issue{RANGE-OF-START-AND-END-PARAMETERS:INTEGER-AND-INTEGER-NIL}
\param{start}, \param{end}---\term{bounding index designators} of \param{string}.
 \Defaults{\param{start} and \param{end}}{\f{0} and \nil}
\endissue{RANGE-OF-START-AND-END-PARAMETERS:INTEGER-AND-INTEGER-NIL}

\issue{STREAM-ACCESS:ADD-TYPES-ACCESSORS}
\param{string-stream}---an \term{input} \term{string stream}.
\endissue{STREAM-ACCESS:ADD-TYPES-ACCESSORS}

\label Description::

%% 21.2.0 8
Returns an \term{input} \term{string stream}.  
This \term{stream} will supply, in order, the \term{characters} in the substring
of \param{string} \term{bounded} by \param{start} and \param{end}.
After the last \term{character} has been supplied, 
the \term{string stream} will then be at \term{end of file}.

%% Implied by bounded. -kmp 14-Feb-92
% \param{Start} marks the beginning position of the substring.
% \param{End}   marks the position following the last element of the substring.

\label Examples::

\code
 (let ((string-stream (make-string-input-stream "1 one ")))
   (list (read string-stream nil nil)
         (read string-stream nil nil)
         (read string-stream nil nil)))
\EV (1 ONE NIL)

 (read (make-string-input-stream "prefixtargetsuffix" 6 12)) \EV TARGET
\endcode

\label Side Effects:\None.

\label Affected By:\None.

\label Exceptional Situations:\None.

\label See Also::

\macref{with-input-from-string}

\label Notes:\None.

\endcom


%%% ========== MAKE-STRING-OUTPUT-STREAM
\begincom{make-string-output-stream}\ftype{Function}

\label Syntax::

\DefunWithValues make-string-output-stream {{\key} element-type} {string-stream}

\label Arguments and Values::

\issue{CHARACTER-PROPOSAL:2-5-6}
\param{element-type}---a \term{type specifier}.
  \Default{\typeref{character}}
\endissue{CHARACTER-PROPOSAL:2-5-6}

\issue{STREAM-ACCESS:ADD-TYPES-ACCESSORS}
\param{string-stream}---an \term{output} \term{string stream}.
\endissue{STREAM-ACCESS:ADD-TYPES-ACCESSORS}

\label Description::

%% 21.2.0 9
Returns
\issue{CHARACTER-PROPOSAL:2-5-6}
an \term{output} \term{string stream} that accepts \term{characters}
and makes available (via \funref{get-output-stream-string})
a \term{string} that contains the \term{characters} that were actually output.

The \param{element-type} names the \term{type} of the \term{elements}
of the \term{string}; a \term{string} is constructed of the most specialized
\term{type} that can accommodate \term{elements} of that \term{element-type}.
\endissue{CHARACTER-PROPOSAL:2-5-6}

\label Examples::

\code
 (let ((s (make-string-output-stream)))
   (write-string "testing... " s)
   (prin1 1234 s)
   (get-output-stream-string s))
\EV "testing... 1234"
\endcode

\label Affected By:\None.

\label Exceptional Situations:\None.

\None.

\label See Also::

\funref{get-output-stream-string}, \macref{with-output-to-string}

\label Notes:\None.

\endcom

%%% ========== WITH-INPUT-FROM-STRING
\begincom{with-input-from-string}\ftype{Macro}

\issue{DECLS-AND-DOC}

\label Syntax::

\DefmacWithValuesNewline with-input-from-string
			 {\paren{var string {\key} index start end}
			  \starparam{declaration}
			  \starparam{form}}
			 {\starparam{result}}

\label Arguments and Values::

\param{var}---a \term{variable} \term{name}.

\param{string}---a \term{form}; evaluated to produce a \term{string}.

\param{index}---a \term{place}.

\issue{SUBSEQ-OUT-OF-BOUNDS:IS-AN-ERROR}
\param{start}, \param{end}---\term{bounding index designators} of \param{string}.
 \Defaults{\param{start} and \param{end}}{\f{0} and \nil}
\endissue{SUBSEQ-OUT-OF-BOUNDS:IS-AN-ERROR}

\param{declaration}---a \misc{declare} \term{expression}; \noeval.

\param{forms}---an \term{implicit progn}.

\param{result}---the \term{values} returned by the \param{forms}.

\label Description::

Creates an
\issue{STREAM-ACCESS:ADD-TYPES-ACCESSORS}
\term{input} \term{string stream}, 
\endissue{STREAM-ACCESS:ADD-TYPES-ACCESSORS}
provides an opportunity to perform operations on the \term{stream} 
 (returning zero or more \term{values}),
and then closes the \term{string stream}.

%% 21.2.0 12
\param{String} is evaluated first, and \param{var} is bound to
a character \term{input} \term{string stream} that supplies 
\term{characters} from the subsequence of the resulting \term{string} \term{bounded} by
\param{start} and \param{end}. 
The body is executed as an \term{implicit progn}.

%% 21.2.0 13
The \term{input} \term{string stream} is automatically closed on exit from
\macref{with-input-from-string}, no matter whether the exit is normal or abnormal.
\issue{WITH-OPEN-FILE-STREAM-EXTENT:DYNAMIC-EXTENT}
The \term{input} \term{string stream} to which the \term{variable} \param{var}
is \term{bound} has \term{dynamic extent};
its \term{extent} ends when the \term{form} is exited.
\endissue{WITH-OPEN-FILE-STREAM-EXTENT:DYNAMIC-EXTENT}

%% 21.2.0 14
%% 21.2.0 15
% \beginlist
% \itemitem{\kwd{index}}

The \param{index} is a pointer within the \param{string} to be advanced.
If \macref{with-input-from-string}
is exited normally, then \param{index} will have
% stored into it 
as its \term{value}
the index into the \param{string} indicating the first character not read
which is {\tt (length \param{string})} if all characters were used.
The place specified by \param{index}
is not updated as reading progresses, but only at the
end of the operation. 

%% This is redundant with "bounded".
% %% 21.2.0 16
% \itemitem{\kwd{start}}
% 
% \kwd{start} indicates the beginning of a substring of \param{string} to be used.
% If \kwd{start} is not supplied or \nil, the beginning position is 0.
% 
% %% 21.2.0 17
% \itemitem{\kwd{end}}
% 
% \kwd{end} indicates the end of a substring of \param{string} to be used.
% If \kwd{end} is not supplied or \nil, the ending position is {\tt (length \param{string})}.
% \endlist

%% 21.2.0 19
\param{start} and \param{index} may both specify the same variable, 
which is a pointer within the \param{string} to be advanced,
perhaps repeatedly by some containing loop.

\issue{WITH-OPEN-FILE-SETQ:EXPLICITLY-VAGUE}
The consequences are undefined if an attempt is made to \term{assign} 
the \term{variable} \param{var}.
%% Sandra thinks this isn't needed.
% The compiler may choose to issue a
% warning if such an attempt is detected.
\endissue{WITH-OPEN-FILE-SETQ:EXPLICITLY-VAGUE}

\label Examples::
%% 21.2.0 18
\code
 (with-input-from-string (s "XXX1 2 3 4xxx"
                             :index ind
                             :start 3 :end 10)
    (+ (read s) (read s) (read s))) \EV 6
 ind \EV 9
 (with-input-from-string (s "Animal Crackers" :index j :start 6)
   (read s)) \EV CRACKERS
\endcode
The variable \f{j} is set to \f{15}.

\label Side Effects::

The \term{value} of the \term{place} named by \param{index}, if any, is modified.

%% Sandra thinks this is excessive.
%A \term{string stream} is created (upon entry) and closed (upon exit).

\label Affected By:\None.

\label Exceptional Situations:\None.

\label See Also::

\funref{make-string-input-stream},
\issue{MAPPING-DESTRUCTIVE-INTERACTION:EXPLICITLY-VAGUE}
{\secref\TraversalRules}
\endissue{MAPPING-DESTRUCTIVE-INTERACTION:EXPLICITLY-VAGUE}


\label Notes:\None.

\endissue{DECLS-AND-DOC}

\endcom

%%% ========== WITH-OUTPUT-TO-STRING
\begincom{with-output-to-string}\ftype{Macro}

\issue{DECLS-AND-DOC}

\label Syntax::

\DefmacWithValuesNewline with-output-to-string
		  	 {\paren{var {\opt} string-form {\key} element-type}
 			  \starparam{declaration}
 			  \starparam{form}}
			 {\starparam{result}}

\label Arguments and Values::

\param{var}---a \term{variable} \term{name}.

\param{string-form}---a \term{form} or \nil;
		      if \term{non-nil}, evaluated to produce \param{string}.

\param{string}---a \term{string} that has a \term{fill pointer}.

\issue{CHARACTER-PROPOSAL:2-5-6}
\param{element-type}---a \term{type specifier}; \eval.
\issue{CHARACTER-PROPOSAL:2-3-6}
  \Default{\typeref{character}}
\endissue{CHARACTER-PROPOSAL:2-3-6}
\endissue{CHARACTER-PROPOSAL:2-5-6}

\param{declaration}---a \misc{declare} \term{expression}; \noeval.

\param{forms}---an \term{implicit progn}.

\param{results}---If a \param{string-form} is not supplied or \nil,
		  a \term{string}; otherwise, 
		  the \term{values} returned by the \param{forms}.

\label Description::

\macref{with-output-to-string} creates a
\issue{STREAM-ACCESS:ADD-TYPES-ACCESSORS}
character \term{output} \term{stream}, performs a series of operations
that may send results to this \term{stream}, and then closes the \term{stream}. 
\endissue{STREAM-ACCESS:ADD-TYPES-ACCESSORS}

\issue{CHARACTER-PROPOSAL:2-3-6}
The \param{element-type} names the \term{type} of the elements
of the \term{stream}; a \term{stream} is constructed of the most specialized
\term{type} that can accommodate elements of the given \term{type}.  
\endissue{CHARACTER-PROPOSAL:2-3-6}

%% 21.2.0 20
The body is executed as an \term{implicit progn} with \param{var}
bound to an \term{output} \term{string stream}.  
All output to that \term{string stream} is saved in a \term{string}.  

%% 21.2.0 22
\issue{WITH-OUTPUT-TO-STRING-APPEND-STYLE:VECTOR-PUSH-EXTEND}
If \param{string} is supplied, \param{element-type} is ignored, 
and the output is incrementally appended to \param{string} as
if by use of \funref{vector-push-extend}.
\endissue{WITH-OUTPUT-TO-STRING-APPEND-STYLE:VECTOR-PUSH-EXTEND}

%% 21.2.0 24
%% Sandra: Huh?
%In either case, 
The \term{output} \term{stream} 
is automatically closed on exit from \macref{with-output-to-string},
no matter whether the exit is normal or abnormal.
\issue{WITH-OPEN-FILE-STREAM-EXTENT:DYNAMIC-EXTENT}
The \term{output} \term{string stream} to which the \term{variable} \param{var}
is \term{bound} has \term{dynamic extent};
its \term{extent} ends when the \term{form} is exited.
\endissue{WITH-OPEN-FILE-STREAM-EXTENT:DYNAMIC-EXTENT}


%% 21.2.0 21
If no \param{string} is provided, then \macref{with-output-to-string} 
\issue{CHARACTER-PROPOSAL:2-5-6}
produces a \term{stream} that accepts characters and returns a \term{string} 
of the indicated \param{element-type}.
\endissue{CHARACTER-PROPOSAL:2-5-6}
If \param{string} is provided,
\macref{with-output-to-string} returns the results of evaluating the last \param{form}.

\issue{WITH-OPEN-FILE-SETQ:EXPLICITLY-VAGUE}
The consequences are undefined if an attempt is made to \term{assign}
the \term{variable} \param{var}.
%% Sandra: unnecessary.
%The compiler may choose to issue a
%warning if such an attempt is detected.
\endissue{WITH-OPEN-FILE-SETQ:EXPLICITLY-VAGUE}

\label Examples::
\code
 (setq fstr (make-array '(0) :element-type 'base-char
                             :fill-pointer 0 :adjustable t)) \EV ""
 (with-output-to-string (s fstr)
    (format s "here's some output")
    (input-stream-p s)) \EV \term{false}
 fstr \EV "here's some output"
\endcode

\label Side Effects::

The \param{string} is modified.

\label Affected By:\None.

\label Exceptional Situations::

\issue{STRING-OUTPUT-STREAM-BASHING:UNDEFINED}
The consequences are undefined if destructive modifications are performed
directly on the \param{string} during the \term{dynamic extent} of the call.
\endissue{STRING-OUTPUT-STREAM-BASHING:UNDEFINED}

\label See Also::

\funref{make-string-output-stream},
\funref{vector-push-extend},
\issue{MAPPING-DESTRUCTIVE-INTERACTION:EXPLICITLY-VAGUE}
{\secref\TraversalRules}
\endissue{MAPPING-DESTRUCTIVE-INTERACTION:EXPLICITLY-VAGUE}

\label Notes:\None.

\endissue{DECLS-AND-DOC}

\endcom

%-------------------- Stream Variables --------------------

% \def\iovarconstraint#1#2 #3{
% \issue{STANDARD-INPUT-INITIAL-BINDING:DEFINED-CONTRACTS}
% \term{implementation-dependent}, but it must be an \term{open},
% \term{#1} \term{stream} which is not a 
% \term{generalized synonym stream} to any of the \term{symbols} #3, but
% which might be a \term{generalized synonym stream} to the 
% \term{value} of one of those \term{symbols}.  #2
% \endissue{STANDARD-INPUT-INITIAL-BINDING:DEFINED-CONTRACTS}
% }
% 
% \def\TerminalIOsynOK{%
% The initial value might also be a \term{generalized synonym stream} 
% to either the \term{symbol} \varref{*terminal-io*} or to the \term{stream} 
% which is its \term{value}.}
% 
% \def\iovarrationale#1{
% \issue{STANDARD-INPUT-INITIAL-BINDING:DEFINED-CONTRACTS}
% The intent of the constraints on the initial \term{value} 
% is to ensure that it is always safe to bind \varref{*#1*} 
% to one of the above-mentioned variables without
% unduly restricting implementation flexibility.
% \endissue{STANDARD-INPUT-INITIAL-BINDING:DEFINED-CONTRACTS}
% }
%
% %%% ========== *DEBUG-IO*
% \begincom{*debug-io*}\ftype{Variable}
% 
% \label Value Type::
% 
% a \term{bidirectional} \term{stream}.
% 
% \label Initial Value::
% 
% \iovarconstraint{bidirectional}{\TerminalIOsynOK}
%  {\varref{*standard-input*},
%   \varref{*standard-output*}, 
%   \varref{*error-output*},
%   \varref{*trace-output*}, or
%   \varref{*query-io*}}
% 
% \label Description::
% 
% %% 21.1.0 7
% A \term{stream} to be used for interactive debugging purposes.  
% 
% \label Examples:\None.
% 
% %% Moon: Not only is this example under *debug-io* disgusting, it is not correct Common
% %%       Lisp because it violates dynamic extent:
% %% KMP:  I concur! Commented out.
% %
% % \code
% %  (with-output-to-string (out)
% %    (with-input-from-string (in "enter > ")
% %      (let ((two-way (make-two-way-stream in out)))
% %        (setq *debug-io* two-way)))) \EV ""           
% %  *debug-io* \EV #<TWO-WAY-STREAM>
% % \endcode
% 
% \label Affected By:\None.
% 
% \label See Also::
% 
% \varref{*query-io*},
% \funref{make-synonym-stream}
% 
% \label Notes::
% 
% \iovarrationale{debug-io}
% 
% Frequently \varref{*debug-io*} is bound to the same \term{stream} as \varref{*query-io*}.
% 
% \endcom
% 
% %%% ========== *ERROR-OUTPUT*
% \begincom{*error-output*}\ftype{Variable}
% 
% \label Value Type::
% 
% an \term{output} \term{stream}.
% 
% \label Initial Value::
% 
% \iovarconstraint{output}{\TerminalIOsynOK}
%  {\varref{*standard-input*},
%   \varref{*standard-output*},
%   \varref{*trace-output*},
%   \varref{*query-io*}, or
%   \varref{*debug-io*}}
% 
% \label Description::
% 
% %% 21.1.0 5
% \Thevalueof{*error-output*} is a \term{stream} to which error messages should be sent.  
% 
% \label Examples::
% 
% \code
%  (with-output-to-string (*error-output*)
%    (warn "this string is sent to *error-output*"))
%  \EV "Warning: this string is sent to *error-output*
% " ;The exact format of this string is \term{implementation-dependent}.
% \endcode
% 
% \label Affected By:\None.
% 
% %%Sandra: Gratuitous
% %The \term{implementation}.
% 
% \label See Also::
% 
% \varref{*standard-output*},
% \varref{*terminal-io*},
% \funref{make-synonym-stream},
% \funref{warn},
% \funref{error},
% \funref{cerror},
% \funref{break},
% \macref{check-type},
% \macref{assert},
% \macref{etypecase},
% \macref{ctypecase},
% \macref{ecase},
% \macref{ccase}
%      
% \label Notes::
% 
% \iovarrationale{error-output}
% 
% Frequently \varref{*error-output*} is bound to the \term{same} \term{stream}
% as \varref{*standard-output*}.
% 
% \endcom
% 
% %%% ========== *QUERY-IO*
% \begincom{*query-io*}\ftype{Variable}
% 
% \label Value Type::
% 
% a \term{bidirectional} \term{stream}.
% 
% \label Initial Value::
% 
% \iovarconstraint{bidirectional}{\TerminalIOsynOK}
%  {\varref{*standard-input*},
%   \varref{*standard-output*},
%   \varref{*error-output*},
%   \varref{*trace-output*}, or
%   \varref{*debug-io*}}
% 
% \label Description::
% 
% %% 21.1.0 6
% \Thevalueof{*query-io*}, \term{query I/O}, is a \term{bidirectional} \term{stream} 
% to be used when asking questions of the user.  The question should be output 
% to this \term{stream}, and the answer read from it.
% 
% \label Examples:\None.
% 
% \label Affected By:\None.
% 
% %%Sandra: Gratuitous
% %The \term{implementation}.
% 
% \label See Also::
% 
% \varref{*debug-io*},
% \varref{*terminal-io*},
% \funref{make-synonym-stream},
% \funref{y-or-n-p},
% \funref{yes-or-no-p}
% 
% \label Notes::
% 
% \iovarrationale{query-io}
% 
% %\varref{*query-io*} is normally a \term{synonym stream} to \varref{*terminal-io*}.
% 
% \varref{*query-io*} is used by such functions as \funref{y-or-n-p}
% and \funref{yes-or-no-p}.
% \endcom
% 
% %%% ========== *STANDARD-INPUT*
% \begincom{*standard-input*}\ftype{Variable}
% 
% \label Value Type::
% 
% an \term{input} \term{stream}.
% 
% \label Initial Value::
% 
% \iovarconstraint{input}{\TerminalIOsynOK}
%  {\varref{*standard-output*},
%   \varref{*error-output*},
%   \varref{*trace-output*},
%   \varref{*query-io*}, or
%   \varref{*debug-io*}}
% 
% \label Description::
% 
% \issue{STANDARD-INPUT-INITIAL-BINDING:DEFINED-CONTRACTS}
% \Thevalueof{*standard-input*}, \term{standard input}, is a \term{stream}
% that is used by many \term{operators} as a default source of input when 
% no specific \term{input} \term{stream} is explicitly supplied.
% \endissue{STANDARD-INPUT-INITIAL-BINDING:DEFINED-CONTRACTS} 
% 
% \label Examples::
% 
% \code
%  (with-input-from-string (*standard-input* "1001")
%     (+ 990 (read))) \EV 1991                       
% \endcode
% 
% \label Affected By:\None.
% 
% %%Sandra: Gratuitous
% %The \term{implementation}.
% 
% \label See Also::
% 
% \funref{make-synonym-stream}
% 
% \label Notes::
% 
% \iovarrationale{standard-input}
% 
% %% 21.1.0 3
% %% 22.2.1 1
% In the normal \term{Lisp read-eval-print loop}, input is read from
% \term{standard input}.  Many input functions, including \funref{read} 
% and \funref{read-char}, take a \term{stream} argument that defaults 
% to \term{standard input}.
% 
% \endcom
% 
% 
% %%% ========== *STANDARD-OUTPUT*
% \begincom{*standard-output*}\ftype{Variable}
% 
% \label Value Type::
% 
% an \term{output} \term{stream}.
% 
% \label Initial Value::
% 
% \iovarconstraint{output}{\TerminalIOsynOK}
%  {\varref{*standard-input*},
%   \varref{*error-output*},
%   \varref{*trace-output*},
%   \varref{*query-io*}, or
%   \varref{*debug-io*}}
% 
% \label Description::
% 
% \Thevalueof{*standard-output*}, \term{standard output}, is a \term{stream}
% that is used by many \term{operators} as a default destination for output
% when no specific \term{output} \term{stream} is explicitly supplied.
% 
% \label Examples::
% \code
%  (progn (setq out (with-output-to-string (*standard-output*)
%                      (print "print and format t send things to")
%                      (format t "*standard-output* now going to a string")))
%         :done)
% \EV :DONE
%  out
% \EV "
% \\"print and format t send things to\\" *standard-output* now going to a string"
% \endcode
% 
% \label Affected By:\None.
% 
% %%Sandra: Gratuitous
% %The \term{implementation}.
% 
% \label See Also::
% 
% \funref{make-synonym-stream}
% 
% \label Notes::
% 
% %% 21.1.0 4
% In the normal \term{Lisp read-eval-print loop}, output is sent to \term{standard output}.
% Many output functions, including \funref{print} and \funref{write-char}, 
% take a \term{stream} argument that defaults to \term{standard output}.
% 
% %!!! This needs to be cross-checked against the new descriptions of these streams.
% %% 21.1.0 11
% A program that wants, for example, to divert output to a file should do so by 
% \term{binding} \varref{*standard-output*}; that way error messages sent to
% \varref{*error-output*} can still get to the user by going through
% \varref{*terminal-io*} (if \varref{*error-output*} is bound to \varref{*terminal-io*}),
% which is usually what is desired.
% 
% \endcom
% 
% 
% %%% ========== *TRACE-OUTPUT*
% \begincom{*trace-output*}\ftype{Variable}
% 
% \label Value Type::
% 
% an \term{output} \term{stream}.
% 
% \label Initial Value::
% 
% \iovarconstraint{output}{\TerminalIOsynOK}
%  {\varref{*standard-input*},
%   \varref{*standard-output*},
%   \varref{*error-output*},
%   \varref{*query-io*}, or
%   \varref{*debug-io*}}
% 
% \label Description::
% 
% %% 21.1.0 9
% The \term{stream} on which traced functions (see \macref{trace})
% and \themacro{time} print their output.
% 
% \label Examples::
% 
% \code
%  (defun fact (n) (if (< n 2) 1 (* n (fact (- n 1)))))
% \EV FACT
%  (trace fact)
% \EV (FACT)
% ;; Of course, the format of traced output is implementation-dependent.
%  (with-output-to-string (*trace-output*)
%    (fact 3)) 
% \EV "
% 1 Enter FACT 3
% | 2 Enter FACT 2
% |   3 Enter FACT 1
% |   3 Exit FACT 1
% | 2 Exit FACT 2
% 1 Exit FACT 6"
% \endcode
% 
% \label Affected By:\None.
% 
% %%Sandra: Gratuitous
% %The \term{implementation}.
% 
% \label See Also::
% 
% \funref{make-synonym-stream},
% \macref{trace},
% \macref{time}.
%            
% \label Notes::
% 
% \iovarrationale{trace-output}
% 
% \endcom
%
% %%% ========== *TERMINAL-IO*
% \begincom{*terminal-io*}\ftype{Variable}
% 
% \label Value Type::
% 
% a \term{bidirectional} \term{stream}.
% 
% \label Initial Value::
% 
% \editornote{KMP: Doesn't debug-io belong in this list?}
% 
% \iovarconstraint{bidirectional}{}
%  {\varref{*standard-input*},
%   \varref{*standard-output*},
%   \varref{*error-output*},
%   \varref{*trace-output*}, or
%   \varref{*query-io*}}
% 
% \label Description::
% 
% %% 21.1.0 8
% \Thevalueof{*terminal-io*}, \term{terminal I/O}, is ordinarily 
% a \term{bidirectional} \term{stream} that connects to the user's console.
% Typically, writing to this \term{stream} 
% would cause the output to appear
% on a display screen, for example, and reading from the \term{stream} would
% accept input from a keyboard.  It is intended
% that standard input functions such as \funref{read} and \funref{read-char},
% when used with this \term{stream}, cause echoing of the input
% into the output side of the \term{stream}. The means by which this is
% accomplished are \term{implementation-dependent}.
% 
% %% 21.1.0 11, clarified by KMP
% The effect of changing \thevalueof{*terminal-io*},
% either by \term{binding} or \term{assignment},
% is \term{implementation-defined}.
% 
% \label Examples::
% 
% \code
%  (progn (prin1 'foo) (prin1 'bar *terminal-io*))
% \OUT FOOBAR
% \EV BAR
%  (with-output-to-string (*standard-output*)
%    (prin1 'foo) 
%    (prin1 'bar *terminal-io*))
% \OUT BAR
% \EV "FOO"
% \endcode
% 
% \label Affected By:\None.
% 
% %%Sandra: Gratuitous
% %The \term{implementation}.
% 
% \label See Also:\None.
% 
% \label Notes:\None.
% 
% \endcom

%%% ========== *DEBUG-IO*
%%% ========== *ERROR-OUTPUT*
%%% ========== *QUERY-IO*
%%% ========== *STANDARD-INPUT*
%%% ========== *STANDARD-OUTPUT*
%%% ========== *TRACE-OUTPUT*
\begincom{*debug-io*, *error-output*, *query-io*,
	  *standard-input*, *standard-output*,
	  *trace-output*}\ftype{Variable}

\label Value Type::

For \varref{*standard-input*}:
 an \term{input} \term{stream}

For \varref{*error-output*}, \varref{*standard-output*}, and \varref{*trace-output*}:
 an \term{output} \term{stream}.

For \varref{*debug-io*}, \varref{*query-io*}:
 a \term{bidirectional} \term{stream}.

\label Initial Value::

\term{implementation-dependent}, but
it must be an \term{open} \term{stream} 
that is not a  \term{generalized synonym stream}
to an \term{I/O customization variables}
but that might be a \term{generalized synonym stream} to 
the value of some \term{I/O customization variable}.
The initial value might also be a \term{generalized synonym stream} 
to either the \term{symbol} \varref{*terminal-io*} or to the \term{stream} 
that is its \term{value}.

\label Description::

These \term{variables} are collectively called the
\term{standardized} \term{I/O customization variables}.
They can be \term{bound} or \term{assigned} in order to 
change the default destinations for input and/or output 
used by various \term{standardized} \term{operators} and facilities.

%% 21.1.0 7
\Thevalueof{*debug-io*}, called \term{debug I/O},
is a \term{stream} to be used for interactive debugging purposes.  

%% 21.1.0 5
\Thevalueof{*error-output*}, called \term{error output},
is a \term{stream} to which warnings and non-interactive error messages should be sent.

%% 21.1.0 6
\Thevalueof{*query-io*}, called \term{query I/O},
is a \term{bidirectional} \term{stream} 
to be used when asking questions of the user.  The question should be output 
to this \term{stream}, and the answer read from it.

\issue{STANDARD-INPUT-INITIAL-BINDING:DEFINED-CONTRACTS}
\Thevalueof{*standard-input*}, called \term{standard input},
is a \term{stream} that is used by many \term{operators} 
as a default source of input when no specific \term{input} \term{stream}
is explicitly supplied.
\endissue{STANDARD-INPUT-INITIAL-BINDING:DEFINED-CONTRACTS} 

\Thevalueof{*standard-output*}, called \term{standard output},
is a \term{stream} that is used by many \term{operators} 
as a default destination for output when no specific \term{output} \term{stream}
is explicitly supplied.

%% 21.1.0 9
\Thevalueof{*trace-output*}, called \term{trace output},
is the \term{stream} on which traced functions (see \macref{trace})
and \themacro{time} print their output.

\label Examples::

\code
 (with-output-to-string (*error-output*)
   (warn "this string is sent to *error-output*"))
 \EV "Warning: this string is sent to *error-output*
" ;The exact format of this string is \term{implementation-dependent}.
\medbreak
 (with-input-from-string (*standard-input* "1001")
    (+ 990 (read))) \EV 1991                       
\medbreak
 (progn (setq out (with-output-to-string (*standard-output*)
                     (print "print and format t send things to")
                     (format t "*standard-output* now going to a string")))
        :done)
\EV :DONE
 out
\EV "
\\"print and format t send things to\\" *standard-output* now going to a string"
\medbreak
 (defun fact (n) (if (< n 2) 1 (* n (fact (- n 1)))))
\EV FACT
 (trace fact)
\EV (FACT)
;; Of course, the format of traced output is implementation-dependent.
 (with-output-to-string (*trace-output*)
   (fact 3)) 
\EV "
1 Enter FACT 3
| 2 Enter FACT 2
|   3 Enter FACT 1
|   3 Exit FACT 1
| 2 Exit FACT 2
1 Exit FACT 6"
\endcode

\label See Also::

\varref{*terminal-io*},
\typeref{synonym-stream},
\macref{time},
\macref{trace},
{\chapref\Conditions},
{\chapref\Reader},
{\chapref\Printer}

\label Notes::

\issue{STANDARD-INPUT-INITIAL-BINDING:DEFINED-CONTRACTS}
The intent of the constraints on the initial \term{value} 
of the \term{I/O customization variables} is to ensure that it
is always safe to \term{bind} or \term{assign} such a \term{variable} to 
the \term{value} of another \term{I/O customization variable}, without
unduly restricting \term{implementation} flexibility.
\endissue{STANDARD-INPUT-INITIAL-BINDING:DEFINED-CONTRACTS}

It is common for an \term{implementation} to make
the initial \term{values} of \varref{*debug-io*} and \varref{*query-io*} 
be the \term{same} \term{stream}, 
and to make
the initial \term{values} of \varref{*error-output*} and \varref{*standard-output*} 
be the \term{same} \term{stream}.

The functions \funref{y-or-n-p} and \funref{yes-or-no-p} use \term{query I/O}
for their input and output.

%% 21.1.0 3
%% 22.2.1 1
In the normal \term{Lisp read-eval-print loop},
input is read from \term{standard input}.
Many input functions, including \funref{read} and \funref{read-char},
take a \term{stream} argument that defaults to \term{standard input}.

%% 21.1.0 4
In the normal \term{Lisp read-eval-print loop}, output is sent to \term{standard output}.
Many output functions, including \funref{print} and \funref{write-char}, 
take a \term{stream} argument that defaults to \term{standard output}.

%% 21.1.0 11
A program that wants, for example, to divert output to a file should do so by 
\term{binding} \varref{*standard-output*}; that way error messages sent to
\varref{*error-output*} can still get to the user by going through
\varref{*terminal-io*} (if \varref{*error-output*} is bound to \varref{*terminal-io*}),
which is usually what is desired.

\endcom


%%% ========== *TERMINAL-IO*
\begincom{*terminal-io*}\ftype{Variable}

\label Value Type::

a \term{bidirectional} \term{stream}.

\label Initial Value::

\term{implementation-dependent}, but
it must be an \term{open} \term{stream} 
that is not a  \term{generalized synonym stream}
to an \term{I/O customization variables}
but that might be a \term{generalized synonym stream} to 
the \term{value} of some \term{I/O customization variable}.

\label Description::

%% 21.1.0 8
\Thevalueof{*terminal-io*}, called  \term{terminal I/O}, is ordinarily 
a \term{bidirectional} \term{stream} that connects to the user's console.
Typically, writing to this \term{stream} 
would cause the output to appear
on a display screen, for example, and reading from the \term{stream} would
accept input from a keyboard.  It is intended
that standard input functions such as \funref{read} and \funref{read-char},
when used with this \term{stream}, cause echoing of the input
into the output side of the \term{stream}. The means by which this is
accomplished are \term{implementation-dependent}.

%% 21.1.0 11, clarified by KMP
The effect of changing \thevalueof{*terminal-io*},
either by \term{binding} or \term{assignment},
is \term{implementation-defined}.

\label Examples::

\code
 (progn (prin1 'foo) (prin1 'bar *terminal-io*))
\OUT FOOBAR
\EV BAR
 (with-output-to-string (*standard-output*)
   (prin1 'foo) 
   (prin1 'bar *terminal-io*))
\OUT BAR
\EV "FOO"
\endcode

\label Affected By:\None.

\label See Also::

\varref{*debug-io*},
\varref{*error-output*},
\varref{*query-io*},
\varref{*standard-input*},
\varref{*standard-output*},
\varref{*trace-output*}

\label Notes:\None.

\endcom


%-------------------- Stream Errors --------------------

\begincom{stream-error}\ftype{Condition Type}

\label Class Precedence List::
\typeref{stream-error},
\typeref{error},
\typeref{serious-condition},
\typeref{condition},
\typeref{t}

\label Description::

\Thetype{stream-error}
consists of error conditions that are related to receiving input from
or sending output to a \term{stream}.  
The ``offending stream'' is initialized by
\theinitkeyarg{stream} to \funref{make-condition}, 
and is \term{accessed} by \thefunction{stream-error-stream}.

\label See Also::

\funref{stream-error-stream}

\endcom%{stream-error}\ftype{Condition Type}

%%% ========== STREAM-ERROR-STREAM
\begincom{stream-error-stream}\ftype{Function}

\label Syntax::

\DefunWithValues stream-error-stream {condition} {stream}

\label Arguments and Values::

\param{condition}---a \term{condition} \oftype{stream-error}.

\param{stream}---a \term{stream}.

\label Description::

Returns the offending \term{stream} of a \term{condition} \oftype{stream-error}.

\label Examples::
\code
 (with-input-from-string (s "(FOO")
   (handler-case (read s)
     (end-of-file (c)
       (format nil "~&End of file on ~S." (stream-error-stream c)))))
"End of file on #<String Stream>."
\endcode
% )

\label Side Effects:\None.

\label Affected By:\None.

\label Exceptional Situations:\None.

\label See Also::

\typeref{stream-error},
{\secref\Conditions}

\label Notes:\None.

%% This shouldn't be needed.
%It is an error to use \macref{setf} with \funref{stream-error-stream}.

\endcom

\begincom{end-of-file}\ftype{Condition Type}

\label Class Precedence List::
\typeref{end-of-file},
\typeref{stream-error},
\typeref{error},
\typeref{serious-condition},
\typeref{condition},
\typeref{t}

\label Description::

\Thetype{end-of-file} consists of
error conditions related to read operations that are done on
\term{streams} that have no more data.

\label See Also::

\funref{stream-error-stream}

\endcom%{end-of-file}\ftype{Condition Type}

% -*- Mode: TeX -*-

% Arrays
%  Vectors
%   Bit Vectors
%   Strings

%-------------------- Array types --------------------

%% 2.5.0 1
\begincom{array}\ftype{System Class}

\label Class Precedence List::
\typeref{array},
\typeref{t}

\label Description::

An \term{array} contains \term{objects} arranged according to a
Cartesian coordinate system.
%% 2.5.1 4
% {Symbolics rewrote the following}
An \term{array} provides mappings from a set of
\issue{ARRAY-DIMENSION-IMPLICATIONS:ALL-FIXNUM}
\term{fixnums}
\endissue{ARRAY-DIMENSION-IMPLICATIONS:ALL-FIXNUM}
$\left\{i\sub{0},i\sub{1},\dots,i\sub{r-1}\right\}$ to corresponding \term{elements}
of the \term{array}, 
where $0 \le i\sub{j} < d\sub{j}$,
$r$ is the rank of the array, and $d\sub{j}$ is the size of \term{dimension} $j$ of
the array.

When an \term{array} is created, the program requesting its creation may
declare that all \term{elements} are of a particular \term{type}, 
called the \term{expressed array element type}.
The implementation is permitted to \term{upgrade} this type in order to 
produce the \term{actual array element type},
which is the \term{element type} for the \term{array} is actually \term{specialized}.
\Seefun{upgraded-array-element-type}.

\label Compound Type Specifier Kind::

Specializing.

\label Compound Type Specifier Syntax::

\Deftype{array}{\ttbrac{\curly{element-type | \misc{*}} \brac{dimension-spec}}}

\auxbnf{dimension-spec}{rank | \misc{*} | \paren{\star{\curly{dimension | \misc{*}}}}}

\label Compound Type Specifier Arguments::

\param{dimension}---a \term{valid array dimension}.

\param{element-type}---a \term{type specifier}.

\issue{ARRAY-DIMENSION-IMPLICATIONS:ALL-FIXNUM}
\param{rank}---a non-negative \term{fixnum}.
\endissue{ARRAY-DIMENSION-IMPLICATIONS:ALL-FIXNUM}

\label Compound Type Specifier Description::

%% 4.5.0 6
%% 4.5.0 7

This denotes the set of \term{arrays} whose
  \term{element type},  \term{rank},  and \term{dimensions}
match any given
  \param{element-type}, \param{rank}, and \param{dimensions}.
Specifically:

If \param{element-type} is the \term{symbol} \misc{*},
\term{arrays} are not excluded on the basis of their \term{element type}.
Otherwise, only those \param{arrays} are included whose \term{actual array element type}
\issue{ARRAY-TYPE-ELEMENT-TYPE-SEMANTICS:UNIFY-UPGRADING}
is the result of \term{upgrading} \param{element-type};
\seesection\ArrayUpgrading.
%The following will be left out:
%\param{element-type}
%% The following seems redundant. -kmp 20-Oct-91
% {\tt (array \param{type-specifier})}
% refers only to those \term{arrays} 
% that can result from giving \param{type-specifier} as the
% \kwd{element-type} argument to \funref{make-array}.  
\endissue{ARRAY-TYPE-ELEMENT-TYPE-SEMANTICS:UNIFY-UPGRADING}

If the \param{dimension-spec} is a \param{rank},
the set includes only those \param{arrays} having that \term{rank}.
If the \param{dimension-spec} is a \term{list} of \param{dimensions},
the set includes only those \param{arrays} having a \term{rank}
given by the \term{length} of the \param{dimensions},
and having the indicated \param{dimensions}; 
in this case, \misc{*} matches any value for the corresponding \term{dimension}.
If the \param{dimension-spec} is the \term{symbol} \misc{*},
the set is not restricted on the basis of \term{rank} or \term{dimension}.

\issue{ARRAY-TYPE-ELEMENT-TYPE-SEMANTICS:UNIFY-UPGRADING}
%The following will be left out:
%
%For declaration purposes, this \term{type} encompasses those 
%\term{arrays}
%that can result by supplying \param{element-type} as the element type
%to \thefunction{make-array}; this might be different
%from what the \term{type} means for discrimination purposes.
%For example:
%
%\code
% (array integer 3)       ;Three-dimensional arrays of integers
% (array integer (* * *)) ;Three-dimensional arrays of integers
% (array * (4 5 6))       ;4-by-5-by-6 arrays
% (array character (3 *)) ;Two-dimensional arrays of characters that have 
%                         ;three rows
% (array short-float \empty)   ;Zero-rank arrays of short-floats
%\endcode
\endissue{ARRAY-TYPE-ELEMENT-TYPE-SEMANTICS:UNIFY-UPGRADING}

\label See Also::

\varref{*print-array*},
\funref{aref},
\funref{make-array},
\typeref{vector},
{\secref\SharpsignA},
{\secref\PrintingOtherArrays}

\label Notes::

Note that the type {\tt (array t)}     
is a proper \term{subtype} of the type {\tt (array *)}.
The reason is that the type {\tt (array t)} is the set of \term{arrays} 
that can
hold any \term{object} (the \term{elements} are \oftype{t},  which includes
all \term{objects}).
On the other hand, the type {\tt (array *)}
is the set of all \term{arrays} whatsoever, including for example
\term{arrays} that can hold only \term{characters}. 
%% possible issue here that: ARRAY-TYPE-ELEMENT-TYPE-SEMANTICS says that
%%this example only works in implementations that have specialized arrays for
%%elements of type character, not required in CL. But char committee may
%%require them.
The type {\tt (array character)} 
is not a \term{subtype} of the type {\tt (array t)}; 
the two sets                                              
are \term{disjoint} because the type {\tt (array character)} is not the
set of all \term{arrays} that can hold 
\term{characters}, but rather the set of
\term{arrays} 
that are specialized to hold precisely \term{characters} and no
other \term{objects}. 

%The following expression cannot be used to determine if
%array \f{foo} can hold a \term{character}:
%
%\code
% (typep foo '(array character))
%\endcode
%The following expression can be used to determine if
%array \f{foo} can hold a \term{character}:
%
%\code
% (subtypep 'character (array-element-type foo))
%\endcode


\endcom%{array}\ftype{System Class}
%% 2.5.0 9
\begincom{simple-array}\ftype{Type}

\label Supertypes:: 

\typeref{simple-array},
%% 2.15.0 20
\typeref{array},
\typeref{t}

\label Description::

%Per Symbolics comments, this is reworded to clarify that this only defines
%things that might be simple, not what must be:
%An \term{array}
%that is not displaced to another \term{array}, has no \term{fill pointer},
%and is not to have its size adjusted dynamically after
%creation is a \term{simple array}.
The \term{type} of an \term{array} that is not displaced 
to another \term{array}, has no \term{fill pointer}, and is
% \reviewer{Barrett: Reexamine in light of ADJUST-ARRAY-NOT-ADJUSTABLE--specifically,
%  all arrays are now valid to adjust-array--they just don't all get changed by side-effect.}
not
\term{expressly adjustable} is a \subtypeof{simple-array}.
The concept of a \term{simple array}
exists to allow the implementation to use a specialized representation
and to allow the user to declare that certain values will always be
\term{simple arrays}. 

%% 2.15.0 21
The \term{types} \typeref{simple-vector},
		 \typeref{simple-string},
             and \typeref{simple-bit-vector}
are \term{disjoint} \subtypesof{simple-array}, 
for they respectively mean \f{(simple-array t (*))},
		           the union of all \f{(simple-array \i{c} (*))} 
 			    for any \i{c} being a \subtypeof{character},
			and \f{(simple-array bit (*))}.

\label Compound Type Specifier Kind::

Specializing.

%% 4.5.0 8
\label Compound Type Specifier Syntax::

\Deftype{simple-array}{\ttbrac{\curly{element-type | \misc{*}} \brac{dimension-spec}}}

\auxbnf{dimension-spec}{rank | \misc{*} | \paren{\star{\curly{dimension | \misc{*}}}}}

\label Compound Type Specifier Arguments::

\param{dimension}---a \term{valid array dimension}.

\param{element-type}---a \term{type specifier}.

\issue{ARRAY-DIMENSION-IMPLICATIONS:ALL-FIXNUM}
\param{rank}---a non-negative \term{fixnum}.
\endissue{ARRAY-DIMENSION-IMPLICATIONS:ALL-FIXNUM}

\label Compound Type Specifier Description::
                   
This \term{compound type specifier} is treated exactly as the corresponding
\term{compound type specifier} for \term{type} \typeref{array} would be treated,
except that the set is further constrained to include only \term{simple arrays}.

% This denotes the set of \term{simple arrays} whose elements are all of \term{type}
% \issue{ARRAY-TYPE-ELEMENT-TYPE-SEMANTICS:UNIFY-UPGRADING}
% {\penalty-50}\f{(upgraded-array-element-type \param{element-type})}
% %The following will be left out:
% %\param{element-type}
% \endissue{ARRAY-TYPE-ELEMENT-TYPE-SEMANTICS:UNIFY-UPGRADING}
% and whose dimensions are \param{dimensions}.
% \param{Element-type} must be a valid \term{type specifier} or \misc{*}.
% \issue{ARRAY-DIMENSION-LIMIT-IMPLICATIONS:ALL-FIXNUM}
% \param{Dimensions} must be a non-negative \term{fixnum},
% \endissue{ARRAY-DIMENSION-LIMIT-IMPLICATIONS:ALL-FIXNUM}
% which is the number
% of dimensions, or a \term{list} of \term{valid array dimensions}
% representing the length of each dimension (any dimension
% can be \misc{*} instead), or it can be \misc{*}.

\label Notes::

It is \term{implementation-dependent} 
whether \term{displaced arrays},
        \term{vectors} with \term{fill pointers},
     or arrays that are \term{actually adjustable}
  are \term{simple arrays}.

\issue{ARRAY-TYPE-ELEMENT-TYPE-SEMANTICS:UNIFY-UPGRADING}
{\tt (simple-array *)} refers to all \term{simple arrays} 
regardless of element type, {\tt (simple-array \param{type-specifier})}
refers only to those \term{simple arrays}
that can result from giving \param{type-specifier} as the
\kwd{element-type} argument to \funref{make-array}.  
\endissue{ARRAY-TYPE-ELEMENT-TYPE-SEMANTICS:UNIFY-UPGRADING}

\endcom%{simple-array}\ftype{Type}
%% 2.5.1 1
\begincom{vector}\ftype{System Class}

\label Class Precedence List::
\typeref{vector},
\typeref{array},
\typeref{sequence},
\typeref{t}

\label Description::

Any one-dimensional \term{array} is a \term{vector}.

%% 2.15.0 19
\Thetype{vector} is a \subtypeof{array}; 
for all \term{types} \f{x}, {\tt (vector x)} is the same as {\tt (array x (*))}.

%% 2.15.0 18
The \term{type} {\tt (vector t)}, \thetype{string}, and \thetype{bit-vector}
are \term{disjoint} \subtypesof{vector}.

\label Compound Type Specifier Kind::

Specializing.

\label Compound Type Specifier Syntax::

%% 4.5.0 9
\Deftype{vector}{\ttbrac{\curly{element-type | \misc{*}} \brac{\curly{size | \misc{*}}}}}

\label Compound Type Specifier Arguments::

\issue{ARRAY-DIMENSION-IMPLICATIONS:ALL-FIXNUM}
\param{size}---a non-negative \term{fixnum}.
\endissue{ARRAY-DIMENSION-IMPLICATIONS:ALL-FIXNUM}

\param{element-type}---a \term{type specifier}.

\label Compound Type Specifier Description::

This denotes the set of specialized \term{vectors}
whose \term{element type} and \param{dimension} match the specified values.
Specifically:

If \param{element-type} is the \term{symbol} \misc{*},
\term{vectors} are not excluded on the basis of their \term{element type}.
Otherwise, only those \param{vectors} are included whose \term{actual array element type}
\issue{ARRAY-TYPE-ELEMENT-TYPE-SEMANTICS:UNIFY-UPGRADING}
is the result of \term{upgrading} \param{element-type};
\seesection\ArrayUpgrading.
\endissue{ARRAY-TYPE-ELEMENT-TYPE-SEMANTICS:UNIFY-UPGRADING}

If a \param{size} is specified,
the set includes only those \param{vectors} whose only \term{dimension}
is \param{size}.
If the \term{symbol} \misc{*} is specified instead of a \param{size},
the set is not restricted on the basis of \term{dimension}.

\label See Also::

{\secref\RequiredSpecializedArrays},
{\secref\SharpsignLeftParen},
{\secref\PrintingOtherVectors},
{\secref\SharpsignA}

\label Notes::

The \term{type} \f{(vector \param{e} \param{s})} 
is equivalent to the \term{type} \f{(array \param{e} (\param{s}))}.

The type \f{(vector bit)} has the name \typeref{bit-vector}.

The union of all \term{types} \f{(vector $C$)}, 
where $C$ is any \term{subtype} of \typeref{character},
has the name \typeref{string}.
%Every implementation of \clisp\ must provide distinct representations for
%these as distinct specialized \term{types}.

\issue{ARRAY-TYPE-ELEMENT-TYPE-SEMANTICS:UNIFY-UPGRADING}
{\tt (vector *)} refers to all \term{vectors} 
regardless of element type, {\tt (vector \param{type-specifier})}
refers only to those \term{vectors} 
that can result from giving \param{type-specifier} as the
\kwd{element-type} argument to \funref{make-array}.  
\endissue{ARRAY-TYPE-ELEMENT-TYPE-SEMANTICS:UNIFY-UPGRADING}

\endcom%{vector}\ftype{System Class}
\begincom{simple-vector}\ftype{Type}

\label Supertypes::

\typeref{simple-vector},
\typeref{vector},
\typeref{simple-array},
\typeref{array},
\typeref{sequence},
\typeref{t}

\label Description::

% This sentence replaced per Symbolics comments:
% A \term{vector} that is not displaced to another \term{array}, has no
% \term{fill pointer}, is able to hold elements of any \term{type},
% and is not to have its size adjusted dynamically after creation is a
% \term{simple vector}.
The \term{type} of a \term{vector} that is not displaced to another
\term{array}, has no \term{fill pointer}, is not 
%to have its size adjusted dynamically after creation, 
\term{expressly adjustable}
and is able to hold 
elements of any \term{type} is a \subtypeof{simple-vector}.

%% 2.15.0 22                       
\Thetype{simple-vector} is a \subtypeof{vector},
and is a \term{subtype} of \term{type} \f{(vector t)}.

\label Compound Type Specifier Kind::

Specializing.

\label Compound Type Specifier Syntax::

\Deftype{simple-vector}{\ttbrac{size}}

\label Compound Type Specifier Arguments::

\issue{ARRAY-DIMENSION-IMPLICATIONS:ALL-FIXNUM}
\param{size}---a non-negative \term{fixnum},
	    or the \term{symbol} \misc{*}.
  \Default{the \term{symbol} \misc{*}}
\endissue{ARRAY-DIMENSION-IMPLICATIONS:ALL-FIXNUM}

\label Compound Type Specifier Description::

%% 4.5.0 10                            
%\itemitem{\tt (simple-vector \param{size})}

This is the same as {\tt (simple-array t (\param{size}))}.
%This is the same as {\tt (and (vector t \param{size}) simple-array)}.

% This is the same                
% as the type \f{(vector t \param{size})}
% except that \typeref{simple-vector} also specifies
% that its members are \term{simple arrays}.

\endcom%{simple-vector}\ftype{Type}
\begincom{bit-vector}\ftype{System Class}

\label Class Precedence List::
\typeref{bit-vector},
\typeref{vector},
\typeref{array},
\typeref{sequence},
\typeref{t}

\label Description::

A \term{bit vector} is a \term{vector} the \term{element type} of which is \term{bit}.

%% 2.15.0 17
\Thetype{bit-vector} is a \subtypeof{vector}, 
for \typeref{bit-vector} means \f{(vector bit)}.

\label Compound Type Specifier Kind::

Abbreviating.

\label Compound Type Specifier Syntax::

%% 4.6.0 12
\Deftype{bit-vector}{\ttbrac{size}}

\label Compound Type Specifier Arguments::

\issue{ARRAY-DIMENSION-IMPLICATIONS:ALL-FIXNUM}
\param{size}---a non-negative \term{fixnum},
	    or the \term{symbol} \misc{*}.
\endissue{ARRAY-DIMENSION-IMPLICATIONS:ALL-FIXNUM}

\label Compound Type Specifier Description::
                                                        
This denotes the same \term{type} as the \term{type} {\tt (array bit (\param{size}))};
that is, the set of \term{bit vectors} of size \param{size}.

\label See Also::

{\secref\SharpsignStar},
{\secref\PrintingBitVectors},
{\secref\RequiredSpecializedArrays}

\endcom%{bit-vector}\ftype{System Class}
\begincom{simple-bit-vector}\ftype{Type}

\label Supertypes:: 

\typeref{simple-bit-vector},
%% 2.15.0 25
\typeref{bit-vector},
\typeref{vector},
\typeref{simple-array},
\typeref{array},
\typeref{sequence},
\typeref{t}

\label Description::

% Reworded per Symbolics comments:
%   A \term{bit vector} that is not displaced to another
% \term{array}, has no \term{fill pointer}, and is not to have its
% size adjusted dynamically after creation is a \term{simple bit vector}.
The \term{type} of a \term{bit vector} that is not displaced
to another \term{array}, has no \term{fill pointer}, and is 
not
%to have its size adjusted dynamically after creation 
\term{expressly adjustable}
is a
\subtypeof{simple-bit-vector}.

\label Compound Type Specifier Kind::

Abbreviating.

\label Compound Type Specifier Syntax::

%% 4.6.0 13
\Deftype{simple-bit-vector}{\ttbrac{size}}

\label Compound Type Specifier Arguments::

\issue{ARRAY-DIMENSION-IMPLICATIONS:ALL-FIXNUM}
\param{size}---a non-negative \term{fixnum},
	    or the \term{symbol} \misc{*}.
  \Default{the \term{symbol} \misc{*}}
\endissue{ARRAY-DIMENSION-IMPLICATIONS:ALL-FIXNUM}

\label Compound Type Specifier Description::

This denotes the same type as the \term{type}
\f{(simple-array bit (\param{size}))}; 
that is, the set of \term{simple bit vectors} of size \param{size}.

\endcom%{simple-bit-vector}\ftype{Type}


%-------------------- Arrays --------------------

%%% ========== MAKE-ARRAY
\begincom{make-array}\ftype{Function}

\label Syntax::

\DefunWithValuesNewline make-array
			{dimensions {\key} \vtop{\hbox{element-type}
                                                 \hbox{initial-element}
                                                 \hbox{initial-contents}
                                                 \hbox{adjustable}
                                                 \hbox{fill-pointer}
                                                 \hbox{displaced-to}
                                                 \hbox{displaced-index-offset}}}
		 	{new-array}

\label Arguments and Values::

%% 17.1.0 3
\param{dimensions}---a \term{designator} for a \term{list} of \term{valid array dimensions}.

%% 17.1.0 6
\param{element-type}---a \term{type specifier}. 
 \Default{\typeref{t}}

\param{initial-element}---an \term{object}.

\param{initial-contents}---an \term{object}.
%Per Barmar, remove (or sequence vector) because 0-dimension case doesn't care. -kmp 24-Jul-91

\param{adjustable}---a \term{generalized boolean}.
 \Default{\nil}

%!!! Is there a designator for this?
\param{fill-pointer}---a \term{valid fill pointer} for the \term{array} to be created,
		       or \t\ or \nil.
 \Default{\nil}

%% 17.1.0 11
\param{displaced-to}---an \term{array} or \nil.
 \Default{\nil}
 This option must not be supplied if either \param{initial-element}
 or \param{initial-contents} is supplied.

%% 17.1.0 12
\param{displaced-index-offset}---a \term{valid array row-major index} 
 for \param{displaced-to}. \Default{\f{0}}
 This option must not be supplied unless a \term{non-nil} \param{displaced-to} is supplied.

\param{new-array}---an \term{array}.

\label Description::

Creates and returns an \term{array} constructed of the most \term{specialized}
\term{type} that can accommodate elements of \term{type} given by \param{element-type}.
If \param{dimensions} is \nil\ then a zero-dimensional \term{array} is created.

\param{Dimensions} represents the dimensionality of the new \term{array}.

\param{element-type} indicates the \term{type} of the elements intended to be stored
in the \param{new-array}.  The \param{new-array} can actually store any \term{objects}
of the \term{type} which results from \term{upgrading} \param{element-type};
\seesection\ArrayUpgrading.
                                                                       
%% 17.1.0 7
If \param{initial-element} is supplied, 
it is used to initialize each \term{element} of \param{new-array}. 
If \param{initial-element} is supplied,
it must be of the \term{type} given by \param{element-type}.
\param{initial-element} cannot be supplied if either the \kwd{initial-contents} option
is supplied or \param{displaced-to} is \term{non-nil}.
If \param{initial-element} is not supplied,
\issue{UNINITIALIZED-ELEMENTS:CONSEQUENCES-UNDEFINED}
%% This issue finally did pass at the March-93 meeting. -kmp 5-May-93
% the initial values of the \term{array} \term{elements} are 
% %%Rewritten to conform to the fact that issue UNINITIALIZED-ELEMENTS failed
% %%to clarify this in the other direction because consensus was that this was well-defined
% %%already. -kmp 15-Jan-92
% %undefined 
% \term{implementation-dependent} 
the consequences of later reading an uninitialized \term{element} of \param{new-array}
are undefined
\endissue{UNINITIALIZED-ELEMENTS:CONSEQUENCES-UNDEFINED}
unless either \param{initial-contents} is supplied 
or \param{displaced-to} is \term{non-nil}.


%% 17.1.0 8
\param{initial-contents} is used to initialize the contents of \term{array}.
For example:

\code
 (make-array '(4 2 3) :initial-contents
             '(((a b c) (1 2 3))
              ((d e f) (3 1 2))
              ((g h i) (2 3 1))
              ((j k l) (0 0 0))))
\endcode

%!!! Is ADJUST-ARRAY-NOT-ADJUSTABLE:IMPLICIT-COPY fully merged?

\param{initial-contents} is composed of a nested structure of \term{sequences}. 
The numbers of levels in the structure must equal the rank of \term{array}.
Each leaf of the nested structure must be of the \term{type} given by \param{element-type}. 
If \term{array} is zero-dimensional, then \param{initial-contents} specifies the single
\term{element}.  Otherwise, \param{initial-contents} must be a \term{sequence}
whose length is equal to the first dimension; each element must be a nested 
structure for an \term{array} whose dimensions are the remaining dimensions, 
and so on.
\param{Initial-contents} cannot be supplied if either 
\param{initial-element} is supplied
or \param{displaced-to} is \term{non-nil}.
If \param{initial-contents} is not supplied,
\issue{UNINITIALIZED-ELEMENTS:CONSEQUENCES-UNDEFINED}
%% This issue finally did pass at the March-93 meeting. -kmp 5-May-93
% the initial values of the \term{array} \term{elements} are 
% %%Rewritten to conform to the fact that issue UNINITIALIZED-ELEMENTS failed
% %%to clarify this in the other direction because consensus was that this was well-defined
% %%already. -kmp 15-Jan-92
% %undefined
% \term{implementation-dependent}
the consequences of later reading an uninitialized \term{element} of \param{new-array}
are undefined
\endissue{UNINITIALIZED-ELEMENTS:CONSEQUENCES-UNDEFINED}
unless either \param{initial-element} is supplied
or \param{displaced-to} is \term{non-nil}.

%% KMP's rewrite:
% %% 17.1.0 9
% If \param{adjustable} is \term{non-nil},
% the array is both \term{expressly adjustable} 
% 	      and \term{actually adjustable};
% otherwise, the array is not \term{expressly adjustable} 
%        and it is \term{implementation-dependent} whether 
% 	    the array is \term{actually adjustable}.
%% To which Barrett suggested:
% If \param{adjustable} is \term{non-nil},
% the array is \term{expressly adjustable};
% otherwise, it is \term{implementation-dependent} whether 
%   the array is \term{actually adjustable}.
%% We have compromised on:
%% 17.1.0 9
If \param{adjustable} is \term{non-nil},
the array is \term{expressly adjustable} 
	      (and so \term{actually adjustable});
otherwise, the array is not \term{expressly adjustable} 
       (and it is \term{implementation-dependent} whether 
	    the array is \term{actually adjustable}).
% ...
% it is possible to alter the \term{array}'s size dynamically after it is created.
%\issue{ADJUST-ARRAY-NOT-ADJUSTABLE:DONKEY}
%and \funref{adjustable-array-p} will be true.
%If \param{adjustable} is not supplied or \nil, 
%\term{array} is not required to be
%adjustable, but a non-adjustable \term{array} is not guaranteed.
%\endissue{ADJUST-ARRAY-NOT-ADJUSTABLE:DONKEY}

%% 17.1.0 10
If \param{fill-pointer} is \term{non-nil},
the \term{array} must be one-dimensional;
that is, the \term{array} must be a \term{vector}.
If \param{fill-pointer} is \t,
the length of the \term{vector} is used to initialize the \term{fill pointer}.
If \param{fill-pointer} is an \term{integer},
it becomes the initial \term{fill pointer} for the \term{vector}.

If \param{displaced-to} is \term{non-nil},
\funref{make-array} will create a \term{displaced array} 
and \param{displaced-to} is the \term{target} of that \term{displaced array}. 
In that case, the consequences are undefined if the \term{actual array element type} of 
\param{displaced-to} is not \term{type equivalent} to the \term{actual array element type}
of the \term{array} being created.
If \param{displaced-to} is \nil, the \term{array} is not a \term{displaced array}.

%!!! should "index offset" or "displaced index offset" be a formal term?
The \param{displaced-index-offset} is made to be the index offset of the \term{array}.
%% 17.1.0 13
When an array A is given as
\thekeyarg{displaced-to} to \funref{make-array} 
when creating array B,
then array B is said to be displaced to array A.  The
total number of elements in an \term{array}, 
called the total size of the \term{array},
is calculated as the product of all the dimensions.
It is required that the total size of A be no smaller than the sum
of the total size of B plus the offset \f{n} supplied by
the \param{displaced-index-offset}.
The effect of displacing is that array B does not have any
elements of its own, but instead maps \term{accesses} to itself into
\term{accesses} to array A.  The mapping treats both \term{arrays} as if they
were one-dimensional by taking the elements in row-major order,
and then maps an \term{access} to element \f{k} of array B to an \term{access} to element
\f{k}+\f{n} of array A.

%% 17.1.0 14
If \funref{make-array} is called with \param{adjustable}, \param{fill-pointer},
and \param{displaced-to} each \nil, 
then the result is a \term{simple array}.
\issue{ADJUST-ARRAY-NOT-ADJUSTABLE:IMPLICIT-COPY}
If \funref{make-array} is called with one or more of \param{adjustable},
\param{fill-pointer}, or \param{displaced-to} being \term{true}, whether the
resulting \term{array} is a \term{simple array} is \term{implementation-dependent}.
\endissue{ADJUST-ARRAY-NOT-ADJUSTABLE:IMPLICIT-COPY}

%% 17.1.0 13
  When an array A is given as \thekeyarg{displaced-to} to
  \funref{make-array} when creating array B, then array B is said to
  be displaced to array A.  The total number of elements in an \term{array}, 
  called the total size of the \term{array}, is calculated as the product
  of all the dimensions.
The consequences are unspecified if
the total size of A is smaller than the sum
of the total size of B plus the offset \f{n} supplied by
the \param{displaced-index-offset}.
The effect of displacing is that array B does not have any
elements of its own, but instead maps \term{accesses} to itself into
\term{accesses} to array A.  The mapping treats both \term{arrays} as if they
were one-dimensional by taking the elements in row-major order,
and then maps an \term{access} to element \f{k} of array B to an \term{access} 
to \term{element} \f{k}+\f{n} of array A.

%% Redundant with previous paragraph.
% When \kwd{displaced-to} is supplied to \funref{make-array},
% the \param{displaced-index-offset} is made to be the
% index offset of the \term{array}.
%% Redundant with Arguments and Values above.
% If the \param{displaced-index-offset} is not supplied,
% the index offset is zero.

\label Examples::
%% 17.1.0 15
\code

 (make-array 5) ;; Creates a one-dimensional array of five elements.
 (make-array '(3 4) :element-type '(mod 16)) ;; Creates a 
                ;;two-dimensional array, 3 by 4, with four-bit elements.
 (make-array 5 :element-type 'single-float) ;; Creates an array of single-floats.
\endcode

\code
 (make-array nil :initial-element nil) \EV #0ANIL
 (make-array 4 :initial-element nil) \EV #(NIL NIL NIL NIL)
 (make-array '(2 4) 
              :element-type '(unsigned-byte 2) 
              :initial-contents '((0 1 2 3) (3 2 1 0)))
\EV #2A((0 1 2 3) (3 2 1 0))
 (make-array 6
              :element-type 'character 
              :initial-element #\\a 
              :fill-pointer 3) \EV "aaa"
\endcode

The following is an example of making a \term{displaced array}.

\code
 (setq a (make-array '(4 3))) 
\EV #<ARRAY 4x3 simple 32546632>
 (dotimes (i 4)
   (dotimes (j 3)
     (setf (aref a i j) (list i 'x j '= (* i j)))))
\EV NIL
 (setq b (make-array 8 :displaced-to a
                       :displaced-index-offset 2))
\EV #<ARRAY 8 indirect 32550757>
 (dotimes (i 8)
   (print (list i (aref b i))))
\OUT (0 (0 X 2 = 0)) 
\OUT (1 (1 X 0 = 0)) 
\OUT (2 (1 X 1 = 1)) 
\OUT (3 (1 X 2 = 2)) 
\OUT (4 (2 X 0 = 0)) 
\OUT (5 (2 X 1 = 2)) 
\OUT (6 (2 X 2 = 4)) 
\OUT (7 (3 X 0 = 0)) 
\EV NIL
\endcode
The last example depends on the fact that \term{arrays} are, in effect,
stored in row-major order. 

\code
 (setq a1 (make-array 50))
\EV #<ARRAY 50 simple 32562043>
 (setq b1 (make-array 20 :displaced-to a1 :displaced-index-offset 10))
\EV #<ARRAY 20 indirect 32563346>
 (length b1) \EV 20

 (setq a2 (make-array 50 :fill-pointer 10))
\EV #<ARRAY 50 fill-pointer 10 46100216>
 (setq b2 (make-array 20 :displaced-to a2 :displaced-index-offset 10))
\EV #<ARRAY 20 indirect 46104010>
 (length a2) \EV 10
 (length b2) \EV 20

 (setq a3 (make-array 50 :fill-pointer 10))
\EV #<ARRAY 50 fill-pointer 10 46105663>
 (setq b3 (make-array 20 :displaced-to a3 :displaced-index-offset 10
                         :fill-pointer 5))
\EV #<ARRAY 20 indirect, fill-pointer 5 46107432>
 (length a3) \EV 10
 (length b3) \EV 5
\endcode
  

\label Affected By:\None.

\label Exceptional Situations:\None.

\label See Also::

\funref{adjustable-array-p},
\funref{aref},
\funref{arrayp},
\funref{array-element-type},
\conref{array-rank-limit},
\conref{array-dimension-limit},
\funref{fill-pointer},
\funref{upgraded-array-element-type}

\label Notes::

\issue{ADJUST-ARRAY-NOT-ADJUSTABLE:IMPLICIT-COPY}
There is no specified way to create an \term{array} 
for which \funref{adjustable-array-p} definitely
returns \term{false}.
There is no specified way to create an \term{array} 
that is not a \term{simple array}.
\endissue{ADJUST-ARRAY-NOT-ADJUSTABLE:IMPLICIT-COPY}
 
\endcom


%%% ========== ADJUST-ARRAY
\begincom{adjust-array}\ftype{Function}

%!!! Barmar's review comments of this were made on the basis of document that
%    did not include various cleanups.  This should be re-reviewed carefully later.
%    -kmp 17-Dec-90

\label Syntax::

\DefunWithValuesNewline adjust-array
			{array new-dimensions {\key} \vtop{\hbox{element-type}
                                                           \hbox{initial-element}
                                                           \hbox{initial-contents}
                                                           \hbox{fill-pointer}
                                                           \hbox{displaced-to}
                                                           \hbox{displaced-index-offset}}}
			{adjusted-array}

\label Arguments and Values:: 
                                  
\param{array}---an \term{array}.

\param{new-dimensions}---a \term{valid array dimension} 
   		      or a \term{list} of \term{valid array dimensions}.

%% 17.6.0 6
\param{element-type}---a \term{type specifier}.
%!!! Not sure how to express the default.  Mail sent to Quinquevirate asking for thoughts.

%% 17.6.0 7
\param{initial-element}---an \term{object}.
\issue{ADJUST-ARRAY-DISPLACED-TO-CONSTRAINTS:ALLOW-NIL}
  \param{Initial-element} must not be supplied if either 
  \param{initial-contents} or \param{displaced-to} is supplied.
\endissue{ADJUST-ARRAY-DISPLACED-TO-CONSTRAINTS:ALLOW-NIL}
%!!! How to express this default??

%!!! "array contents designator" ? -kmp 14-May-91
\param{initial-contents}---an \term{object}.
 If \term{array} has rank greater than zero, then \param{initial-contents}
 is composed of nested \term{sequences}, the depth of which must equal
 the rank of \param{array}.  Otherwise, \term{array} is zero-dimensional and
 \param{initial-contents} supplies the single element.
\issue{ADJUST-ARRAY-DISPLACED-TO-CONSTRAINTS:ALLOW-NIL}
 \param{initial-contents} must not be supplied if either 
 \param{initial-element} or \param{displaced-to} is given.
\endissue{ADJUST-ARRAY-DISPLACED-TO-CONSTRAINTS:ALLOW-NIL}

%% 17.6.0 8
\param{fill-pointer}---a \term{valid fill pointer} for the
 \term{array} to be created, or \t, or \nil.
 \Default{\nil}

\param{displaced-to}---an \term{array} or \nil.
 \param{initial-elements} and \param{initial-contents} must not be supplied
 if \param{displaced-to} is supplied.
%!!! Default?

\issue{ARRAY-DIMENSION-LIMIT-IMPLICATIONS:ALL-FIXNUM}
\param{displaced-index-offset}---an \objectoftype{(fixnum 0 \i{n})} 
 where \i{n} is {\tt (array-total-size \param{displaced-to})}.
\issue{ADJUST-ARRAY-DISPLACED-TO-CONSTRAINTS:ALLOW-NIL}
  \param{displaced-index-offset} may be supplied only if \param{displaced-to} is supplied.
\endissue{ADJUST-ARRAY-DISPLACED-TO-CONSTRAINTS:ALLOW-NIL}
%!!! Default?
\endissue{ARRAY-DIMENSION-LIMIT-IMPLICATIONS:ALL-FIXNUM}

\param{adjusted-array}---an \term{array}.

\label Description::

%% 17.6.0 3
\funref{adjust-array} changes the dimensions or elements of \param{array}.
The result is an \term{array} of the same \term{type} and rank as \param{array},
%% 17.6.0 4
%% 17.6.0 9
that is either the modified \param{array},
or a newly created \term{array} to which
\param{array} can be displaced, and that has 
the given \param{new-dimensions}.

%% Sandra points out that this is implied by the definition of "valid array dimension".
% The total number of elements that \param{array} 
% can contain (as given by the
% product of all \param{new-dimensions}) must be less than 
% \conref{array-total-size-limit}.                         

\param{New-dimensions} specify the size of each \term{dimension} of \param{array}. 

\param{Element-type} specifies the \term{type} of the \term{elements}
of the resulting \term{array}.  If \param{element-type} is supplied,
% % the consequences are unspecified if it is not a \term{type}
% % %What does "compatible" mean here?? -kmp 29-Apr-91
% % %Sandra's curious, too. -kmp 14-Feb-92
% % that is compatible with {\tt (array-element-type \param{array})}.
% %% Sandra's suggested rewrite:
% the consequences are unspecified if
% \f{(upgraded-array-element-type \param{element-type})}
% is not the same as {\tt (array-element-type \param{array})}.
%% KMP's rewrite
the consequences are unspecified if
the \term{upgraded array element type} of \param{element-type}
is not the same as the \term{actual array element type} of \param{array}.
%% Sandra thinks this is redundant.
% If the resulting \term{array} is a displaced \term{array},
% the consequences are unspecified if \param{element-type}
% is not compatible with the \term{element type} of the \term{array} 
% that the resulting \term{array} shares elements with.
%% Redundant with Arguments and Values.
%If \kwd{element-type} is not supplied, the elements are of type \t.

If \param{initial-contents} is supplied, it is treated as for
\funref{make-array}.  In this case none of the original contents of
\param{array} appears in the resulting \term{array}.

\issue{ADJUST-ARRAY-FILL-POINTER}
If \param{fill-pointer} is an \term{integer},
it becomes the \term{fill pointer} for the resulting \term{array}.
If \param{fill-pointer} is the symbol \t,
it indicates that the size of the resulting \term{array} 
should be used as the \term{fill pointer}.
If \param{fill-pointer} is \nil,
it indicates that the \term{fill pointer} should be left as it is.
\endissue{ADJUST-ARRAY-FILL-POINTER}

If \param{displaced-to}
%is supplied and
\term{non-nil}, a \term{displaced array}
is created. The resulting \term{array} shares its contents with the \term{array} given by
\param{displaced-to}.
The resulting \term{array} cannot contain more elements than the \term{array}
it is displaced to.  
If \param{displaced-to} is not supplied or \nil,
the resulting \term{array} is not a \term{displaced array}.
If array $A$ is created displaced to array $B$ and subsequently
array $B$ is given to \funref{adjust-array}, array $A$ will still be
displaced to array $B$.
Although \param{array} might be a \term{displaced array}, 
the resulting \term{array} is not a \term{displaced array} unless
\param{displaced-to} is supplied and not \nil.
\issue{ADJUST-ARRAY-DISPLACEMENT}
The interaction between \funref{adjust-array} and 
displaced \term{arrays} 
is as follows given three \term{arrays}, {\tt A}, {\tt B}, and~{\tt C}:

\beginlist
\itemitem{{\tt A} is not displaced before or after the call}

\code
 (adjust-array A ...)
\endcode

The dimensions of {\tt A} are altered, and the
contents rearranged as appropriate.  
Additional elements of {\tt A} are taken from
\param{initial-element}.  
The use of \param{initial-contents} causes all old contents to be
discarded.

\itemitem{{\tt A} is not displaced before, but is displaced to 
{\tt C} after the call}  

\code
 (adjust-array A ... :displaced-to C)
\endcode
               
None of the original contents of {\tt A} appears in 
{\tt A} afterwards; {\tt A} now contains
the contents of {\tt C}, without any rearrangement of {\tt C}.

\itemitem{{\tt A} is displaced to {\tt B} 
before the call, and is displaced to {\tt C} after 
the call}

\code
 (adjust-array A ... :displaced-to B)
 (adjust-array A ... :displaced-to C)
\endcode

{\tt B} and {\tt C} might be the same. The contents of {\tt B} do not appear in 
{\tt A} afterward unless such contents also happen to be in {\tt C}  If
\param{displaced-index-offset} 
is not supplied in the \funref{adjust-array} call, it defaults
to zero; the old offset into {\tt B} is not retained.
                                  
%% 17.6.0 11
\itemitem{{\tt A} is displaced to {\tt B} before the call, but not displaced
afterward.}  

\code
 (adjust-array A ... :displaced-to B)
 (adjust-array A ... :displaced-to nil)
\endcode
{\tt A} gets a
new ``data region,'' and contents of {\tt B} are copied into it as appropriate to
maintain the existing old contents; additional elements of {\tt A} 
are taken from
\param{initial-element} if supplied.  However, 
the use of \param{initial-contents} causes all old contents
to be discarded.
%If \param{array} is displaced
%to another array \f{x} when \funref{adjust-array}
%is applied to \param{array}, then afterwards neither 
%\param{array} nor the returned
%result is displaced to \f{x} unless such displacement is explicitly
%re-specified in the call to \funref{adjust-array}.
\endlist
\endissue{ADJUST-ARRAY-DISPLACEMENT}
                      
If \param{displaced-index-offset} is supplied,
it specifies the offset
of the resulting \term{array} from the beginning of 
the \term{array} that it is displaced to.           
If \param{displaced-index-offset} is not supplied, the offset is~0.  
The size of the resulting \term{array} plus the 
offset value cannot exceed the size of
the \term{array} that it is displaced to.

%% 17.6.0 5
If only \param{new-dimensions}
and an \param{initial-element} argument are supplied,
those elements of \param{array} that
are still in bounds appear in the resulting \term{array}. The elements of
the resulting \term{array} that are not in the bounds of 
\term{array} are initialized
to \param{initial-element}; if \param{initial-element} is not provided,
\issue{UNINITIALIZED-ELEMENTS:CONSEQUENCES-UNDEFINED}
%% This issue finally did pass at the March-93 meeting. -kmp 5-May-93
% then the initial contents of any resulting \term{elements}
% %%Rewritten to conform to the fact that issue UNINITIALIZED-ELEMENTS failed
% %%to clarify this in the other direction because consensus was that this was well-defined
% %%already. -kmp 15-Jan-92
% %are undefined.
% is \term{implementation-dependent}.
the consequences of later reading any such new \term{element} of \param{new-array}
before it has been initialized
are undefined.
\endissue{UNINITIALIZED-ELEMENTS:CONSEQUENCES-UNDEFINED}

If \param{initial-contents} or \param{displaced-to} is supplied,
then none of the original contents of \param{array} appears in the new \term{array}.

%% 17.6.0 10

\issue{ADJUST-ARRAY-NOT-ADJUSTABLE:IMPLICIT-COPY}
%The following will be deleted:
%
%The consequences are unspecified
%if \funref{adjust-array} is invoked with an \param{array} argument
%that is not adjustable.

\endissue{ADJUST-ARRAY-NOT-ADJUSTABLE:IMPLICIT-COPY}
%If \param{fill-pointer} is \nil,
%then the consequences are unspecified if
%\param{array} has a \term{fill pointer}.
The consequences are unspecified if \param{array} is adjusted 
to a size smaller than its \term{fill pointer} without supplying
the \param{fill-pointer} argument so that its \term{fill-pointer}
is properly adjusted in the process.

If {\tt A} is displaced to {\tt B}, the consequences are unspecified 
if {\tt B} is adjusted in such a way that it no longer has enough elements
to satisfy {\tt A}.  

\issue{ADJUST-ARRAY-NOT-ADJUSTABLE:IMPLICIT-COPY}

If \funref{adjust-array} is applied to an \term{array} that is \term{actually adjustable},
the \term{array} returned is \term{identical} to \param{array}.
%% This was only needed when "\term{expressly adjustable}" was used in the previous sentence.
% It is \term{implementation-dependent} whether \funref{adjust-array} 
% returns an \term{array} that is \term{identical} to its first argument for any
% other \term{arrays}.
If the \term{array} returned by \funref{adjust-array} 
is \term{distinct} from \param{array}, then the argument \param{array} is unchanged.
\endissue{ADJUST-ARRAY-NOT-ADJUSTABLE:IMPLICIT-COPY}
 
Note that if an \term{array} $A$ is displaced to another \term{array} $B$,
and $B$ is displaced to another \term{array} $C$, and $B$ is altered by
\funref{adjust-array}, $A$ must now refer to the adjust contents of $B$.
This means that an implementation cannot collapse the chain to make $A$
refer to $C$ directly and forget that the chain of reference passes through
$B$.  However, caching techniques are permitted as long as they preserve the 
semantics specified here.

\label Examples::

\code
 (adjustable-array-p
  (setq ada (adjust-array
              (make-array '(2 3)
                          :adjustable t
                          :initial-contents '((a b c) (1 2 3)))
              '(4 6)))) \EV T 
 (array-dimensions ada) \EV (4 6) 
 (aref ada 1 1) \EV 2 
 (setq beta (make-array '(2 3) :adjustable t))
\EV #2A((NIL NIL NIL) (NIL NIL NIL)) 
 (adjust-array beta '(4 6) :displaced-to ada)
\EV #2A((A B C NIL NIL NIL)
       (1 2 3 NIL NIL NIL)
       (NIL NIL NIL NIL NIL NIL) 
       (NIL NIL NIL NIL NIL NIL))
 (array-dimensions beta) \EV (4 6)
 (aref beta 1 1) \EV 2 
\endcode

%% 17.6.0 11
Suppose that the 4-by-4 array in \f{m} looks like this:

\code
#2A(( alpha     beta      gamma     delta )
    ( epsilon   zeta      eta       theta )
    ( iota      kappa     lambda    mu    )
    ( nu        xi        omicron   pi    ))
\endcode
Then the result of

\code
 (adjust-array m '(3 5) :initial-element 'baz)
\endcode
is a 3-by-5 array with contents       

\code
#2A(( alpha     beta      gamma     delta     baz )
    ( epsilon   zeta      eta       theta     baz )
    ( iota      kappa     lambda    mu        baz ))
\endcode

\label Affected By:\None!

%%I don't think this sort of thing is worthwhile to note. -kmp 7-Jan-91
%How \param{array} was created.

\label Exceptional Situations::

\issue{ADJUST-ARRAY-DISPLACED-TO-CONSTRAINTS:ALLOW-NIL}
An error \oftype{error} is signaled if \param{fill-pointer} is supplied
and \term{non-nil} but \param{array} has no \term{fill pointer}.
\endissue{ADJUST-ARRAY-DISPLACED-TO-CONSTRAINTS:ALLOW-NIL}

%\issue{ADJUST-ARRAY-NOT-ADJUSTABLE:DONKEY}
%An error \oftype{error} is signaled if an attempt
%  is made to adjust an \param{array} that is not adjustable (that is, an
%  \param{array} for which \funref{adjustable-array-p} returns \term{false}).
% 
%\endissue{ADJUST-ARRAY-NOT-ADJUSTABLE:DONKEY}
\label See Also::

\funref{adjustable-array-p},    \funref{make-array},
\conref{array-dimension-limit}, \conref{array-total-size-limit},
\typeref{array}

\label Notes:\None.

%\issue{ADJUST-ARRAY-NOT-ADJUSTABLE:DONKEY}
%\funref{adjustable-array-p} is an appropriate
%  predicate to determine whether \funref{adjust-array} will reliably succeed.
%\endissue{ADJUST-ARRAY-NOT-ADJUSTABLE:DONKE Y}

%% Sandra thinks this is redundant.
% \issue{ADJUST-ARRAY-NOT-ADJUSTABLE:IMPLICIT-COPY}
% The value returned by \funref{adjust-array} might not be \term{identical} 
% to the argument \param{array}.
% \endissue{ADJUST-ARRAY-NOT-ADJUSTABLE:IMPLICIT-COPY}

\endcom                 

%%% ========== ADJUSTABLE-ARRAY-P
\begincom{adjustable-array-p}\ftype{Function}

%!!! Barmar's review comments of this were made on the basis of document that
%    did not include various cleanups.  This should be re-reviewed carefully later.
%    -kmp 17-Dec-90

\label Syntax::

\DefunWithValues adjustable-array-p {array} {generalized-boolean}

\label Arguments and Values::

\param{array}---an \term{array}.

\param{generalized-boolean}---a \term{generalized boolean}.

\label Description::

\issue{ADJUST-ARRAY-NOT-ADJUSTABLE:IMPLICIT-COPY}

%% 17.3.0 13
Returns true if and only if \funref{adjust-array} could return a \term{value}
which is \term{identical} to \param{array} when given that \term{array} as its
first \term{argument}.
\endissue{ADJUST-ARRAY-NOT-ADJUSTABLE:IMPLICIT-COPY}

\label Examples::

\code
 (adjustable-array-p 
   (make-array 5
               :element-type 'character 
               :adjustable t 
               :fill-pointer 3)) \EV \term{true}
 (adjustable-array-p (make-array 4)) \EV \term{implementation-dependent}
\endcode

\label Affected By:\None.

\label Exceptional Situations::

Should signal an error \oftype{type-error} if its argument is not an \term{array}.

\label See Also::

\funref{adjust-array}, \funref{make-array}

\label Notes:\None.
%\issue{ADJUST-ARRAY-NOT-ADJUSTABLE:DONKEY}
%
%  
%  An \term{array} ``is adjustable'' 
%if {\tt (adjustable-array-p \term{array})} is \term{true}.
%  The adjustability of an \term{array} has no necessary
%  relation to any value that was given (or not given) by the \kwd{adjustable} 
%  \term{argument}
%  in the call to \funref{make-array} that created the \term{array}.
%
%  There is no portable way to create a non-adjustable
%  \term{array} (that is, an \term{array} for which \funref{adjustable-array-p} is
%  guaranteed to return \term{false}).
%\endissue{ADJUST-ARRAY-NOT-ADJUSTABLE:DONKEY}
\endcom

%%% ========== AREF
\begincom{aref}\ftype{Accessor}

\label Syntax::

\DefunWithValues aref {array {\rest} subscripts} {element}
\Defsetf         aref {array {\rest} subscripts} {new-element}

\label Arguments and Values::

\param{array}---an \term{array}.

\issue{ARRAY-DIMENSION-LIMIT-IMPLICATIONS:ALL-FIXNUM}
\param{subscripts}---a \term{list} of \term{valid array indices} for the \param{array}.
\endissue{ARRAY-DIMENSION-LIMIT-IMPLICATIONS:ALL-FIXNUM}

\param{element}, \param{new-element}---an \term{object}.

\label Description::

%% 17.2.0 3
\term{Accesses} the \param{array} \term{element} specified by the \param{subscripts}.
If no \param{subscripts} are supplied and \param{array} is zero rank,
\funref{aref} \term{accesses} the sole element of \param{array}.

%% 17.2.0 4
\funref{aref} ignores \term{fill pointers}.
It is permissible to use \funref{aref} 
to \term{access} any \param{array} \term{element},
whether \term{active} or not.

%% This is implied by the use of "access" above.
% %% 17.2.0 5
% \macref{setf} can be used with \funref{aref} to destructively replace
% an \term{array} element with a new value.

\label Examples::

%% 2.5.0 6
If the variable \f{foo} names a 3-by-5 array,
then the first index could be 0, 1, or 2, and then second index
could be 0, 1, 2, 3, or 4.  The array elements can be referred to by using
\thefunction{aref}; for example, \f{(aref foo 2 1)}
refers to element (2, 1) of the array.  

\code
 (aref (setq alpha (make-array 4)) 3) \EV \term{implementation-dependent}
 (setf (aref alpha 3) 'sirens) \EV SIRENS
 (aref alpha 3) \EV SIRENS
 (aref (setq beta (make-array '(2 4) 
                    :element-type '(unsigned-byte 2)
                    :initial-contents '((0 1 2 3) (3 2 1 0))))
        1 2) \EV 1
 (setq gamma '(0 2))
 (apply #'aref beta gamma) \EV 2
 (setf (apply #'aref beta gamma) 3) \EV 3
 (apply #'aref beta gamma) \EV 3
 (aref beta 0 2) \EV 3
\endcode

\label Affected By:\None.

\label Exceptional Situations:

\issue{AREF-ACCESSOR-DIMENSION-ERROR:SIGNAL-ERROR-IN-SAFE-CODE}
\issue{AREF-ACCESSOR-RANK-ERROR:SIGNAL-ERROR-IN-SAFE-CODE}
\issue{AREF-ACCESSOR-TYPE-ERROR:SIGNAL-ERROR-IN-SAFE-CODE}
\None.
\endissue{AREF-ACCESSOR-TYPE-ERROR:SIGNAL-ERROR-IN-SAFE-CODE}
\endissue{AREF-ACCESSOR-RANK-ERROR:SIGNAL-ERROR-IN-SAFE-CODE}
\endissue{AREF-ACCESSOR-DIMENSION-ERROR:SIGNAL-ERROR-IN-SAFE-CODE}

\label See Also::

\funref{bit},
\funref{char},
\funref{elt},
\funref{row-major-aref},
\funref{svref},
\issue{CONSTANT-MODIFICATION:DISALLOW}
{\secref\ConstantModification}
\endissue{CONSTANT-MODIFICATION:DISALLOW}

\label Notes:\None.

\endcom

%%% ========== ARRAY-DIMENSION
\begincom{array-dimension}\ftype{Function}

\label Syntax::

\DefunWithValues array-dimension {array axis-number} {dimension}

\label Arguments and Values::

\param{array}---an \term{array}.

\param{axis-number}---an \term{integer} greater than or equal to zero
		      and less than the \term{rank} of the \param{array}.

% Barmar observes that nothing requires the implementation to enforce ARRAY-DIMENSION-LIMIT.
% It's a requirement on the user to be prepared to lose if he exceeds it,
% but it is not a requirement on the implementation to make him lose in that case.
\param{dimension}---a non-negative \term{integer}.

\label Description::

%!!! (array-dimension (make-array nil) ??)

%% 17.3.0 6
\funref{array-dimension} returns the \param{axis-number} 
\term{dimension}\meaning{1} of \param{array}.
%This next is implied by the definition of dimension, but retained for now for safety. -kmp
(Any \term{fill pointer} is ignored.)

\label Examples::

\code
 (array-dimension (make-array 4) 0) \EV 4
 (array-dimension (make-array '(2 3)) 1) \EV 3
\endcode


\label Affected By::    
None.
\label Exceptional Situations:\None.

\label See Also::

\funref{array-dimensions}, \funref{length}

\label Notes::
\code
 (array-dimension array n) \EQ (nth n (array-dimensions array))
\endcode
\endcom

%%% ========== ARRAY-DIMENSIONS
\begincom{array-dimensions}\ftype{Function}

\label Syntax::

\DefunWithValues array-dimensions {array} {dimensions}

\label Arguments and Values::

\param{array}---an \term{array}.

% Barmar observes that nothing requires the implementation to enforce ARRAY-DIMENSION-LIMIT.
% It's a requirement on the user to be prepared to lose if he exceeds it,
% but it is not a requirement on the implementation to make him lose in that case.
\param{dimensions}---a \term{list} of \term{integers}.

\label Description::                    

%% 17.3.0 8
Returns a \term{list} of the \term{dimensions} of \param{array}.
(If \param{array} is a \term{vector} with a \term{fill pointer}, 
 that \term{fill pointer} is ignored.)

\label Examples::

\code
 (array-dimensions (make-array 4)) \EV (4)
 (array-dimensions (make-array '(2 3))) \EV (2 3)
 (array-dimensions (make-array 4 :fill-pointer 2)) \EV (4)
\endcode

\label Affected By:\None!

\label Exceptional Situations::

Should signal an error \oftype{type-error} if its argument is not an \term{array}.

\label See Also::

\funref{array-dimension}

%% Per X3J13. -kmp 05-Oct-93
\label Notes:\None.
                                
\endcom

%%% ========== ARRAY-ELEMENT-TYPE
\begincom{array-element-type}\ftype{Function}

\label Syntax::

\DefunWithValues array-element-type {array} {typespec}

\label Arguments and Values::

\param{array}---an \term{array}.

\param{typespec}---a \term{type specifier}.

\label Description::

%% 17.3.0 3
\issue{ARRAY-TYPE-ELEMENT-TYPE-SEMANTICS:UNIFY-UPGRADING}
Returns a \term{type specifier} which represents the \term{actual array element type}
of the array, which is the set of \term{objects} that such an \param{array} can hold.
%Barmar points out that this is redundant.
% after upgrading.
(Because of \term{array} \term{upgrading}, this \term{type specifier} can in
some cases denote a \term{supertype} of the \term{expressed array element type}
of the \param{array}.)
%\term{type} requested by the user when \param{array} was created.
\endissue{ARRAY-TYPE-ELEMENT-TYPE-SEMANTICS:UNIFY-UPGRADING}

\label Examples::

\code
 (array-element-type (make-array 4)) \EV T
 (array-element-type (make-array 12 :element-type '(unsigned-byte 8))) 
\EV \term{implementation-dependent}
 (array-element-type (make-array 12 :element-type '(unsigned-byte 5)))
\EV \term{implementation-dependent}
\endcode

\code
 (array-element-type (make-array 5 :element-type '(mod 5)))
\endcode
could be \f{(mod 5)}, \f{(mod 8)}, \f{fixnum}, \f{t}, or any other
type of which \f{(mod 5)} is a \term{subtype}.

\label Affected By::

The \term{implementation}.

\label Exceptional Situations::

Should signal an error \oftype{type-error} if its argument is not an \term{array}.

\label See Also::

\typeref{array},
\funref{make-array},
\funref{subtypep},
\funref{upgraded-array-element-type}

\label Notes:\None.

\endcom

%%% ========== ARRAY-HAS-FILL-POINTER-P
\begincom{array-has-fill-pointer-p}\ftype{Function}

\label Syntax::

\DefunWithValues array-has-fill-pointer-p {array} {generalized-boolean}

\label Arguments and Values::

\param{array}---an \term{array}.

\param{generalized-boolean}---a \term{generalized boolean}.

\label Description::

%% 17.5.0 5
Returns \term{true} if \param{array} has a \term{fill pointer};
otherwise returns \term{false}.

\label Examples::

\code
 (array-has-fill-pointer-p (make-array 4)) \EV \term{implementation-dependent}
 (array-has-fill-pointer-p (make-array '(2 3))) \EV \term{false}
 (array-has-fill-pointer-p
   (make-array 8 
               :fill-pointer 2 
               :initial-element 'filler)) \EV \term{true}
\endcode

\label Affected By:\None!

\label Exceptional Situations::

Should signal an error \oftype{type-error} if its argument is not an \term{array}.

\label See Also::

\funref{make-array}, \funref{fill-pointer}

\label Notes::

Since \term{arrays} of \term{rank} other than one cannot have a \term{fill pointer},
\funref{array-has-fill-pointer-p} always returns \nil\ when its argument
is such an array.

\endcom

%--------------------

\begincom{array-displacement}\ftype{Function}

\issue{DISPLACED-ARRAY-PREDICATE:ADD}

\label Syntax::

\DefunWithValues array-displacement {array} {displaced-to, displaced-index-offset}

\label Arguments and Values::

\param{array}---an \term{array}.

\param{displaced-to}---an \param{array} or \nil.

\param{displaced-index-offset}---a non-negative \term{fixnum}.

\label Description::

If the \param{array} is a \term{displaced array},
returns the \term{values} of the \kwd{displaced-to} and \kwd{displaced-index-offset}
options
for the \term{array} (\seefuns{make-array} and \funref{adjust-array}).
If the \param{array} is not a \term{displaced array},
\nil\ and \f{0} are returned.

If \funref{array-displacement} is called on an \param{array}
for which a \term{non-nil} \term{object} was provided as the
\kwd{displaced-to} \term{argument} to \funref{make-array} 
or \funref{adjust-array}, it must return that \term{object}
as its first value. It is \term{implementation-dependent}
whether \funref{array-displacement} returns a \term{non-nil}
\term{primary value} for any other \param{array}.

\label Examples::

\code
 (setq a1 (make-array 5)) \EV #<ARRAY 5 simple 46115576>
 (setq a2 (make-array 4 :displaced-to a1
                        :displaced-index-offset 1))
\EV #<ARRAY 4 indirect 46117134>
 (array-displacement a2)
\EV #<ARRAY 5 simple 46115576>, 1
 (setq a3 (make-array 2 :displaced-to a2
                        :displaced-index-offset 2))
\EV #<ARRAY 2 indirect 46122527>
 (array-displacement a3)
\EV #<ARRAY 4 indirect 46117134>, 2
\endcode

\label Affected By:\None!

\label Exceptional Situations::

\Shouldchecktype{array}{an \term{array}}

\label See Also::

\funref{make-array}

%% Per X3J13. -kmp 05-Oct-93
\label Notes:\None.
\endissue{DISPLACED-ARRAY-PREDICATE:ADD}

\endcom


%%% ========== ARRAY-IN-BOUNDS-P
\begincom{array-in-bounds-p}\ftype{Function}

\label Syntax::

\DefunWithValues array-in-bounds-p {array {\rest} subscripts} {generalized-boolean}

\label Arguments and Values::

\param{array}---an \term{array}.

% Note that they needn't be valid array indices, 
% since that would make this function a lot less interesting!
\param{subscripts}---a list of \term{integers} 
		     of length equal to the \term{rank} of the \term{array}.

\param{generalized-boolean}---a \term{generalized boolean}.

\label Description::

%% 17.3.0 10
Returns \term{true} if the \param{subscripts} are all in bounds for \param{array};
otherwise returns \term{false}.
(If \param{array} is a \term{vector} with a \term{fill pointer}, 
 that \term{fill pointer} is ignored.)

\label Examples::
\code
 (setq a (make-array '(7 11) :element-type 'string-char))
 (array-in-bounds-p a 0  0) \EV \term{true}
 (array-in-bounds-p a 6 10) \EV \term{true}
 (array-in-bounds-p a 0 -1) \EV \term{false}
 (array-in-bounds-p a 0 11) \EV \term{false}
 (array-in-bounds-p a 7  0) \EV \term{false}
\endcode

\label Affected By:\None.

\label Exceptional Situations:\None.

\label See Also::

\funref{array-dimensions}

\label Notes::
\code
 (array-in-bounds-p array subscripts)   
 \EQ (and (not (some #'minusp (list subscripts)))
         (every #'< (list subscripts) (array-dimensions array)))
\endcode
\endcom

%%% ========== ARRAY-RANK
\begincom{array-rank}\ftype{Function}

\label Syntax::

\DefunWithValues array-rank {array} {rank}

\label Arguments and Values::

\param{array}---an \term{array}.

% Barmar observes that nothing requires the implementation to enforce ARRAY-RANK-LIMIT.
% It's a requirement on the user to be prepared to lose if he exceeds it,
% but it is not a requirement on the implementation to make him lose in that case.
\param{rank}---a non-negative \term{integer}.

\label Description::

%% 17.3.0 4                                         
Returns the number of \term{dimensions} of \param{array}.
                                     
\label Examples::

\code
 (array-rank (make-array '())) \EV 0
 (array-rank (make-array 4)) \EV 1
 (array-rank (make-array '(4))) \EV 1
 (array-rank (make-array '(2 3))) \EV 2
\endcode

\label Affected By:\None.
                          
\label Exceptional Situations::

Should signal an error \oftype{type-error} if its argument is not an \term{array}.

\label See Also::

\conref{array-rank-limit}, \funref{make-array}

\label Notes:\None.

\endcom

%%% ========== ARRAY-ROW-MAJOR-INDEX
\begincom{array-row-major-index}\ftype{Function}

\label Syntax::

\DefunWithValues array-row-major-index {array {\rest} subscripts} {index}

\label Arguments and Values::

\param{array}---an \term{array}.

\param{subscripts}---a \term{list} of \term{valid array indices} for the \param{array}.

\param{index}---a \term{valid array row-major index} for the \param{array}.

\label Description::

%% 17.3.0 11
Computes the position according to the row-major ordering of \param{array}
for the element that is specified by \param{subscripts}, and returns the
offset of the element in the computed position from the beginning of \param{array}.

For a one-dimensional \param{array}, 
the result of \funref{array-row-major-index}
equals \param{subscript}.

\funref{array-row-major-index} ignores \term{fill pointers}.

\label Examples::

\code
 (setq a (make-array '(4 7) :element-type '(unsigned-byte 8)))
 (array-row-major-index a 1 2) \EV 9
 (array-row-major-index 
    (make-array '(2 3 4) 
                :element-type '(unsigned-byte 8)
                :displaced-to a
                :displaced-index-offset 4)
    0 2 1) \EV 9
\endcode

\label Affected By:\None.

\label Exceptional Situations:\None.

\label See Also:\None.

\label Notes::

%% 17.3.0 12
A possible definition of \funref{array-row-major-index}, 
with no error-checking, is

\code
 (defun array-row-major-index (a &rest subscripts)
   (apply #'+ (maplist #'(lambda (x y)
                            (* (car x) (apply #'* (cdr y))))
                       subscripts
                       (array-dimensions a))))
\endcode

\endcom

%%% ========== ARRAY-TOTAL-SIZE
\begincom{array-total-size}\ftype{Function}

\label Syntax::

\DefunWithValues array-total-size {array} {size}

\label Arguments and Values::

\param{array}---an \term{array}.

% Barmar observes that nothing requires the implementation to enforce ARRAY-TOTAL-SIZE-LIMIT.
% It's a requirement on the user to be prepared to lose if he exceeds it,
% but it is not a requirement on the implementation to make him lose in that case.
\param{size}---a non-negative \term{integer}.

\label Description::

%% 17.3.0 9
Returns the \term{array total size} of the \param{array}.

\label Examples::

\code
 (array-total-size (make-array 4)) \EV 4
 (array-total-size (make-array 4 :fill-pointer 2)) \EV 4
 (array-total-size (make-array 0)) \EV 0
 (array-total-size (make-array '(4 2))) \EV 8
 (array-total-size (make-array '(4 0))) \EV 0
 (array-total-size (make-array '())) \EV 1
\endcode

\label Affected By:\None!

\label Exceptional Situations::

Should signal an error \oftype{type-error} if its argument is not an \term{array}.

\label See Also::

\funref{make-array}, \funref{array-dimensions}

\label Notes::

If the \param{array} is a \term{vector} with a \term{fill pointer},
the \term{fill pointer} is ignored when calculating the \term{array total size}.

Since the product of no arguments is one, the \term{array total size} of a
zero-dimensional \term{array} is one.

\code
 (array-total-size x)
    \EQ (apply #'* (array-dimensions x))
    \EQ (reduce #'* (array-dimensions x))
\endcode

\endcom

%%% ========== ARRAYP
\begincom{arrayp}\ftype{Function}

\label Syntax::

\DefunWithValues arrayp {object} {generalized-boolean}

\label Arguments and Values::

\param{object}---an \term{object}.

\param{generalized-boolean}---a \term{generalized boolean}.

\label Description::

%% 6.2.2 24
\TypePredicate{object}{array}
       
\label Examples::

\code
 (arrayp (make-array '(2 3 4) :adjustable t)) \EV \term{true}
 (arrayp (make-array 6)) \EV \term{true}
 (arrayp #*1011) \EV \term{true}
 (arrayp "hi") \EV \term{true}
 (arrayp 'hi) \EV \term{false}
 (arrayp 12) \EV \term{false}
\endcode

\label Affected By:\None.

\label Exceptional Situations:\None!

\label See Also::

\funref{typep}

\label Notes::

\code
 (arrayp \param{object}) \EQ (typep \param{object} 'array)
\endcode

\endcom

%%% ========== FILL-POINTER
\begincom{fill-pointer}\ftype{Accessor}

\label Syntax::

%% 17.5.0 7
\DefunWithValues fill-pointer {vector} {fill-pointer}
\Defsetf         fill-pointer {vector} {new-fill-pointer}

\label Arguments and Values::

\param{vector}---a \term{vector} with a \term{fill pointer}.

\param{fill-pointer}, \param{new-fill-pointer}---a \term{valid fill pointer} 
  for the \param{vector}.

\label Description::

%% 17.5.0 6
\term{Accesses} the \term{fill pointer} of \param{vector}.

\label Examples::

\code
 (setq a (make-array 8 :fill-pointer 4)) \EV #(NIL NIL NIL NIL)
 (fill-pointer a) \EV 4
 (dotimes (i (length a)) (setf (aref a i) (* i i))) \EV NIL
 a \EV #(0 1 4 9)
 (setf (fill-pointer a) 3) \EV 3
 (fill-pointer a) \EV 3
 a \EV #(0 1 4)
 (setf (fill-pointer a) 8) \EV 8
 a \EV #(0 1 4 9 NIL NIL NIL NIL)
\endcode

\label Side Effects:\None!

\label Affected By:\None!

\label Exceptional Situations::

\Shouldchecktype{vector}{a \term{vector} with a \term{fill pointer}}

\label See Also::

\funref{make-array}, \funref{length}

\label Notes::

There is no \term{operator} that will remove a \term{vector}'s \term{fill pointer}.

\endcom

%%% ========== ROW-MAJOR-AREF
\begincom{row-major-aref}\ftype{Accessor}
\issue{AREF-1D}

\label Syntax:: 

\DefunWithValues row-major-aref {array index} {element}
\Defsetf         row-major-aref {array index} {new-element}

\label Arguments and Values::

\param{array}---an \term{array}.

\issue{ARRAY-DIMENSION-LIMIT-IMPLICATIONS:ALL-FIXNUM}
\param{index}---a \term{valid array row-major index} for the \param{array}.
\endissue{ARRAY-DIMENSION-LIMIT-IMPLICATIONS:ALL-FIXNUM}

\param{element}, \param{new-element}---an \term{object}.

\label Description::

Considers \term{array} as a \term{vector} by viewing its \term{elements}
in row-major order, and returns the \term{element} of that \term{vector} 
which is referred to by the given \param{index}.

\funref{row-major-aref} is valid for use with \macref{setf}.

\label Examples:\None.

\label Side Effects:\None.

\label Affected By:\None.

\label Exceptional Situations:\None.

\label See Also::

\funref{aref},
\funref{array-row-major-index}

\label Notes::

\code
 (row-major-aref array index) \EQ
   (aref (make-array (array-total-size array)
                     :displaced-to array
                     :element-type (array-element-type array))
         index)

 (aref array i1 i2 ...) \EQ
     (row-major-aref array (array-row-major-index array i1 i2))
\endcode

\endissue{AREF-1D}

\endcom


%%% ========== UPGRADED-ARRAY-ELEMENT-TYPE
\begincom{upgraded-array-element-type}\ftype{Function}

\issue{SUBTYPEP-ENVIRONMENT:ADD-ARG}
\issue{ARRAY-TYPE-ELEMENT-TYPE-SEMANTICS:UNIFY-UPGRADING}
\label Syntax::

\DefunWithValues upgraded-array-element-type {typespec {\opt} environment}
		 {upgraded-typespec}

\label Arguments and Values::

\param{typespec}---a \term{type specifier}.

\param{environment}---an \term{environment} \term{object}.
  \Default{\nil, denoting the \term{null lexical environment}
	   and the current \term{global environment}}
%!!! Need to say what happens with the environment.

\param{upgraded-typespec}---a \term{type specifier}.

\label Description::

\issue{ARRAY-UPGRADING-UNDECIDABLE:RESTRICT}

Returns the \term{element type} of 
the most \term{specialized} \term{array} representation capable of 
holding items of the \term{type} denoted by \param{typespec}.

%Added by KMP in response to a Barmar comment. -kmp 29-Jul-91
The \param{typespec} is a \term{subtype} of 
(and possibly \term{type equivalent} to)
the \param{upgraded-typespec}.

If \param{typespec} is \typeref{bit},
 the result is \term{type equivalent} to \f{bit}.
\issue{CHARACTER-VS-CHAR:LESS-INCONSISTENT-SHORT}
If \param{typespec} is \typeref{base-char},
 the result is \term{type equivalent} to \f{base-char}.
\endissue{CHARACTER-VS-CHAR:LESS-INCONSISTENT-SHORT}
If \param{typespec} is \typeref{character},
 the result is \term{type equivalent} to \f{character}.

The purpose of \funref{upgraded-array-element-type} is to reveal how
an implementation does its \term{upgrading}.

The \param{environment} is used to expand any \term{derived type specifiers}
that are mentioned in the \param{typespec}.

\endissue{ARRAY-UPGRADING-UNDECIDABLE:RESTRICT}

\label Examples:\None.

\label Side Effects:\None.

\label Affected By:\None.

\label Exceptional Situations:

\issue{INVALID-TYPE-SPECIFIER}
\None.
\endissue{INVALID-TYPE-SPECIFIER}

\label See Also::

\funref{array-element-type},
\funref{make-array}

\label Notes::

Except for storage allocation consequences and dealing correctly with the
optional \param{environment} \term{argument},
\funref{upgraded-array-element-type} could be defined as:

\code
 (defun upgraded-array-element-type (type &optional environment)
   (array-element-type (make-array 0 :element-type type)))
\endcode
                                                     
\endissue{ARRAY-TYPE-ELEMENT-TYPE-SEMANTICS:UNIFY-UPGRADING}
\endissue{SUBTYPEP-ENVIRONMENT:ADD-ARG}

\endcom

%%% ========== ARRAY-DIMENSION-LIMIT
\begincom{array-dimension-limit}\ftype{Constant Variable}

\label Constant Value::

A positive
\issue{FIXNUM-NON-PORTABLE:TIGHTEN-DEFINITION}
\term{fixnum},
\endissue{FIXNUM-NON-PORTABLE:TIGHTEN-DEFINITION}
the exact magnitude of which is \term{implementation-dependent},
but which is not less than \f{1024}.

\label Description::

%% 17.1.0 18
The upper exclusive bound on each individual \term{dimension} of an \term{array}.

\label Examples:\None.

\label See Also::

\funref{make-array}

\label Notes:\None.

\endcom

%%% ========== ARRAY-RANK-LIMIT
\begincom{array-rank-limit}\ftype{Constant Variable}

\label Constant Value::

A positive
\issue{ARRAY-DIMENSION-LIMIT-IMPLICATIONS:ALL-FIXNUM}
\term{fixnum},
\endissue{ARRAY-DIMENSION-LIMIT-IMPLICATIONS:ALL-FIXNUM}
the exact magnitude of which is \term{implementation-dependent},
but which is not less than \f{8}.

\label Description::

%% 17.1.0 17
The upper exclusive bound on the \term{rank} of an \term{array}.

\label Examples:\None.

\label See Also::

\funref{make-array}

\label Notes:\None.

\endcom

%%% ========== ARRAY-TOTAL-SIZE-LIMIT
\begincom{array-total-size-limit}\ftype{Constant Variable}

\label Constant Value::

A positive
\issue{ARRAY-DIMENSION-LIMIT-IMPLICATIONS:ALL-FIXNUM}
\term{fixnum},
\endissue{ARRAY-DIMENSION-LIMIT-IMPLICATIONS:ALL-FIXNUM}
the exact magnitude of which is \term{implementation-dependent},
but which is not less than \f{1024}.

\label Description::

%% 17.1.0 19
The upper exclusive bound on the \term{array total size} of an \term{array}.

%% 17.1.0 20
The actual limit on the \term{array total size} 
imposed by the \term{implementation}
might vary according the \term{element type} of the \term{array};
in this case, the value of \conref{array-total-size-limit} 
will be the smallest of these possible limits.

\label Examples:\None.

\label See Also::

\funref{make-array}, \funref{array-element-type}

\label Notes:\None.

\endcom

%-------------------- Vectors --------------------

%%% ========== SIMPLE-VECTOR-P
\begincom{simple-vector-p}\ftype{Function}

\label Syntax::

\DefunWithValues simple-vector-p {object} {generalized-boolean}

\label Arguments and Values::

\param{object}---an \term{object}.

\param{generalized-boolean}---a \term{generalized boolean}.

\label Description::

%% 6.2.2 21
\TypePredicate{object}{simple-vector}.

\label Examples::        

\code
 (simple-vector-p (make-array 6)) \EV \term{true}
 (simple-vector-p "aaaaaa") \EV \term{false}
 (simple-vector-p (make-array 6 :fill-pointer t)) \EV \term{false}
\endcode

\label Side Effects:\None.

\label Affected By:\None.

\label Exceptional Situations:\None!

\label See Also:: 

\typeref{simple-vector}

\label Notes::

\code
 (simple-vector-p \param{object}) \EQ (typep \param{object} 'simple-vector)
\endcode

\endcom

%%% ========== SVREF
\begincom{svref}\ftype{Accessor}

\label Syntax::

\DefunWithValues svref {simple-vector index} {element}
\Defsetf         svref {simple-vector index} {new-element}
                                                            
\label Arguments and Values::

\param{simple-vector}---a \term{simple vector}.

\param{index}---a \term{valid array index} for the \param{simple-vector}.

\param{element}, \param{new-element}---an \term{object}
  (whose \term{type} is a \term{subtype} 
   of the \term{array element type} of the \param{simple-vector}).

\label Description::

%% 17.2.0 7
\term{Accesses} the \term{element} of \param{simple-vector} specified by \param{index}.

%% Implied by use of "Access". -kmp 15-Jan-92
% %% 17.2.0 7
% \macref{setf} may be used with \funref{svref} to destructively replace
% a \term{simple vector} element with a new value.

\label Examples::

\code
 (simple-vector-p (setq v (vector 1 2 'sirens))) \EV \term{true}
 (svref v 0) \EV 1
 (svref v 2) \EV SIRENS
 (setf (svref v 1) 'newcomer) \EV NEWCOMER               
 v \EV #(1 NEWCOMER SIRENS)
\endcode

\label Side Effects:\None.

\label Affected By:\None.

\label Exceptional Situations:\None.

\label See Also::

\funref{aref},
\funref{sbit},
\funref{schar},
\funref{vector},
\issue{CONSTANT-MODIFICATION:DISALLOW}
{\secref\ConstantModification}
\endissue{CONSTANT-MODIFICATION:DISALLOW}

\label Notes::

%% 17.2.0 8
\funref{svref} is identical to \funref{aref} 
except that it requires its first argument to be a \term{simple vector}.  

\code
 (svref \param{v} \param{i}) \EQ (aref (the simple-vector \param{v}) \param{i})
\endcode

\endcom

%%% ========== VECTOR
\begincom{vector}\ftype{Function}

\label Syntax::

\DefunWithValues vector {{\rest} objects} {vector}

\label Arguments and Values:: 

\param{object}---an \term{object}.

\param{vector}---a \term{vector} of \term{type} \f{(vector t {\tt *})}.

\label Description::

%% 17.1.0 21
Creates a \term{fresh} \term{simple general vector} whose size
corresponds to the number of \param{objects}. 

The \term{vector} is initialized to contain the \param{objects}.

\label Examples::

\code
 (arrayp (setq v (vector 1 2 'sirens))) \EV \term{true}
 (vectorp v) \EV \term{true}
 (simple-vector-p v) \EV \term{true}         
 (length v) \EV 3
\endcode

\label Side Effects:\None.

\label Affected By:\None.

\label Exceptional Situations:\None!

\label See Also::

\funref{make-array}

\label Notes::

\funref{vector} is analogous to \funref{list}.

\code
 (vector a\ssso a\ssst ... a\sssn)
  \EQ (make-array (list \i{n}) :element-type t
                          :initial-contents 
                            (list a\ssso a\ssst ... a\sssn))
\endcode
\endcom

%%% ========== VECTOR-POP
\begincom{vector-pop}\ftype{Function}

\label Syntax::

\DefunWithValues vector-pop {vector} {element}

\label Arguments and Values::

\param{vector}---a \term{vector} with a \term{fill pointer}.

\param{element}---an \term{object}.

\label Description::

%% 17.5.0 10
Decreases the \term{fill pointer} of \param{vector} by one, 
and retrieves the \term{element} of \param{vector} that is
designated by the new \term{fill pointer}.

\label Examples::

\code
 (vector-push (setq fable (list 'fable))
              (setq fa (make-array 8
                                   :fill-pointer 2
                                   :initial-element 'sisyphus))) \EV 2 
 (fill-pointer fa) \EV 3 
 (eq (vector-pop fa) fable) \EV \term{true}
 (vector-pop fa) \EV SISYPHUS 
 (fill-pointer fa) \EV 1 
\endcode

\label Side Effects::

The \term{fill pointer} is decreased by one.

\label Affected By::

The value of the \term{fill pointer}.

\label Exceptional Situations::

An error \oftype{type-error} is signaled if \param{vector} does not have a \term{fill pointer}.

If the \term{fill pointer} is zero, \funref{vector-pop} signals an error \oftype{error}.

\label See Also::

\funref{vector-push}, \funref{vector-push-extend}, \funref{fill-pointer}

\label Notes:\None.

\endcom

%%% ========== VECTOR-PUSH
%%% ========== VECTOR-PUSH-EXTEND
\begincom{vector-push, vector-push-extend}\ftype{Function}

\label Syntax::

\DefunWithValues vector-push {new-element vector} {new-index-p}

\DefunWithValues vector-push-extend {new-element vector {\opt} extension} {new-index}

\label Arguments and Values::

\param{new-element}---an \term{object}.

\param{vector}---a \term{vector} with a \term{fill pointer}.

\param{extension}---a positive \term{integer}.
 \Default{\term{implementation-dependent}}

\param{new-index-p}---a \term{valid array index} for \param{vector}, or \nil.

\param{new-index}---a \term{valid array index} for \param{vector}.

\label Description::

%% 17.5.0 8
\funref{vector-push} and \funref{vector-push-extend} store 
\param{new-element} in \param{vector}.
\funref{vector-push} attempts to store
\param{new-element} 
in the element of \param{vector} designated by the \term{fill pointer},
and to increase the \term{fill pointer} by one.  If the 
\f{(>= (fill-pointer \param{vector}) (array-dimension \param{vector} 0))},
neither \param{vector} nor its \term{fill pointer} are affected.
Otherwise, the store and increment take
place and \funref{vector-push} 
returns the former value of the \term{fill pointer}
which is one less than the one it leaves in \param{vector}.

%% 17.5.0 9
\funref{vector-push-extend} is just like \funref{vector-push} except
that if the \term{fill pointer} gets too large, \param{vector} is extended using
\funref{adjust-array} so that it can contain more elements.
\param{Extension}
is the minimum number of elements to be added to \param{vector} if it
must be extended.

\funref{vector-push} and 
\funref{vector-push-extend} return the index of \param{new-element} in \param{vector}.
If \f{(>= (fill-pointer \param{vector}) (array-dimension \param{vector} 0))},
\funref{vector-push} returns \nil.

\label Examples::

\code
 (vector-push (setq fable (list 'fable))
              (setq fa (make-array 8 
                                   :fill-pointer 2
                                   :initial-element 'first-one))) \EV 2 
 (fill-pointer fa) \EV 3 
 (eq (aref fa 2) fable) \EV \term{true}
 (vector-push-extend #\\X
                    (setq aa 
                          (make-array 5
                                      :element-type 'character
                                      :adjustable t
                                      :fill-pointer 3))) \EV 3 
 (fill-pointer aa) \EV 4 
 (vector-push-extend #\\Y aa 4) \EV 4 
 (array-total-size aa) \EV at least 5 
 (vector-push-extend #\\Z aa 4) \EV 5 
 (array-total-size aa) \EV 9 ;(or more)
\endcode

\label Affected By::
The value of the \term{fill pointer}.

How \param{vector} was created.

\label Exceptional Situations::         

An error \oftype{error} is signaled by \funref{vector-push-extend}
if it tries to extend \param{vector} and \param{vector} is not \term{actually adjustable}.

An error \oftype{error} is signaled if \param{vector} does not 
have a \term{fill pointer}.

\label See Also::

\funref{adjustable-array-p}, \funref{fill-pointer}, \funref{vector-pop}

\label Notes:\None.

\endcom

%%% ========== VECTORP
\begincom{vectorp}\ftype{Function}

\label Syntax::

\DefunWithValues vectorp {object} {generalized-boolean}

\label Arguments and Values::

\param{object}---an \term{object}.

\param{generalized-boolean}---a \term{generalized boolean}.

\label Description::

%% 6.2.2 20
\TypePredicate{object}{vector}

\label Examples::

\code
 (vectorp "aaaaaa") \EV \term{true}
 (vectorp (make-array 6 :fill-pointer t)) \EV \term{true}
 (vectorp (make-array '(2 3 4))) \EV \term{false}
 (vectorp #*11) \EV \term{true}
 (vectorp #b11) \EV \term{false}
\endcode

\label Side Effects:\None.

\label Affected By:\None.

\label Exceptional Situations:\None!

\label See Also:\None.

\label Notes::
\code
 (vectorp \param{object}) \EQ (typep \param{object} 'vector)
\endcode

\endcom

%-------------------- Bit Vectors --------------------

%%% ========== BIT
%%% ========== SBIT
\begincom{bit, sbit}\ftype{Accessor}

\label Syntax::

\DefunMultiWithValues {bit-array {\rest} subscripts} {bit}
   {\entry{bit}
    \entry{sbit}}
\DefsetfMulti         {bit-array {\rest} subscripts} {new-bit}
   {\entry{bit}
    \entry{sbit}}

\label Arguments and Values::

\param{bit-array}---for \funref{bit},  a \term{bit array};
		    for \funref{sbit}, a \term{simple bit array}.

\param{subscripts}---a \term{list} of \term{valid array indices} 
		     for the \param{bit-array}.

\param{bit}---a \term{bit}.

\label Description::

%% 17.4.0 3
%% 17.4.0 4
\funref{bit} and \funref{sbit} \term{access} the \param{bit-array} 
\term{element} specified by \param{subscripts}.

%!!! Is this necessary to say? How is it said elsewhere?
These \term{functions} ignore the \term{fill pointer} when \term{accessing} \term{elements}.

%% This is implied by "access" above.
% %% 17.4.0 6
% \macref{setf} can be used with \funref{bit} or \funref{sbit} to destructively 
% replace a \term{bit array} \term{element} with a new value.

\label Examples::

\code
 (bit (setq ba (make-array 8 
                            :element-type 'bit 
                            :initial-element 1))
       3) \EV 1
 (setf (bit ba 3) 0) \EV 0
 (bit ba 3) \EV 0
 (sbit ba 5) \EV 1
 (setf (sbit ba 5) 1) \EV 1
 (sbit ba 5) \EV 1
\endcode

\label Affected By:\None.

\label Exceptional Situations:\None.

\label See Also::

\funref{aref},
\issue{CONSTANT-MODIFICATION:DISALLOW}
{\secref\ConstantModification}
\endissue{CONSTANT-MODIFICATION:DISALLOW}

\label Notes::

%% 17.4.0 7
\funref{bit} and \funref{sbit} are like \funref{aref}
except that they require \param{arrays} to be
a \term{bit array} and a \term{simple bit array}, respectively.

%% 17.4.0 5
\funref{bit} and \funref{sbit}, unlike \funref{char} and \funref{schar},
allow the first argument to be an \term{array} of any \term{rank}.

\endcom

%%% ========== BIT-AND
%%% ========== BIT-ANDC1
%%% ========== BIT-ANDC2
%%% ========== BIT-EQV
%%% ========== BIT-IOR
%%% ========== BIT-NAND
%%% ========== BIT-NOR
%%% ========== BIT-NOT
%%% ========== BIT-ORC1
%%% ========== BIT-ORC2
%%% ========== BIT-XOR
\begincom{bit-and, bit-andc1, bit-andc2, bit-eqv,
	  bit-ior, bit-nand, bit-nor, bit-not, bit-orc1, bit-orc2, bit-xor}\ftype{Function}

\label Syntax::

\DefunMultiWithValues {bit-array1 bit-array2 {\opt} opt-arg} {resulting-bit-array}
  {\entry{bit-and}
   \entry{bit-andc1}
   \entry{bit-andc2}
   \entry{bit-eqv}
   \entry{bit-ior}
   \entry{bit-nand}
   \entry{bit-nor}
   \entry{bit-orc1}
   \entry{bit-orc2}
   \entry{bit-xor}}

\DefunWithValues bit-not {bit-array {\opt} opt-arg} {resulting-bit-array}

\label Arguments and Values::

\param{bit-array}, \param{bit-array1}, \param{bit-array2}---a \term{bit array}.

\param{Opt-arg}---a \term{bit array}, or \t, or \nil.
 \Default{\nil}

%!!! This needs work. -kmp 24-May-91
\param{Bit-array}, \param{bit-array1}, \param{bit-array2}, and \param{opt-arg}
(if an \term{array}) must all be of the same \term{rank} and \term{dimensions}.  

\param{resulting-bit-array}---a \term{bit array}.
                 
\label Description::

%% 17.4.0 8
These functions perform 
bit-wise logical operations on \param{bit-array1} and \param{bit-array2}
and return an \term{array} 
of matching \term{rank} and \term{dimensions},
such that any given bit of the result
is produced by operating on corresponding bits from each of the arguments.

%% 17.4.0 11
In the case of \funref{bit-not}, an \term{array}
of \term{rank} and \term{dimensions} matching \param{bit-array}
is returned that contains a copy of \param{bit-array} 
with all the bits inverted.

%% 17.4.0 9
%% 17.4.0 12
If \param{opt-arg} is of type \f{(array bit)} the contents of the 
result are destructively placed into \param{opt-arg}.
If \param{opt-arg} is the symbol \t,
\param{bit-array} or \param{bit-array1} is replaced with the result; 
if \param{opt-arg} is \nil\ or omitted, a new \term{array} is created
to contain the result.  

%% 17.4.0 10
\Thenextfigure\ indicates the logical operation
performed by each of the \term{functions}.

\boxfig 
{\dimen0=2pc
\tabskip 2\dimen0
\halign to \hsize {#\hfil\tabskip \dimen0&#\hfil\tabskip 0pt plus 1fil\cr
\noalign{\vskip -9pt}
{\bf Function}&{\bf Operation}\cr
\noalign{\vskip 2pt\hrule\vskip 2pt}
\cr
\funref{bit-and}&and\cr
\funref{bit-eqv}&equivalence (exclusive nor)\cr
\funref{bit-not}&complement\cr
\funref{bit-ior}&inclusive or\cr
\funref{bit-xor}&exclusive or\cr
\funref{bit-nand}&complement of \param{bit-array1} and \param{bit-array2}\cr
\funref{bit-nor}&complement of \param{bit-array1} or \param{bit-array2}\cr
\funref{bit-andc1}&and complement of \param{bit-array1} with \param{bit-array2}\cr
\funref{bit-andc2}&and \param{bit-array1} with complement of \param{bit-array2}\cr
\funref{bit-orc1}&or complement of \param{bit-array1} with \param{bit-array2}\cr
\funref{bit-orc2}&or \param{bit-array1} with complement of \param{bit-array2}\cr
\noalign{\vskip -9pt}
\caption{Bit-wise Logical Operations on Bit Arrays}\cr
}}
\endfig

\label Examples::
\code
 (bit-and (setq ba #*11101010) #*01101011) \EV #*01101010
 (bit-and #*1100 #*1010) \EV #*1000      
 (bit-andc1 #*1100 #*1010) \EV #*0010
 (setq rba (bit-andc2 ba #*00110011 t)) \EV #*11001000
 (eq rba ba) \EV \term{true}
 (bit-not (setq ba #*11101010)) \EV #*00010101
 (setq rba (bit-not ba 
                     (setq tba (make-array 8 
                                           :element-type 'bit))))
\EV #*00010101
 (equal rba tba) \EV \term{true}
 (bit-xor #*1100 #*1010) \EV #*0110
\endcode

\label Affected By:\None.

\label Exceptional Situations:\None.

\label See Also::

\funref{lognot}, \funref{logand}
                
\label Notes:\None.

\endcom

%%% ========== BIT-VECTOR-P
\begincom{bit-vector-p}\ftype{Function}

\label Syntax::

\DefunWithValues bit-vector-p {object} {generalized-boolean}

\label Arguments and Values::

\param{object}---an \term{object}.

\param{generalized-boolean}---a \term{generalized boolean}.

\label Description::

%% 6.2.2 19
\TypePredicate{object}{bit-vector}

\label Examples::

\code
 (bit-vector-p (make-array 6 
                           :element-type 'bit 
                           :fill-pointer t)) \EV \term{true}
 (bit-vector-p #*) \EV \term{true}
 (bit-vector-p (make-array 6)) \EV \term{false}
\endcode

\label Affected By:\None.

\label Exceptional Situations:\None!

\label See Also::                  

\funref{typep}

\label Notes::

\code
 (bit-vector-p \param{object}) \EQ (typep \param{object} 'bit-vector)
\endcode

\endcom

%%% ========== SIMPLE-BIT-VECTOR-P
\begincom{simple-bit-vector-p}\ftype{Function}

\label Syntax::

\DefunWithValues simple-bit-vector-p {object} {generalized-boolean}

\label Arguments and Values::             

\param{object}---an \term{object}.

\param{generalized-boolean}---a \term{generalized boolean}.

\label Description::

%% 6.2.2 23
\TypePredicate{object}{simple-bit-vector}

\label Examples::           
\code
 (simple-bit-vector-p (make-array 6)) \EV \term{false}
 (simple-bit-vector-p #*) \EV \term{true}
\endcode


\label Side Effects:\None.

\label Affected By:\None.

\label Exceptional Situations:\None!

\label See Also:: 

\funref{simple-vector-p}

\label Notes::
\code
 (simple-bit-vector-p \param{object}) \EQ (typep \param{object} 'simple-bit-vector)
\endcode

\endcom

